\section{Related Work}
En ~\cite{compositeRetrival} se sugieren diferentes algoritmos para hallar una solución al problema. Produce and Choose es uno de ellos y en el que se encuetra basado este artículo. En el artículo original se mencionan otras dos alternativas más para la solución de este tipo de problemas, una basada en técnicas de clustering y otra en programación lineal. Es por las los resultados obtenidos de las ejecuciones con PAC que se decidió utilizarlo para realizar los cambios propuestos.\\
La estructura del algoritmo permite optimizar y agregar mejoras para obtener mejores resultados. Se implementó otra heurística con un enfoque diferente como fue una de tipo golosa pero por los resultados obtenidos y el tiempo de ejecución en este trabajo nos enfocaremos unicamente en el algoritmo PAC y las mejoras que se realizaron en él al añadirle búsquedas locales.\\
En la etapa de producción de los bundles puede ocurrir que los mismos no sean los óptimos ya que no se trata de algoritmos exactos. Una vez finalizada la etapa de producción y a diferencia de las soluciones anteriores, se intenta mejorar aquellos bundles que por un tema de ordenamiento y elección de lo items no resultaron siendo mejores. El tema de utilizar el resto de los items del universo que no fueron escogidos para formar parte de la solución final en pos de mejorar la solución, modificando los bundles ya generados, es clave para las mejoras propuestas.
\subsection{Data Model}\label{body-data-model}
El modelo de datos de la instancia de \textit{articulo italianos} contiene las entidades: artículos, autores, venues, affiliations y topics. Los artículos están etiquetados con un topic profile que representa el porcentaje de cada tópico encontrado en el artículo, así por ejemplo el paper \textit{paper1} esta catalogado con los siguientes tópicos: \textit{50\% topic1, 50\% topic2}. A partir del topic profile de cada uno de los artículos se pudo definir la noción de similitud entre los artículos.\\
Como también es interesante hacer consultas sobre los autores, se necesitaba que éstos también tengan un perfil el cuál no se encontraba en la base de datos. Con el objetivo de no depender de ninguna otra fuente se utilizaron los perfiles de los artículos para lograr el objetivo y de de esta manera definir su similitud. En orden de lograr un perfil, para cada autor se tomaron todos los papers en los cuales figura como autor y se sumaron los porcentajes de cada uno de los tópicos y luego normalizaron los valores para que tomen valore válidos (entre 0 y 1). Si bien no es cierto que el perfil del autor es que aquí se calcula ya que solo contiene información acotado, pero a medida que la base de datos se complete esta información será cada vez más precisa.\\
Utilizando la misma técnica anteriormente descripta se puede obtener el perfil del resto de los objetos (Universidad, venue).\\
Para generar los resultados de \textit{composite} se tiene un conjunto de objetos bibliográficos $I$ que son unívocamente identificado y contienen un conjunto de atributos y una función de similitud entre los objetos $ s: I \times I \rightarrow [0;1]$. En este trabajo la función de similitud se definió a partir del coseno del vector del topic profile.\\
\subsection{Problem Statement}
Formalmente el problema consiste en dado un conjunto de items $ I=\left\{i_1 \ldots i_n\right\} $, una función de similitud $ s(u,v) $ para cada par $ (u,v) \in IxI $ un atributo complementario $\alpha$, una función budget, un presupesto y un entero k se debe hallar $ S=\left\{S_1 \ldots S_k\right\} $ que maximice:(funcion objetivo).

\section{Produce and Choose}
PAC consiste en dos partes: primero en generar bundles válidos y luego seleccionar los k mejores que formarán parte de la solución. Para la parte de generación del bundle se utilizaron las estrategias de clusterización hierarchical y bobo. Para la parte de selección se utiliza un algoritmo goloso.\\
Para la instancia de los items bibliográficos se tuvo que optimizar el algoritmo de clusterización jerárquico propuesto en 1 por la cantidad de elementos que de la base de datos.\\
 
\section{Tabu Search}
Se explora un conjunto de soluciones para encontrar una solución más cohesiva. En esta implementación de la búsqueda tabú se reemplaza el item de menor similitud del centroide del bundle menos cohesivo de la solución por algún ítem que no pertenezca a la solución con mayor similitud al centroide. El item reemplazado se agrega a la lista de items tabú.