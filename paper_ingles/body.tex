\section{Related Work}
En ~\cite{compositeRetrival} se sugieren diferentes algoritmos para hallar una solución al problema, entre ellos se encuentra Produce and Choose. Por la eficiencia de los resultados obtenidos con PAC en aquel artículo, se decidió utilizarlo en este. Además la estructura del algoritmo permite optimizar y agregar mejoras para obtener un resultado más óptimo. Se realizaron implementaciones de otras heurísticas, pero por los resultados obtenidos y el tiempo de ejecución en este trabajo nos enfocaremos unicamente en el algoritmo PAC y la mejora que se hizo a través de agregarle al algoritmo búsquedas locales.

\subsection{Data Model}\label{body-data-model}
El modelo de datos de la instancia de \textit{articulo italianos} contiene las entidades: artículos, autores, venues, affiliations y topics. Los artículos estan etiquetados con el topic profile que es un porcentaje de cada tópico encontrado en el artículo, así por ejemplo el paper \textit{saraza} esta catalogado con los siguientes tópicos: \textit{ejemplo}.\\
Gracias al topic profile se pudo definir la similitud entre los artículos. Interpolando el topic profile de los articulos de un autor se puede obtener el topic profile del mismo y de esta manera poder realizar la similitud entre autores. Continuando con esta interpolación se puede obtener el topic profile del resto de los objetos (Universidad, venue)\\
Para generar los resultados de \textit{composite} se tiene un conjunto de objetos bibliográficos $I$ que son unívocamente identificado y contienen un conjunto de atributos y una función de similitud entre los objetos $ s: I \times I \rightarrow [0;1]$. En este trabajo la función de similitud se definió a partir del coseno del vector del topic profile.\\
\subsection{Problem Statement}
Formalmente el problema consiste en dado un conjunto de items $ I=\left\{i_1 \ldots i_n\right\} $, una función de similitud $ s(u,v) $ para cada par $ (u,v) \in IxI $ un atributo complementario $\alpha$, una función budget, un presupesto y un entero k se debe hallar $ S=\left\{S_1 \ldots S_k\right\} $ que maximice:(funcion objetivo).

\section{Algorithm}
\subsection{Produce and Choose}
Como se mencionó anteriormente el alogoritmo con el que se obtuvo los mejores resultados fue PAC. PAC consiste en dos partes: primero en generar un conjunto de bundles válidos y luego seleccionar los k mejores que formarán parte de la solución. Para la parte de generación de bundles se plantearon los métodos de clusterización \texttt{Efficient C-HAC} y \texttt{k-BOBO}.\\
\texttt{Efficient C-HAC} es un 

 Para la parte de selección se utiliza un algoritmo goloso.\\
Para la instancia de los items bibliográficos se tuvo que optimizar el algoritmo de clusterización jerárquico propuesto en 1 por la cantidad de elementos que de la base de datos.\\
 
\subsection{Tabu Search}
Se explora un conjunto de soluciones para encontrar una solución más cohesiva. En esta implementación de la búsqueda tabú se reemplaza el item de menor similitud del centroide del bundle menos cohesivo de la solución por algún ítem que no pertenezca a la solución con mayor similitud al centroide. El item reemplazado se agrega a la lista de items tabú.