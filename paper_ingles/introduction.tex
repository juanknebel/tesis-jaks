
\section{Introduction}
In conventional search query is entered and waits for a collection of items as a result. The user expects that the items match the search criteria that have been chosen. What happens in general is that several elements of the universe match each other with different degrees of relevance and an ordered list is used by the most search engines out there to show results. The ranked results are obtain by usinig a logic representation of the elements, this includes all the neccesary metadata to operate over theirs. The disadvantage of the previous model is that the similarity between the query and metada is the only criteria used, leaving behaind the existing relation between elements. Turning the process into a tedious and repetitive task, forcing the user to change the original query and explore another collection of results until you finds the desired items.\\
Article \textbf{Composite Retrieval of Diverse and Complementary Bundles}\cite{compositeRetrival} intends to present groups of items in a list rather than a vertical view of the same elements. Internally every item belonging to a group must be related with each other under the chosen similarity and the list should be sorted logically in order that one o more sets meet the user expectations. Thus it is not needed a new intervention for rethink the query and will enhance the user experience.\\
Lets take as an example the organization of a trip to a certain city. Typically requires multiple search in different engine in order to gather all the information of the different destinations to visit. These inclueds the geographical distance, the price of the attractions, the activities to realize or read the comments about the selected destinations.\\
In a typical search results are an extensive sorted list under the relevance of the query and indiscriminately mixing different solutions required by the user. Such solutions do not provide answers that relate the criterion sought with the other elements of the resulting list.\\
Another example is when a customer from an online music store likes to hear music from all around the world. He has a limited budget of \$70 and he is not interested in any genre of music particularly, but he wants to buy a set of music belonging to the same genre. When the customer enters the following search pattern \textquotedblleft Classic Rock \& Roll  \textquotedblright\ will get a list similar to the following:
\begin{itemize}
  \item Physical Graffiti - Led Zeppelin
  \item Led Zeppelin - Led Zeppelin
  \item It's Hard - The Who
  \item Perfect Strangers - Deep Purple
  \item El Cielo Puede Esperar - Attaque 77
  \item Wheels of Fire - Cream
  \item Confesiones de Invierno - Sui Generis
  \item The White Album - The Beatles
  \item Innuendo - Queen
  \item Sticky Fingers - The Rolling Stones
  \item Kamikaze - Luis Alberto Spinetta
\end{itemize}
%desde aca traduce hk
From the obtained list, the user  must choose those disks of his preference, with the corresponding possibility of choosing more than one disk from the same origin.  Second, from the list, manual addition and elimination  must be done by the user in the case that the disk cost goes too far from the scheduled  budget. Third, not necessarily the partial disk  group choise will be  the best  to optimize his budget and at the same time the different disks origin.\\
For this kind of searches, the proposed solution is focused on those consults  about the  requirements  to obtain a group of elements which are in relationship as an answer. A clustering result may be done, but in the traditional procedures the group is done by the similitude between items. In the disk example, with a traditional clustering, where the similitude may be the music gender, may be a lot of clusters will be generates as much as disk gender exist and in each cluster all the disk of that gender will be found .  Once the result is obtained all the clusters must be explored in order to choose the expected disks.\\
But if you use the mentioned procedures in \textit{``Composite Retrieval of Diverse and Complementary Bundles''} the obtained results will be in accordance with the budget and each one of the items included in the bundle will be complimentary between them, in such a way that the user  have the option to choose each bundle of the result and can be sure that the selected one obeys the early objective, that belongs to the same musical gender and that there is variety in the choice.\\
If in the disk market example, the complementary attribute of showing the band origin and with a \$ 70 budget is set, a possible solution should be:
\begin{itemize}
  \item Bundle 1:
  \begin{itemize}
    \item Physical Graffiti - Led Zeppelin (Inglaterra) \$20
    \item After chabón - Sumo (Argentina) \$20
    \item Back in Black - AC/DC (Estados Unidos) \$20
  \end{itemize}
  \item Bundle 2:
  \begin{itemize}
    \item Natty Dread - Bob Marley (Jamaica) \$30
    \item El ritual de la Banana - Los Pericos (Argentina) \$15
    \item Labour of Love - UB40 (Inglaterra) \$15
  \end{itemize}
	  \item Bundle 3:
  \begin{itemize}
    \item Ramones - Ramones (Estados Unidos) \$17
    \item El Cielo Puede Esperar - Attaque 77 (Argentina) \$17
    \item Sandinista! - The Clash (Inglaterra) \$15
		\item Upstyledown - 28 Days (Australia) \$15
  \end{itemize}
\end{itemize}
In the abovementioned examples,or in similar searching cases, the idea is that the user receive bundle groups which apply to the following properties: 
\begin{itemize}
  \item \textbf{Cover}: Maximize the number of elements in the bundle.
  \item \textbf{Compativility}: The elements of the bundle should be similar.
  \item \textbf{Validity}: The total cost of the elements of the bundle should not exceed the budget.
  \item \textbf{Diversity}: Each one of the bundles must be different.
\end{itemize}
%hasta aca traduce hk
\section{Motivation}
Consideramos puede ser de utilidad para instancias bibliográficas que el resultado de una búsqueda este compuesto por bundles de artículos complementarios, sujetos a una cantidad máxima de los mismos.
Lo que permite que el usuario pueda explorar items bibliográficos como libros, editoriales o autores diversos y acotados por el criterio que elija. Por ejemplo si un usuario esta interesado en un tema especifico entonces uno de los bundles puede que contenga el conjunto de libros que lo satisfaga, esto se da porque el contenido de bundles es de objetos similares pero con atributo que lo diferencia. Del ejemplo se puede establecer que el atributo diferencial sean los autores entonces el bundle le otorga al usuario un amplitud del tema ya que tiene diversidad porque los libros son de distintos autores y a la vez es cohesivo porque esos libros son similares. De esto modo al usuario se le simplifica el proceso de exploración ya que seguramente uno de los bundles contiene los objetos que satisfacen su necesidad. Por este tipo de escenarios nos pareció que este tipo de information retrival puede ser muy util para los objetos bibliograficos.   \\
La base de datos con la que se trabajo en este artículo es la proporcionada por "Data-Driven Journey through Software Engineering Research" que contiene artículos relacionados con la ingeniería de software presentados en diferentes conferencias entre los años 1975 y 2011 catalogados por autores, tópicos, venues y afiliaciones. Sobre esta base se realizaron diferentes consultas para las que se debió definir la similitud entre los items, el atributo complementario y la cota por bundle. Por ejemplo una de las consultas realizadas 'Artículos de diferentes conferencias'
en la que el resultado esperado consiste de una lista de bundles en la que cada bundle contiene artículos similares dictados en distintas conferencia. La similitud entre los artículos se definió por los tópicos, de esta consulta se obtuvieron los siguientes resultados:\\
(aca va extraccion del resultado obtenido)