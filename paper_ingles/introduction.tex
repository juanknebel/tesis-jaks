
\section{Introduction}
In conventional search query is entered and waits for a collection of items as a result. The user expects that the items match the search criteria that have been chosen. What happens in general is that several elements of the universe match each other with different degrees of relevance and an ordered list is used by the most search engines out there to show results. The ranked results are obtain by usinig a logic representation of the elements, this includes all the neccesary metadata to operate over theirs. The disadvantage of the previous model is that the similarity between the query and metada is the only criteria used, leaving behaind the existing relation between elements. Turning the process into a tedious and repetitive task, forcing the user to change the original query and explore another collection of results until you finds the desired items.\\
Article \textbf{Composite Retrieval of Diverse and Complementary Bundles}\cite{compositeRetrival} intends to present groups of items in a list rather than a vertical view of the same elements. Internally every item belonging to a group must be related with each other under the chosen similarity and the list should be sorted logically in order that one o more sets meet the user expectations. Thus it is not needed a new intervention for rethink the query and will enhance the user experience.\\
Lets take as an example the organization of a trip to a certain city. Typically requires multiple search in different engine in order to gather all the information of the different destinations to visit. These inclueds the geographical distance, the price of the attractions, the activities to realize or read the comments about the selected destinations.\\
In a typical search results are an extensive sorted list under the relevance of the query and indiscriminately mixing different solutions required by the user. Such solutions do not provide answers that relate the criterion sought with the other elements of the resulting list.\\
Another example is when a customer from an online music store likes to hear music from all around the world. He has a limited budget of \$70 and he is not interested in any genre of music particularly, but he wants to buy a set of music belonging to the same genre. When the customer enters the following search pattern \textquotedblleft Classic Rock \& Roll  \textquotedblright\ will get a list similar to the following:
\begin{itemize}
  \item Physical Graffiti - Led Zeppelin
  \item Led Zeppelin - Led Zeppelin
  \item It's Hard - The Who
  \item Perfect Strangers - Deep Purple
  \item El Cielo Puede Esperar - Attaque 77
  \item Wheels of Fire - Cream
  \item Confesiones de Invierno - Sui Generis
  \item The White Album - The Beatles
  \item Innuendo - Queen
  \item Sticky Fingers - The Rolling Stones
  \item Kamikaze - Luis Alberto Spinetta
\end{itemize}
%desde aca traduce hk
From the obtained list, the user  must choose those disks of his preference, with the corresponding possibility of choosing more than one disk from the same origin.  Second, from the list, manual addition and elimination  must be done by the user in the case that the disk cost goes too far from the scheduled  budget. Third, not necessarily the partial disk  group choise will be  the best  to optimize his budget and at the same time the different disks origin.\\
For this kind of searches, the proposed solution is focused on those consults  about the  requirements  to obtain a group of elements which are in relationship as an answer. A clustering result may be done, but in the traditional procedures the group is done by the similitude between items. In the disk example, with a traditional clustering, where the similitude may be the music gender, may be a lot of clusters will be generates as much as disk gender exist and in each cluster all the disk of that gender will be found .  Once the result is obtained all the clusters must be explored in order to choose the expected disks.\\
But if you use the mentioned procedures in \textit{``Composite Retrieval of Diverse and Complementary Bundles''} the obtained results will be in accordance with the budget and each one of the items included in the bundle will be complimentary between them, in such a way that the user  have the option to choose each bundle of the result and can be sure that the selected one obeys the early objective, that belongs to the same musical gender and that there is variety in the choice.\\
If in the disk market example, the complementary attribute of showing the band origin and with a \$ 70 budget is set, a possible solution should be:
\begin{itemize}
  \item Bundle 1:
  \begin{itemize}
    \item Physical Graffiti - Led Zeppelin (Inglaterra) \$20
    \item After chabón - Sumo (Argentina) \$20
    \item Back in Black - AC/DC (Estados Unidos) \$20
  \end{itemize}
  \item Bundle 2:
  \begin{itemize}
    \item Natty Dread - Bob Marley (Jamaica) \$30
    \item El ritual de la Banana - Los Pericos (Argentina) \$15
    \item Labour of Love - UB40 (Inglaterra) \$15
  \end{itemize}
	  \item Bundle 3:
  \begin{itemize}
    \item Ramones - Ramones (Estados Unidos) \$17
    \item El Cielo Puede Esperar - Attaque 77 (Argentina) \$17
    \item Sandinista! - The Clash (Inglaterra) \$15
		\item Upstyledown - 28 Days (Australia) \$15
  \end{itemize}
\end{itemize}
In the abovementioned examples,or in similar searching cases, the idea is that the user receive bundle groups which apply to the following properties: 
\begin{itemize}
  \item \textbf{Cover}: Maximize the number of elements in the bundle.
  \item \textbf{Compativility}: The elements of the bundle should be similar.
  \item \textbf{Validity}: The total cost of the elements of the bundle should not exceed the budget.
  \item \textbf{Diversity}: Each one of the bundles must be different.
\end{itemize}
%hasta aca traduce hk
\section{Motivation}
We consider that it can be useful for bibliographic relations, that the search result could be conformed by bundles of complimentary articles but not overpassing a maximum quantity of them. It permits that users can explore bibliographic items like books, newsletters or different authors and limited for the fixed criterion selected. For example whether a user is interested in some specific subject, then it is possible that one of the bundles may contain the group of books that he is looking for, and this is because the bundle con is composed of similar objects but with attributes which make differences.
From the example it can be determined that the differential attribute can the authors, then the bundle brings to the user a clear and more wide idea of the subject because the books come from different authors and at the same time is cohesive because those books are similar. So the exploration process is easier for the user because surely one of the bundles contains the required subjects so as to satisfy the requirements. Accordingly to these scenarios we think that this kind of retrieval information can be very useful for the bibliographic subjects.\\
The data base used for this presentation is \textquotedblleft Data-Driven Journey through Software Engineering Research\textquotedblright, which contains articles related with soft engineering and exposed in different conferences between 1975 and 2011 years and cataloged as authors, topics, venues and affiliations. Considering this situation different consults were made and it was necessary to define the similitude between items, the complementary attribute and a limit for bundle.
For example, one of of the consults \textquoteleft Articles from different conferences\textquoteright, in which the expected result was composed for a bundle list and each bundle contains similar articles presented in different conferences. The similitude between the articles was defined for the topics and from this consult following results were obtained:\\
(aca va extraccion del resultado obtenido)