
\section{Introduction}
In conventional search query is entered and waits for a collection of items as a result. The user expects that the items match the search criteria that have been chosen. What happens in general is that several elements of the universe match each other with different degrees of relevance and an ordered list is used by the most search engines out there to show results. The ranked results are obtain by usinig a logic representation of the elements, this includes all the neccesary metadata to operate over theirs. The disadvantage of the previous model is that the similarity between the query and metada is the only criteria used, leaving behaind the existing relation between elements. Turning the process into a tedious and repetitive task, forcing the user to change the original query and explore another collection of results until you finds the desired items.\\
Article \textbf{Composite Retrieval of Diverse and Complementary Bundles}\cite{compositeRetrival} intends to present groups of items in a list rather than a vertical view of the same elements. Internally every item belonging to a group must be related with each other under the chosen similarity and the list should be sorted logically in order that one o more sets meet the user expectations. Thus it is not needed a new intervention for rethink the query and will enhance the user experience.\\
Lets take as an example the organization of a trip to a certain city. Typically requires multiple search in different engine in order to gather all the information of the different destinations to visit. These inclueds the geographical distance, the price of the attractions, the activities to realize or read the comments about the selected destinations.\\
In a typical search results are an extensive sorted list under the relevance of the query and indiscriminately mixing different solutions required by the user. Such solutions do not provide answers that relate the criterion sought with the other elements of the resulting list.\\
Another example is when a customer from an online music store likes to hear music from all around the world. He has a limited budget of \$70 and he is not interested in any genre of music particularly, but he wants to buy a set of music belonging to the same genre. When the customer enters the following search pattern \textquotedblleft Classic Rock \& Roll  \textquotedblright\ will get a list similar to the following:
\begin{itemize}
  \item Physical Graffiti - Led Zeppelin
  \item Led Zeppelin - Led Zeppelin
  \item It's Hard - The Who
  \item Perfect Strangers - Deep Purple
  \item El Cielo Puede Esperar - Attaque 77
  \item Wheels of Fire - Cream
  \item Confesiones de Invierno - Sui Generis
  \item The White Album - The Beatles
  \item Innuendo - Queen
  \item Sticky Fingers - The Rolling Stones
  \item Kamikaze - Luis Alberto Spinetta
\end{itemize}
%desde aca traduce hk
De la lista obtenida el usuario deberá seleccionar aquellos discos que sean de su interes con el posible error de elegir más de un disco del mismo origen. Segundo, deberá ir agregando y eliminando de su lista manualmente en el caso que la elección de un disco superase el presupuesto que él posee. Tercero, no necesariamente elegirá el mejor subconjunto de discos que maximice su presupuesto y a su vez el origen de los discos sean distintos.\\
Para este tipo de búsquedas la solución que se propone está pensada para aquellas consultas que requieren obtener un conjunto de elementos que se relacionan como respuesta. Se podría realizar una clusterización de los resultados pero, en las técnicas tradicionales la agrupación se hace por la similitud entre ítems. En el ejemplo de los discos con una clusterización tradicional, donde la similitud sea el género musical, seguramente se generen tantos cluster como géneros de discos existan y en cada cluster se encontrarán todos los discos de ese género. Una vez obtenido el resultado se deberá explorar todos los clusters para elegir los discos.\\
En cambio si se aplicase las técnicas mencionadas en \textit{``Composite Retrieval of Diverse and Complementary Bundles''} las soluciones obtenidas se ajustarían al presupuesto y cada uno de los ítems dentro del bundle (es el nombre que se le da al agrupamiento de ítems) sean complementarios entre sí, de modo tal que el usuario pordrá optar por cualquier bundle de la solución y estar seguro que su elección cumple con su objetivo inicial, que pertenece a un mismo género musical y exista variedad en la elección.\\
Si en el ejemplo de la tienda de discos se establece la complementariedad del atributo que refleja el origen de la banda y se establece un presupuesto de \$70 una solución posible sería:
\begin{itemize}
  \item Bundle 1:
  \begin{itemize}
    \item Physical Graffiti - Led Zeppelin (Inglaterra) \$20
    \item After chabón - Sumo (Argentina) \$20
    \item Back in Black - AC/DC (Estados Unidos) \$20
  \end{itemize}
  \item Bundle 2:
  \begin{itemize}
    \item Natty Dread - Bob Marley (Jamaica) \$30
    \item El ritual de la Banana - Los Pericos (Argentina) \$15
    \item Labour of Love - UB40 (Inglaterra) \$15
  \end{itemize}
	  \item Bundle 3:
  \begin{itemize}
    \item Ramones - Ramones (Estados Unidos) \$17
    \item El Cielo Puede Esperar - Attaque 77 (Argentina) \$17
    \item Sandinista! - The Clash (Inglaterra) \$15
		\item Upstyledown - 28 Days (Australia) \$15
  \end{itemize}
\end{itemize}
Lo que se quiere lograr en los ejemplos descriptos y en cualquier otro problema similar de búsquedas es otorgarle al usuario un conjunto de bundles que cumplan siempre con las siguientes propiedades: 
\begin{itemize}
  \item \textbf{Cubrimiento}: Maximizar la cantidad de elementos en el bundle.
  \item \textbf{Compatibilidad}: Los elementos del bundle deben ser similares.
  \item \textbf{Validez}: El costo total de los elementos del bundle no debe superar el presupuesto.
  \item \textbf{Diversidad}: Los bundles entre si deben ser diversos.
\end{itemize}
%hasta aca traduce hk
\section{Motivation}
Consideramos que puede ser de utilidad que para instancias bibliográficas el resultado de una búsqueda este compuesto por bundles de items complementarios, sujetos a un presupuesto. 
Lo que permite que el usuario pueda explorar items bibliográficos como libros, editoriales o autores diversos y acotados por algún criterio.\\
La base de datos con la que se trabajo en este artículo es la proporcionada por "Data-Driven Journey through Software Engineering Research" que contiene artículos relacionados con la ingeniería de software presentados en diferentes conferencias entre los años 1975 y 2011 catalogados por autores, tópicos, venues y afiliaciones. Sobre esta base se realizaron diferentes consultas para las que se debió definir la similitud entre los items, el atributo complementario y la cota por bundle. Por ejemplo una de las consultas realizadas 'Artículos de diferentes conferencias'
en la que el resultado esperado consiste de una lista de bundles en la que cada bundle contiene artículos similares dictados en distintas conferencia. La similitud entre los artículos se definió por los tópicos, de esta consulta se obtuvieron los siguientes resultados: