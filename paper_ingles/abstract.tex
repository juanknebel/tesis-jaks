\begin{abstract}
Traditional searchs usually offer solutions that only consider a unique attribute of the elements and not the relation between them and the rest of the universe. These searchs offer a ranking list of results which are related to the used criteria. Many times you need to rethink the original query to accomplish the right solution.\\
As a response to the last behavior appears \textbf{Composite Retrieval of Diverse and Complementary Bundles}\cite{compositeRetrival}. Its objective is to group elements into bundles, in which the items are related each other under the criteria of similarity and also they are complementary. In that way the bundles should satisfy the users expectations without the needing of a new intervention, achieving the improve of the searching experience.\\
In this paper we applied the ideas previously developed in ~\cite{compositeRetrival} to the resolution of queries over a database of scientific articles belonging to Software Engenieering ~\cite{dataDrive}. An example for the query \textquoteleft articles from different universities\textquoteright , the obtained result is a list of groups in which each one of them contains similar papers written in various universities.\\
Moreover we proposed changes to improve the algorithm’s complexity and  added new searching tecnichs pretending to polish the quality of the solutions.\\
\end{abstract}