\section{Related Work}
En ~\cite{compositeRetrival} se sugieren diferentes algoritmos para hallar una solución al problema, entre ellos se encuentra Produce and Choose que consiste en dos partes, primero en generar bundles válidos y luego seleccionar los mejores que formarán parte de la solución.
\subsection{Data Model}\label{body-data-model}
El modelo de datos de la instancia de \textit{articulo italianos} contiene las entidades: artículos, autores, venues, affiliations y topics. Los artículos estan etiquetados con el topic profile que es un porcentaje de cada tópico encontrado en el artículo, así por ejemplo el paper \textit{saraza} tiene \textit{ejemplo}. Para generar los resultados de \textit{composite} se tiene un conjunto de objetos bibliográficos $I$ que son unívocamente identificado y contienen un conjunto de atributos y una función de similitud entre los objetos $ s: I \times I \rightarrow [0;1]$. En este trabajo la función de similitud se definió a partir del coseno del vector del topic profile.\\
\subsection{Problem Statement}
Formalmente el problema consiste en dado un conjunto de items $ I=\left\{i_1 \ldots i_n\right\} $, una función de similitud $ s(u,v) $ para cada par $ (u,v) \in IxI $ un atributo complementario $\alpha$, una función budget, un presupesto y un entero k se debe hallar $ S=\left\{S_1 \ldots S_k\right\} $ que maximice:(funcion objetivo).

\section{Produce and Choose}
Como se explico anteriormente el algoritmo utilizado para hallar una solución es PAC. Este algoritmo consiste en dos etapas, la primera en producir un conjunto valido de bundles y luego seleccionar los k bundles de la solución. Para la instancia de los items bibliográficos se optimizó el algoritmo de clusterización jerárquico propuesto en 1.
 
\section{Tabu Search}
Se explora un conjunto de soluciones para encontrar una solución más cohesiva. En esta implementación de la búsqueda tabú se reemplaza el item de menor similitud del centroide del bundle menos cohesivo de la solución por algún ítem que no pertenezca a la solución con mayor similitud al centroide. El item reemplazado se agrega a la lista de items tabú.