\section{Related Work}
En ~\cite{compositeRetrival} se sugieren diferentes algoritmos para hallar una solución al problema, entre ellos se encuentra Produce and Choose que consiste en dos partes, primero en generar bundles válidos y luego seleccionar los mejores que formarán parte de la solución.\\
\subsection{Data Model}
Dado el conjunto de objetos $I$ y una función de similitud $ s: I \times I \rightarrow [0;1]$, cada objeto es unívocamente identificado y contiene un conjunto de atributos. La entrada puede pensarse como un grafo completo con peso en las aristas $G=(I,E,s)$ donde el peso del vértice $(u,v)$ es $s(u,v)$. Se define también la \textit{función de distancia} $d(u,v) = 1 - s(u,v)$ que también toma valores en el intervalo $[0;1]$.\\
En este trabajo en particular los objetos son papers presentados en diferentes conferencias entre los años 1975 y 2011, cada uno de ellos tiene asignado un perfil que está compuesto por los porcentajes del tópico del área al cuál hace referencia su contenido. Cada paper puede tener 100\% de un tema en particular como por ejemplo Visualizations o puede estar conformado de varios, 33\% Concurrency, 33\% Models, 34\% Databases. A los perfiles se lo puede interpretar como un vector de n posiciones, entonces se calcula el coseno del ángulo que forman uno a uno los vectores resultando ese valor en la simimlitud que un papaer tiene con otro. Estamos seguros que la similitud esta en el rango $[0:1]$ porque los perfiles de los papers están formados por porcentajes los cuáles siempre son mayores a 0 y la suma de sus componentes siempre será 1.
\subsection{Problem Statement}
La base de datos ~\cite{dataDrive} cuenta con cerca de 8000 papers 
\subsection{Problem Complexity}
La complejidad del problema
\section{Produce and Choose}
Explicar el algoritmo produce and choose y que para el produce se utilizo el Hierichal
