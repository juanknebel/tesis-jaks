\section{Related Work}
En ~\cite{compositeRetrival} se sugieren diferentes algoritmos para hallar una solución al problema, entre ellos se encuentra Produce and Choose que consiste en dos partes, primero en generar bundles válidos y luego seleccionar los mejores que formarán parte de la solución.
\subsection{Data Model}
El modelo de datos de la instancia de \textit{articulo italianos} contiene las entidades: artículos, autores, venues, affiliations y topics. Los artículos estan etiquetados con el topic profile que es un porcentaje de cada tópico encontrado en el artículo, así por ejemplo el paper \textit{saraza} tiene \textit{ejemplo}. Para generar los resultados de \textit{composite} se tiene un conjunto de objetos bibliográficos $I$ que son unívocamente identificado y contienen un conjunto de atributos y una función de similitud entre los objetos $ s: I \times I \rightarrow [0;1]$. En este trabajo la función de similitud se definió a partir del coseno del vector del topic profile.\\
\subsection{Problem Statement}
La base de datos ~\cite{dataDrive} cuenta con cerca de 8000 papers 
\subsection{Problem Complexity}
La complejidad del problema
\section{Produce and Choose}
La etapa de producción de bundles se realizó utilizando dos técnicas diferentes
