\section{Related Work}
En el artículo en cuál esta inspirado éste trabajo \texttt{Composite Retrieval of Diverse and Complementary Bundles}, se propusieron diferentes maneras de solucionar el problema usando técnicas de clusterización. La estrategía de producir y seleccionar (PAC) tenía dos varianetes una inspirada en k-means llamada BOBO y la restante basada en técnicas jerarquicas con restricción para el momento de la producción de bundles, en cambio en la selección de los bundles que luego formarían parte de la solución del problema se traducía a un problema de grafos (max edge subgraph). En este trabajo se toman éstas mismas ideas y se propuso aplicarles mejoras primero en la etapa de producción de bundles de los algoritmos jerárquicos, como fue generar una búsqueda tabú luego del proceso en pos de mejorar las bundles unitarios o las malas uniones de los bundles. En particular en el c-hac se modificaron las estructuras de datos usadas (colas de prioridad) mejorando la complejidad temporal permitiendo que se puedan ejecutar instancias mas grandes del problema y a la vez mejorar las uniones de los bundles. En la etapa de selección se presentaron dos estrategías distintas a las que menciona el paper que tienen en cuenta toda la posible solución final y no solo la selección actual.
\subsection{Data Model}
El modelo de datos
\subsection{Problem Statement}
El problema en forma formal
\subsection{Problem Complexity}
La complejidad del problema
\section{Produce and Choose}
Explicar el algoritmo produce and choose y que para el produce se utilizo el Hierichal
\subsubsection{Inline (In-text) Equations}
\subsubsection{Display Equations}
\subsection{Citations}
\subsection{Tables}
\subsection{Figures}
\subsection{Theorem-like Constructs}
\subsection*{A {\secit Caveat} for the \TeX\ Expert}
