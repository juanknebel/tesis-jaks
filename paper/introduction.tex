\section{Introduction}
\textbf{Information Retrieval} es la actividad de obtener información relevante de una inmensa colección de datos a partir de algún criterio. Criterios de lo más variados, desde el resultado de la final del mundial de fútbol, los libros de un autor y hasta el mail de la confirmación de una compra.\\
El proceso de búsqueda comienza cuando el usuario ingresa una consulta esperando que el buscador devuelva una colección de elementos que coincidan con el criterio de búsqueda elegido. En general lo que ocurre es que son varios los elementos del universo que concuerdan pero con grados de relevancia diferentes (ranking de resultados) que se utiliza para ordenar la colección de elementos devueltos. Para obtener el ranking de resultados los sistemas de IR trabajan con una representación lógica de los elementos que incluye los metadatos necesarios para operar sobre ellos. La desventaja de los ranking de resultados es que únicamente se compara la consulta de la búsqueda con los metadatos de los elementos, dejando de lado el análisis de los elementos entre sí y conviritiendo, en ocasiones, al proceso en una acción tediosa y repetitiva ya que el usuario deberá cambiar la consulta original y explorar la colección de elementos hasta lograr encontrar el o los elementos deseado.\\
En el  artículo \textbf{Composite Retrieval of Diverse and Complementary Bundles}\cite{compositeRetrival} se propone presentar una lista de grupos de elementos, en lugar de entregar una lista vertical de los mismos. Cada grupo deberá estar relacionado internamente bajo el criterio de similitud elegido y la lista ordenada de forma lógica con la finalidad de que uno o más conjuntos satisfagan las expectativas del usuario sin necesidad de una nueva intervención para refinar su búsqueda para lograr una mejor experiencia de búsqueda.\\
La finalidad de este trabajo es devolver los resultados de las búsquedas como plantea el artículo \textbf{Composite Retrieval of Diverse and Complementary Bundles} para ello se analizaron e implementaron los algoritmos de agrupamiento (o clustering) que realizan la tarea de agrupar en conjuntos disjuntos a elementos que pertenecen a una misma clase. Las dos técnicas más usadas son agrupamiento jerárquico y no jerárquico. La primera a su vez se puede dividir en dos tipos, aglomerativos donde todos los elementos comienzan como un cluster para luego mezclarse entre ellos y divisivos en el cual se comienza con un único grupo y se comienza a dividir. Para las decisiones de unir o dividir se usan medidas de similitud o disimilitud de los elementos del conjunto. Para la segunda técnica de clusterización se definen previamente cuales serán los grupos finales y se van asignado los demás elementos al grupo que correspondan. Además de las técnicas de clusterización, se desarrollaron heurísticas para buscar una solución mejor.\\