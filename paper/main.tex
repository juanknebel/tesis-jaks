% This is "sig-alternate.tex" V2.1 April 2013
% This file should be compiled with V2.5 of "sig-alternate.cls" May 2012
%
% This example file demonstrates the use of the 'sig-alternate.cls'
% V2.5 LaTeX2e document class file. It is for those submitting
% articles to ACM Conference Proceedings WHO DO NOT WISH TO
% STRICTLY ADHERE TO THE SIGS (PUBS-BOARD-ENDORSED) STYLE.
% The 'sig-alternate.cls' file will produce a similar-looking,
% albeit, 'tighter' paper resulting in, invariably, fewer pages.
%
% ----------------------------------------------------------------------------------------------------------------
% This .tex file (and associated .cls V2.5) produces:
%       1) The Permission Statement
%       2) The Conference (location) Info information
%       3) The Copyright Line with ACM data
%       4) NO page numbers
%
% as against the acm_proc_article-sp.cls file which
% DOES NOT produce 1) thru' 3) above.
%
% Using 'sig-alternate.cls' you have control, however, from within
% the source .tex file, over both the CopyrightYear
% (defaulted to 200X) and the ACM Copyright Data
% (defaulted to X-XXXXX-XX-X/XX/XX).
% e.g.
% \CopyrightYear{2007} will cause 2007 to appear in the copyright line.
% \crdata{0-12345-67-8/90/12} will cause 0-12345-67-8/90/12 to appear in the copyright line.
%
% ---------------------------------------------------------------------------------------------------------------
% This .tex source is an example which *does* use
% the .bib file (from which the .bbl file % is produced).
% REMEMBER HOWEVER: After having produced the .bbl file,
% and prior to final submission, you *NEED* to 'insert'
% your .bbl file into your source .tex file so as to provide
% ONE 'self-contained' source file.
%
% ================= IF YOU HAVE QUESTIONS =======================
% Questions regarding the SIGS styles, SIGS policies and
% procedures, Conferences etc. should be sent to
% Adrienne Griscti (griscti@acm.org)
%
% Technical questions _only_ to
% Gerald Murray (murray@hq.acm.org)
% ===============================================================
%
% For tracking purposes - this is V2.0 - May 2012

\documentclass{sig-alternate-05-2015}


\begin{document}

% Copyright
\setcopyright{acmcopyright}
%\setcopyright{acmlicensed}
%\setcopyright{rightsretained}
%\setcopyright{usgov}
%\setcopyright{usgovmixed}
%\setcopyright{cagov}
%\setcopyright{cagovmixed}


% DOI
\doi{10.475/123_4}

% ISBN
\isbn{123-4567-24-567/08/06}

%Conference
\conferenceinfo{PLDI '13}{June 16--19, 2013, Seattle, WA, USA}

\acmPrice{\$15.00}

%
% --- Author Metadata here ---
\conferenceinfo{WOODSTOCK}{'97 El Paso, Texas USA}
%\CopyrightYear{2007} % Allows default copyright year (20XX) to be over-ridden - IF NEED BE.
%\crdata{0-12345-67-8/90/01}  % Allows default copyright data (0-89791-88-6/97/05) to be over-ridden - IF NEED BE.
% --- End of Author Metadata ---

\title{Alternate {\ttlit ACM} SIG Proceedings Paper in LaTeX
Format\titlenote{(Produces the permission block, and
copyright information). For use with
SIG-ALTERNATE.CLS. Supported by ACM.}}
\subtitle{[Extended Abstract]
\titlenote{A full version of this paper is available as
\textit{Author's Guide to Preparing ACM SIG Proceedings Using
\LaTeX$2_\epsilon$\ and BibTeX} at
\texttt{www.acm.org/eaddress.htm}}}

\input{authors.tex}
\maketitle
\begin{abstract}
Las búsquedas tradicionales ofrecen soluciones que solo tiene en cuenta un solo atributo de los elementos y no la relación que éstos tienen con el resto del universo. Las suelen ofrecer una lista ordenada de resultados relacionadas con el criterio utilizado, ocasionando muchas veces reformular la consulta original para así lograr una solución adecuada al criterio de búsqueda.\\
Como respuesta a éste último comportamiento surge \textbf{Composite Retrieval} ~\cite{compositeRetrival}, su objetivo es agrupar elementos en bundles bajo un mismo atributo logrando, al mismo tiempo, que éstos sean complementarios entre sí por algún otro atributo definido previamente.\\
En este paper se tomaron las ideas ya desarrolladas previamente en ~\cite{compositeRetrival} y se aplicaron a la resolución de búsquedas sobre una base de datos de artículos científicos pertenecientes a la Ingeniería de Software ~\cite{dataDrive}. Más aún se propusieron cambios para mejorar la complejidad de los algoritmos y también se adicionaron nuevas técnicas de búsquedas pretendiendo refinar la calidad de las soluciones.\\
\end{abstract}
%
% The code below should be generated by the tool at
% http://dl.acm.org/ccs.cfm
% Please copy and paste the code instead of the example below. 
%
\begin{CCSXML}
<ccs2012>
<concept>
<concept_id>10002951.10003317.10003338.10003345</concept_id>
<concept_desc>Information systems~Information retrieval diversity</concept_desc>
<concept_significance>300</concept_significance>
</concept>
</ccs2012>
\end{CCSXML}

\ccsdesc[300]{Information systems~Information retrieval diversity}

%
% End generated code
%

%
%  Use this command to print the description
%
\printccsdesc
% We no longer use \terms command
%\terms{Theory}
\keywords{}

\section{Introduction}
In conventional search query is entered and waits for a collection of items as a result. The user expects that the items match the search criteria that have been chosen. What happens in general is that several elements of the universe match each other with different degrees of relevance and an ordered list is used by the most search engines out there to show results. The ranked results are obtain by usinig a logic representation of the elements, this includes all the neccesary metadata to operate over theirs. The disadvantage of the previous model is that the similarity between the query and metada is the only criteria used, leaving behaind the existing relation between elements. Turning the process into a tedious and repetitive task, forcing the user to change the original query and explore another collection of results until you finds the desired items.\\
Article \textbf{Composite Retrieval of Diverse and Complementary Bundles}\cite{compositeRetrival} intends to present groups of items in a list rather than a vertical view of the same elements. Internally every item belonging to a group must be related with each other under the chosen similarity and the list should be sorted logically in order that one o more sets meet the user expectations. Thus it is not needed a new intervention for rethink the query and will enhance the user experience.\\
Lets take as an example the organization of a trip to a certain city. Typically requires multiple search in different engine in order to gather all the information of the different destinations to visit. These inclueds the geographical distance, the price of the attractions, the activities to realize or read the comments about the selected destinations.\\
In a typical search results are an extensive sorted list under the relevance of the query and indiscriminately mixing different solutions required by the user. Such solutions do not provide answers that relate the criterion sought with the other elements of the resulting list.\\
Another example is when a customer from an online music store likes to hear music from all around the world. He has a limited budget of \$70 and he is not interested in any genre of music particularly, but he wants to buy a set of music belonging to the same genre. When the customer enters the following search pattern \textquotedblleft Classic Rock \& Roll  \textquotedblright\ will get a list similar to the following:
\begin{itemize}
  \item Physical Graffiti - Led Zeppelin
  \item Led Zeppelin - Led Zeppelin
  \item It's Hard - The Who
  \item Perfect Strangers - Deep Purple
  \item El Cielo Puede Esperar - Attaque 77
  \item Wheels of Fire - Cream
  \item Confesiones de Invierno - Sui Generis
  \item The White Album - The Beatles
  \item Innuendo - Queen
  \item Sticky Fingers - The Rolling Stones
  \item Kamikaze - Luis Alberto Spinetta
\end{itemize}
%desde aca traduce hk
De la lista obtenida el usuario deberá seleccionar aquellos discos que sean de su interes con el posible error de elegir más de un disco del mismo origen. Segundo, deberá ir agregando y eliminando de su lista manualmente en el caso que la elección de un disco superase el presupuesto que él posee. Tercero, no necesariamente elegirá el mejor subconjunto de discos que maximice su presupuesto y a su vez el origen de los discos sean distintos.\\
Para este tipo de búsquedas la solución que se propone está pensada para aquellas consultas que requieren obtener un conjunto de elementos que se relacionan como respuesta. Se podría realizar una clusterización de los resultados pero, en las técnicas tradicionales la agrupación se hace por la similitud entre ítems. En el ejemplo de los discos con una clusterización tradicional, donde la similitud sea el género musical, seguramente se generen tantos cluster como géneros de discos existan y en cada cluster se encontrarán todos los discos de ese género. Una vez obtenido el resultado se deberá explorar todos los clusters para elegir los discos.\\
En cambio si se aplicase las técnicas mencionadas en \textit{``Composite Retrieval of Diverse and Complementary Bundles''} las soluciones obtenidas se ajustarían al presupuesto y cada uno de los ítems dentro del bundle (es el nombre que se le da al agrupamiento de ítems) sean complementarios entre sí, de modo tal que el usuario pordrá optar por cualquier bundle de la solución y estar seguro que su elección cumple con su objetivo inicial, que pertenece a un mismo género musical y exista variedad en la elección.\\
Si en el ejemplo de la tienda de discos se establece la complementariedad del atributo que refleja el origen de la banda y se establece un presupuesto de \$70 una solución posible sería:
\begin{itemize}
  \item Bundle 1:
  \begin{itemize}
    \item Physical Graffiti - Led Zeppelin (Inglaterra) \$20
    \item After chabón - Sumo (Argentina) \$20
    \item Back in Black - AC/DC (Estados Unidos) \$20
  \end{itemize}
  \item Bundle 2:
  \begin{itemize}
    \item Natty Dread - Bob Marley (Jamaica) \$30
    \item El ritual de la Banana - Los Pericos (Argentina) \$15
    \item Labour of Love - UB40 (Inglaterra) \$15
  \end{itemize}
	  \item Bundle 3:
  \begin{itemize}
    \item Ramones - Ramones (Estados Unidos) \$17
    \item El Cielo Puede Esperar - Attaque 77 (Argentina) \$17
    \item Sandinista! - The Clash (Inglaterra) \$15
		\item Upstyledown - 28 Days (Australia) \$15
  \end{itemize}
\end{itemize}
Lo que se quiere lograr en los ejemplos descriptos y en cualquier otro problema similar de búsquedas es otorgarle al usuario un conjunto de bundles que cumplan siempre con las siguientes propiedades: 
\begin{itemize}
  \item \textbf{Cubrimiento}: Maximizar la cantidad de elementos en el bundle.
  \item \textbf{Compatibilidad}: Los elementos del bundle deben ser similares.
  \item \textbf{Validez}: El costo total de los elementos del bundle no debe superar el presupuesto.
  \item \textbf{Diversidad}: Los bundles entre si deben ser diversos.
\end{itemize}
%hasta aca traduce hk
\section{Motivation}
Consideramos que puede ser de utilidad que para instancias bibliográficas el resultado de una búsqueda este compuesto por bundles de items complementarios, sujetos a un presupuesto. 
Lo que permite que el usuario pueda explorar items bibliográficos como libros, editoriales o autores diversos y acotados por algún criterio. Por ejemplo si un usuario esta interesado en un tema especifico entonces uno de los bundles puede que contenga el conjunto de libros que lo satisfaga, esto se da porque el contenido de bundles es de objetos similares pero con atributo que lo diferencia. Del ejemplo se puede establecer que el atributo diferencial sean los autores entonces el bundle le otorga al usuario un amplitud del tema ya que tiene diversidad porque los libros son de distintos autores y a la vez es cohesivo porque esos libros son similares. De esto modo al usuario se le simplifica el proceso de exploración ya que seguramente uno de los bundles contiene los objetos que satisfacen su necesidad. Por este tipo de escenarios nos pareció que este tipo de information retrival puede ser muy util para los objetos bibliograficos.   \\
La base de datos con la que se trabajo en este artículo es la proporcionada por "Data-Driven Journey through Software Engineering Research" que contiene artículos relacionados con la ingeniería de software presentados en diferentes conferencias entre los años 1975 y 2011 catalogados por autores, tópicos, venues y afiliaciones. Sobre esta base se realizaron diferentes consultas para las que se debió definir la similitud entre los items, el atributo complementario y la cota por bundle. Por ejemplo una de las consultas realizadas 'Artículos de diferentes conferencias'
en la que el resultado esperado consiste de una lista de bundles en la que cada bundle contiene artículos similares dictados en distintas conferencia. La similitud entre los artículos se definió por los tópicos, de esta consulta se obtuvieron los siguientes resultados:\\
(aca va extraccion del resultado obtenido)
\section{Related Work}
\texttt{Composite Retrieval of Diverse and Complementary Bundles} es el artículo que propone esta forma de devolver los resultados. En el mismo sugieren distintos algoritmos para hallar una solución entre ellos se encuentra Produce and Choose que consiste en una primer fase de generar bundles y en la siguiente fase seleccionar los bundles de la solución. En el artículo para la fase de producción uno de los algoritmos que se utiliza es de clusterización jerárquico, más adelante se explicara con más detalle.
\subsection{Data Model}
Dado el conjunto de objetos $I$ y una función de similitud $ s: I \times I \rightarrow [0;1]$, cada objeto es unívocamente identificado y contiene un conjunto de atributos. La entrada puede pensarse como un grafo completo con peso en las aristas $G=(I,E,s)$ donde el peso del vértice $(u,v)$ es $s(u,v)$. Se define también la \textit{función de distancia} $d(u,v) = 1 - s(u,v)$ que también toma valores en el intervalo $[0;1]$.
\subsection{Problem Statement}
El problema consiste en devolver un conjunto de bundles $S = \left\{s_1, \ldots, s_k\right\}$ donde el bundle $S_i \in 2^{I}$ es un conjunto de objetos que satisface las reglas de \textit{complementaridad} que no permite que existen dos objetos con igual atributo en el mismo bundle y de \textit{presupuesto} para que la suma de los costos de los objetos no exceda el presupuesto dado.\\
\textbf{Definición} Dado el conjunto de objetos $I=\left\{i_1,\ldots, i_n\right\}$ el bundle $S \in 2^{I}$ es válido si y sólo si satisface las reglas:
\begin{itemize}
	\item \textbf{Complementaridad:} dada la propiedad $\alpha$ de los objetos, $\forall u,v \in S_i, u.\alpha \neq v.\alpha$
	\item \textbf{Presupuesto:} dada la función de costo $f$ y el presupuesto $\beta$, entonces $\forall S_i \in S, f(S_i) \leq \beta$
\end{itemize}

La definición formal de \textit{Composite Retrieval} es:\\
Dado el conjunto de objetos $I = \left\{i_1, \ldots, i_n \right\}$, la función de similitud $s(u,v)$, el atributo complementario $\alpha$, la función de costo $f$, el presupuesto $\beta$ y el entero $k$ se desea hallar el conjunto válido de bundles $S = \left\{s_1, \ldots, s_k\right\}$ que maximiza la función:
\begin{equation} \label{des:eq-fnObj}
  \sum_{1 \leq i \leq k}{\sum_{u,v \in S_i}{\gamma s(u,v)}} + \sum_{1 \leq i \leq j \leq k}{(1-\gamma) (1-\max_{u \in S_i, v \in S_j}{s(u,v)})}
\end{equation}
Esta es una tipia función objetivo de un problema de clustering, donde la calidad del clustering es una combinación entre la calidad de cada cluster (intra-cluster) y de la separación entre clusters (inter-cluster). A través del parámetro $\gamma$ el usuario puede definir el balance entre intra e inter de una solución. Si El usuario prioriza una solución de bundles cohesivos sobre la diversidad el  valor de $\gamma$ será cerca de uno y si lo que prioriza es la diversidad el valor estará cerca del cero.\\
En \cite{compositeRetrival} se demuestra que la complejidad de devolver $k$ bundles de items complementarios con un presupuesto es NP-Completo, por el momento no se puede encontrar una solución exacta en tiempo polinomial, por lo cual este trabajo se enfoca a encontrar soluciones suficientemente buenas para el problema. Para poder encontrar la mejor solución se implementaron dos algoritmos para poder comparar los resultados: Produce-and-Choose y algoritmo goloso.
\subsection{Problem Complexity}
La complejidad del problema
\section{Produce and Choose}
Explicar el algoritmo produce and choose y que para el produce se utilizo el Hierichal
\section{Proposal}
Sobre el trabajo de 1 se realizaron las siguientes propuestas con los objetivos 1) realizar las búsquedas en instancias más grandes y 2)heuristicas que mejoren el resultado obtenido.\\
Para 1) se realizo una variante del algoritmo jerárquico que reduce el orden de complejidad. 
Para 2) se propusieron dos implementaciones de la metahuristica tabu search para encontrar mejores soluciones en cada una de las etapas de PAC.

\subsection{Produce Phase}
La clusterización jerárquica se clasifica entre los algoritmos \textit{aglomerativo} y \textit{divisivo}. En el aglomerativo inicialmente cada objeto pertenece a un cluster unitario y luego sucesivamente se unen un par de clusters hasta que todos los clusters se hayan unido en un único cluster que contenga a todos los objetos, esete algoritmo es conocido como \textit{hierarchical agglomerative clustering} (HAC). El divisorio comienza con un único cluster al que todos los objetos pertenecen y recursivamente se realiza una división del cluster hasta obtener clusters con un solo objeto.\\
\textit{Constrained hierarchical agglomerative clustering} (C-HAC) es una modificación que se realiza sobre HAC para que nunca se realice la unión de los clusters $S_1$ y $S_2$ si el cluster resultante $S_1 \cup S_2$ es inválido, o sea sí el cluster resultante no cumple con las restricciones de similitud o el costo del bundle supera el presupuesto. El algoritmo C-HAC que se presenta a continuación, en este trabajo se denomina \texttt{Simple C-HAC}, es el que se propone en \cite{compositeRetrival}.\\

\begin{algorithm}[H]
\begin{algorithmic}[1]
\REQUIRE {$I,\alpha,f,\beta,\gamma,\text{ cantidad de bundles }c$}
\ENSURE Conjunto válido de bundles
\STATE $cand \leftarrow \bigcup_{i \in I}\left\{i\right\}$
\WHILE {$ \left|cand\right| > c$}
	\STATE $bestScore \leftarrow -\infty$
	\STATE $bestCandidate \leftarrow \emptyset$
	\FOR{$\text{each}\ S_i\in cand$}
		\FOR{$\text{each}\ S_j\in cand; S_i \neq S_j$}
			\IF {$ValidMerge(S_i,S_j,\alpha,f,\beta)$} \label{validMerge}
				\IF {$Score(S_i \cup S_j) \geq bestcore$} \label{score}
					\STATE $bestScore \leftarrow score(S_i \cup  S_j)$
					\STATE $bestCandidate \leftarrow \left\{S_i,S_j\right\}$
				\ENDIF
			\ENDIF
		\ENDFOR
	\ENDFOR
	\IF {$bestCandidate = \emptyset$}
		\BREAK
	\ENDIF
	\STATE {$cand \leftarrow cand \setminus \left\{S\right\}$ $(\forall S \in bestCandidate)$}
	\STATE $cand \leftarrow cand \cup bestCandidate $
\ENDWHILE
\RETURN $cand$
\end{algorithmic}
\caption{Simple C-HAC}\label{alg:SimpleC-HAC}
\end{algorithm}

El algoritmo ejecuta $N - c$ pasos donde se unen los dos clusters más similares que forman un cluster válido. En cada paso se realiza una comparación entre todos los clusters. Por lo tanto el orden de complejidad de \texttt{Simple C-HAC} es $\mathcal{O}(N^{3})$.\\

En cada iteración se unen los dos clusters que maximizan la función \textit{Score}. El artículo \cite{compositeRetrival} proponé que \textit{Score} se calcule sumarizando la similitud entre todos los objetos del cluster candidato. Este criterio de unión es local, ya que presta atención unicamente a la similitud inter cluster. Por lo que se propuso un criterio que tenga en cuenta toda la estructura del clustering al momento de decidir la unión de los clusters. La función que se propone es \textit{Sim} que sumariza la similitud entre todos los elementos de la unión de los clusters y se le suma la similitud de los dos elementos más disimiles proporcionando el peso con el valor de $\gamma$ entre la similitud y el disímil.\\

Por el orden de complejidad del algoritmo \texttt{Simple C-HAC}  en escenarios donde la clusterización se tenga que hacer entre miles de objetos el tiempo de ejecución será tan elevado que el algoritmo es improductivo. Para contemplar estos escenarios se implemento el algoritmo \texttt{Efficient C-HAC} que la complejidad es $\mathcal{O}(N^{2}\lg n)$. La similitud entre los clusters se guarda en colas de prioridad en orden decreciente, entonces la cola $P\left[k\right].max()$ devuelve el cluster de mayor similitud con el k-ésimo cluster. Luego de combinar los clusters $\omega_{k_{1}}$ y $\omega_{k_{2}}$, $\omega_{k_{1}}$ se utiliza como la representación. Para $\omega_{k_{1}}$ se calcula la similitud con el resto de los clusters y se actualiza en las colas de similitud. Para identificar entre dos cluster que la unión es invalida por alguna de las restricciones, en las colas de similitud el valor de la similitud es $-1$.\\

\begin{algorithm}[H]
\begin{algorithmic}[1]
\REQUIRE {$I,\alpha,f,\beta,\gamma$}
\ENSURE Conjunto válido de bundles
\FOR{$\text{each}\ S_i\in I$}
	\FOR{$\text{each}\ S_j\in I$}
		\IF {$validMerge(S_i,S_j,\alpha,f,\beta)$} 
			\STATE $C[i][j].sim \leftarrow Score(S_i \cup S_j)$
		\ELSE
			\STATE $C[i][j].sim \leftarrow -1$
		\ENDIF
		\STATE $C[i][j].index \leftarrow j$
	\ENDFOR
	\STATE $I[i] \leftarrow 1$
	\STATE {$P[i] \leftarrow $ priority queue for $C[i]$ sorted on sim}
	\STATE {P[i].Delete(C[i][i]) (se elimina así mismo de la pila)}
\ENDFOR
\STATE $A \leftarrow []$
\FOR {$k \leftarrow 1$ to $I.length$}
	\STATE $k_1 \leftarrow \max_{k:I[k]=1}{P[k].max().sim}$
	\IF {$validMerge(S_i,S_j,\alpha,f,\beta)$}
		\BREAK
	\ENDIF
	\STATE $k_2 \leftarrow P[k_1].max().index$
	\STATE $A.Append(\left\langle k_1,k_2 \right\rangle)$
	\STATE $I[k_2] \leftarrow 0$
	\STATE $P[k_1] \leftarrow []$
	\FOR {$i$ with $I[i]-1 \vee i \neq k_1$}
		\STATE $P[i].Delete(C[i][K_1])$
		\STATE $P[i].Delete(C[i][k_2])$
		\IF {$validMerge(S_i,S_j,\alpha,f,\beta)$}
			\STATE $C[i][k_1].sim \leftarrow Sim(i,k_1 \cup k_2,\gamma)$
			\STATE $C[k_1][i].sim \leftarrow Sim(i,k_1 \cup k_2,\gamma)$
		\ELSE
			\STATE $C[i][k_1].sim \leftarrow -1$
			\STATE $C[k_1][i].sim \leftarrow -1$
		\ENDIF
		\STATE $C[i][k_1].index \leftarrow i$
		\STATE $C[k_1][i].index \leftarrow i$		
		\STATE $P[i].Insert(C[i][k_1])$
		\STATE $P[K_1].Insert(C[k_1][i])$		
	\ENDFOR
\ENDFOR

\RETURN $A$
\end{algorithmic}
\caption{Efficient C-HAC}\label{alg:Efficient C-HAC}
\end{algorithm}

\subsection{Tabu search}
Las búsquedas locales consisten en moverse de solución en solución, aplicando cambios a la solución candidata hasta encontrar una mejor solución o satisfacer un criterio de parada. Los algoritmos consisten en comenzar con una solución e iterativamente moverse a una solución vecina, esto es posible solo si se pude definir una relación de vecindad en el espacio de búsqueda. Como una solución puede tener muchas soluciones vecinas se elige siempre la que maximice o minimice (según el problema elegido) el criterio seleccionado, esto produce que el algoritmo pueda estancarse en un mínimo (ó máximo) local y nunca pueda salir de él.\\
\textbf{Tabú search} es una metaheurística, de la familia de las búsquedas locales, que relaja la primer regla de las búsquedas locales tradicionales y permite moverse a una solución vecina que no cumple con el criterio de búsqueda. De esta manera se permite al algoritmo escapar de máximos o mínimos locales y encontrar una mejor solución (en caso que existiese). Otras de las modificaciones que se agregan es que una vez que una solución determinada es visitada, se la marca como tabú para que no vuelva a ser visitada por una determinada cantidad de iteraciones para también de esta manera evitar caer en ciclos y mínimos o máximos locales.\\
Una de las ventajas que tienen este tipo de metaheurísticas es que no son muy costosas en tiempo de ejecución siempre que la cantidad máxima de iteraciones no sea excesiva, con lo cual se puede ejecutar sin problemas y sin importar el algoritmo de generación y selección provenga la solución orginal con el fin de intentar mejorarla.\\
Se implementaron las búsquedas tabú Inter-Bundle e Intra-Bundle. La primera busca encontrar una mejor solución entre la solución actual y los bundles ya generados; la otra consiste en mejorar los bundles con los items que quedaron fuera de la solución.

\subsection{Inter-Bundle}
La búsqueda se concibió especialmente para la fase de selección del algoritmo \texttt{Produce and Choose} que es la fase en la que se selecciona un subconjunto de bundles del conjunto de bundles generados en la fase anterior. De la solución obtenida en la fase de selección se realiza la búsqueda tabú con los bundles generados en la fase del produce con el objetivo de visitar las soluciones vecinas.\\
Los movimientos de la solución $S$ a la solución $S'$ consiste de los siguientes  pasos:
\begin{enumerate}
	\item Quitar de la solución el bundle con menor Inter.
	\item Determinar el bundle centroide de la solución.
	\item Calcular la función objetivo agregando uno de los $K$ bundles generados más lejos del centroide.
	\item Quedarse con la solución con mayor función objetivo. 
\end{enumerate}

Sea $S$ el conjunto de bundles de la solucion y B el conjunto de todos los bundles producidos. El bundle (1) es el más acoplado al de la solución $b_r = \min_{b_1 \in S}{\sum_{b_2 \in S}{\psi(b_1,b_2)}}$. El centroide de (2) es el bundle que tiene mayor similitud entre los bundles de la solución, sin tener en cuenta al bundle a reemplazar, entonces el centroide es:
$$b_c = \min_{b_1 \in S \setminus \left\{b_r\right\}}{\sum_{b_2 \in S \setminus \left\{b_r\right\}}{\psi(b_1,b_2)}}$$
El bundle de (3) se obtiene de $b_n = \min_{b_1 \in S \setminus \left\{b_r\right\}}{\psi(b_1,b_c)}$. Por lo tanto la nueva solución es $S' = (S \setminus \left\{b_r\right\}) \cup \left\{b_n\right\}$. Mientras que el bundle $b_r$ se marca para que no sea seleccionado para las próximas soluciones generadas.
\begin{algorithm}[H]
\begin{algorithmic}[1]
\REQUIRE {$S\text{ solucion},cand\text{ conjunto valido de bundles},\gamma, MAX\_SOL\text{ cantidad de soluciones vecinas visitadas},MAX\_BUND\text{ cantidad bundles para probar}, ITER\_TABU\text{ iteraciones en tabú}$}
\ENSURE Conjunto válido de bundles
\STATE $\omega(S) = \sum_{b \in S}{\sum_{u,v \in b}{\gamma s(u,v)}} + \sum_{b_1,b_2 \in S}{(1-\gamma) (1-\max_{u \in b_1, v \in b_2}{s(u,v)})}$
\STATE $UpdateTabu(S) = \left\{ \left\langle b, n-1 \right\rangle  / \left\langle b, n \right\rangle \in S \wedge n-1 > 0 \right\}$
\STATE $Inter(b_1, S) = \sum_{b_2 \in S, b_1\neq b_2}{(1-\max_{u \in b_1, v \in b_2}{s(u,v)})}$
\STATE $iteration \leftarrow 0$
\STATE $tabuBundles \leftarrow \emptyset$
\STATE $bestSolution \leftarrow S$
\STATE $visitSolution \leftarrow S$
\STATE $thresholdScore \leftarrow -\infty$ 
\WHILE {$iteration < MAX\_ITER$}
  \STATE $worstBundle \leftarrow \min_{b \in visitSolution \setminus tabuBundles}{Inter(b, visitSolution)}$
	\STATE $centroidBundle \leftarrow getCentroid(visitSolution)$
	\STATE $bestBundles \leftarrow \text{fetch first MAX\_BUND} \max_{b \in cand \setminus tabuBundles}{Dist(b, centroidBundle)}$
	\STATE $scoreInter \leftarrow \sum_{b \in visitSolution}{Inter(b, iterationSolution)}$
	\FOR {$\text{each}\ aBundle \in bestBundles$}
    \STATE $iterationSolution \leftarrow (visitSolution \setminus \left\{worstBundle\right\}) \cup \left\{aBundle\right\}$
    \STATE $newScoreInter \leftarrow \sum_{b \in iterationSolution}{Inter(b, iterationSolution)}$
    \IF {$newScoreInter > scoreInter$}
			\STATE $thresholdScore \leftarrow scoreInter$
      \STATE $scoreInter \leftarrow newScoreInter$
      \STATE $visitSolution \leftarrow iterationSolution$
    \ELSE
      \IF {$newScoreInter > thresholdScore$}
        \STATE $thresholdScore \leftarrow newScoreInter$
        \STATE $visitSolution \leftarrow iterationSolution$
      \ENDIF
    \ENDIF
  \ENDFOR
  \IF {$\omega(iterationSolution) > \omega(bestSolution)$}
    \STATE $bestSolution \leftarrow iterationSolution$
  \ENDIF
  \STATE $tabuBundles \leftarrow tabuBundles \cup \left\{
	\left\langle worstBundle, ITER\_TABU \right\rangle\right\}$
	\STATE $tabuBundles \leftarrow UpdateTabu(tabuBundles)$
	\STATE $iteration \leftarrow iteration + 1$
\ENDWHILE
\RETURN $bestSolution$
\end{algorithmic}
\caption{Búsqueda tabú sobre bundles}\label{alg:algBusTabuBundle}
\end{algorithm}

\subsection{Intra-Bundle}
En Intra-Bundle explora soluciones con bundles más cohesivos. De la solución actual se realiza el movimiento a una nueva solución con los pasos:
\begin{enumerate}
	\item Obtener el bundle menos cohesivos de la solución.
	\item Determinar el centroide del bundle.
	\item Hallar el ítem más alejado del centroide.
	\item Calcular el inter del bundle agregando uno de los $K$ items más cercano del centroide.
	\item Generar un nuevo bundle en la solución con el item que maximiza el inter.
\end{enumerate}

Sea $S$ el conjunto de bundles de la solución e $I$ el conjunto de ítems, el bundle de (1) es $b = \min_{b_1 \in S}{\sum_{v,w \in b_1}{s(v,w)}}$. De $b$ se define el centroide $c$ del paso (2) con $c = \max_{v \in b}{\sum_{w \in b}{s(v,w)}}$. El item de (3) se obtiene de $i = \min_{v \in b}{s(v,c)}$. El item para reemplazar a $i$ es $j = \max_{v \in I \setminus items(S)}{s(v,c)}$. Por lo que la nueva solución se define $S' = (S \setminus \left\{b\right\}) \cup \left\{(b \setminus \left\{i\right\})\cup\left\{j\right\}\right\}$

\begin{algorithm}[H]
\begin{algorithmic}[1]
\REQUIRE {$S\text{ solucion},cand\text{ conjunto valido de bundles},I,\alpha,f,\beta,k,\gamma, MAX\_SOL\text{ cantidad de soluciones vecinas visitadas},MAX\_BUND\text{ cantidad bundles para probar}, ITER\_TABU\text{ iteraciones en tabú}$}
\ENSURE Conjunto válido de bundles
\STATE $\omega(S) = \sum_{b \in S}{\sum_{u,v \in b}{\gamma s(u,v)}} + \sum_{b_1,b_2 \in S}{(1-\gamma) (1-\max_{u \in b_1, v \in b_2}{s(u,v)})}$
\STATE $Inter(b_1, S) = \sum_{b_2 \in S, b_1\neq b_2}{(1-\max_{u \in b_1, v \in b_2}{s(u,v)})}$
\STATE $Intra(b) = \sum_{u,v \in b}{\gamma s(u,v)}$
\STATE $UpdateTabu(S) = \left\{ \left\langle b, n-1 \right\rangle  / \left\langle b, n \right\rangle \in S \wedge n-1 > 0 \right\}$
\STATE $iteration \leftarrow 0$
\STATE $tabuBundles \leftarrow \emptyset$
\STATE $tabuElements \leftarrow \emptyset$
\STATE $bestSolution \leftarrow S$
\STATE $visitSolution \leftarrow S$
\STATE $thresholdScore \leftarrow -\inf$
\WHILE {$iteration < MAX\_ITER$}
  \STATE $worstBunlde \leftarrow \min_{b \in visitSolution \setminus tabuBundles}{Intra(b)}$
  \STATE $centroid \leftarrow GetCentroid(worstBunlde)$
  \STATE $faraway \leftarrow GetFarawayElement(worstBunlde,centroid)$
  \STATE $bestElements: \leftarrow \text{fetch first MAX\_BUND} \max_{b \in I \setminus tabuElements}{Dist(b, centroidBundle)}$
	\STATE $scoreInter \leftarrow \omega(visitSolution)$
  \FOR {$near \in bestElements$}
		\STATE $newBundle \leftarrow (worstBunlde \setminus \left\{faraway\right\})\cup\left\{near\right\}$
		\STATE $itSolution \leftarrow (visitSolution \setminus \left\{worstBunlde\right\}) \cup \left\{newBundle\right\}$
    \STATE $newScore \Leftarrow \omega(itSolution)$
		\IF {$newScore > scoreInter$}
			\STATE $thresholdScore \leftarrow scoreInter$
			\STATE $scoreInter \leftarrow newScore$
			\STATE $visitSolution \leftarrow itSolution$
    \ELSIF {$newScore > thresholdScore$}
				\STATE $thresholdScore \leftarrow newScore$
				\STATE $visitSolution \leftarrow itSolution$
    \ENDIF
  \ENDFOR
  \IF {$\omega(visitSolution) > \omega(bestSolution)$}
		\STATE $bestSolution \leftarrow visitSolution$
  \ENDIF
	\STATE $tabuBundles \leftarrow tabuBundles \cup \left\{
	\left\langle worstBunlde, ITER\_TABU \right\rangle\right\}$
  \STATE $tabuElements \leftarrow tabuElements \cup \left\{
	\left\langle faraway, ITER\_TABU \right\rangle\right\}$
	\STATE $tabuBundles \leftarrow UpdateTabu(tabuBundles)$
  \STATE $tabuElements \leftarrow UpdateTabu(tabuElements)$
	\STATE $iteration \leftarrow iteration + 1$
\ENDWHILE
\RETURN $bestSolution$
\end{algorithmic}
\caption{Búsqueda tabú sobre elementos}\label{alg:algBusTabuIntra}
\end{algorithm}

\section{Results}
Las primeras búsquedas que se realizaron fueron sobre los papers y se quiere lograr una solución en la que cada uno de los bundles contenga artículos relacionados entre sí pero que hayan sido presentados en diferentes conferencias. Separando la búsqueda en los conceptos vistos de Composite Retrieval sería la similitud una función que compara el perfil de cada paper que como vimos anteriormente se obtiene usando los perfiles y la complementaridad el lugar en el cual fue presentado cada paper.\\
Los resultados que aquí se verán son los obtenidos de las ejecuciones de los algoritmos mencionados en ~\cite{compositeRetrival} con las modificaciones propuestas para este trabajo. A continuación se muestran 2 gráficos que reflejan el valor de la función objetivo para los algoritmos BOBO y C-HAC. También se puede observar como mejora la solución cuando se aplica la búsqueda Tabú. Los valores de $\gamma$ que se usaron para los experimentos fueron $0.1$ y $0.9$ pero éstos pueden tomar cualquier valor en el rango $[0:1]$. El mismo sirve para indicar que tipo de soluciones se quieren, a menor valor de $\gamma$ los bundles son mas cohesivos.
\begin{figure}[H]
	\centering
	\includegraphics[width=0.25\textwidth]{img/gamma01.png}
	\caption{}
	\label{res:img-gamma01-papers}
\end{figure}

\begin{figure}[H]
	\centering
	\includegraphics[width=0.25\textwidth]{img/gamma09.png}
	\caption{}
	\label{res:img-gamma09-papers}
\end{figure}
En la segunda búsqueda que se realizo se buscaron bundles de autores los cuáles no sean sean de la misma universidad de afiliación. Como similitud se cuenta con la función que compara el perfil de los autores y como complementaridad la universidad de pertenencia del autor.
\begin{figure}[H]
	\centering
	\includegraphics[width=0.25\textwidth]{img/gamma01-autores.png}
	\caption{}
	\label{res:img-gamma01-authors}
\end{figure}

\begin{figure}[H]
	\centering
	\includegraphics[width=0.25\textwidth]{img/gamma09-autores.png}
	\caption{}
	\label{res:img-gamma09-authors}
\end{figure}
\section{Conclusiones}
Como se ve en los gráficos del punto anterior el uso de la estrategía de búsqueda Tabú funcionó en todos los casos. Para el algoritmo BOBO la mejora fue significativamente mejor y no implicó ningún aumento considerable en el tiempo de ejecución, por lo cuál sin importar el algoritmo usado para generar la solución siempre se debe intentar mejorar mediante la búsqueda Tabú.
%ACKNOWLEDGMENTS are optional
\section{Acknowledgments}


%
% The following two commands are all you need in the
% initial runs of your .tex file to
% produce the bibliography for the citations in your paper.
\bibliographystyle{abbrv}
\bibliography{sigproc}  % sigproc.bib is the name of the Bibliography in this case
% You must have a proper ".bib" file
%  and remember to run:
% latex bibtex latex latex
% to resolve all references
%
% ACM needs 'a single self-contained file'!
%
%APPENDICES are optional
%\balancecolumns
\appendix
%Appendix A
%\subsection{Additional Authors}
%This section is inserted by \LaTeX; you do not insert it.
%You just add the names and information in the
%\texttt{{\char'134}additionalauthors} command at the start
%of the document.
\section{Used algorithms}

%\balancecolumns % GM June 2007
% That's all folks!
\end{document}
