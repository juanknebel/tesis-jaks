% This is "sig-alternate.tex" V2.1 April 2013
% This file should be compiled with V2.5 of "sig-alternate.cls" May 2012
%
% This example file demonstrates the use of the 'sig-alternate.cls'
% V2.5 LaTeX2e document class file. It is for those submitting
% articles to ACM Conference Proceedings WHO DO NOT WISH TO
% STRICTLY ADHERE TO THE SIGS (PUBS-BOARD-ENDORSED) STYLE.
% The 'sig-alternate.cls' file will produce a similar-looking,
% albeit, 'tighter' paper resulting in, invariably, fewer pages.
%
% ----------------------------------------------------------------------------------------------------------------
% This .tex file (and associated .cls V2.5) produces:
%       1) The Permission Statement
%       2) The Conference (location) Info information
%       3) The Copyright Line with ACM data
%       4) NO page numbers
%
% as against the acm_proc_article-sp.cls file which
% DOES NOT produce 1) thru' 3) above.
%
% Using 'sig-alternate.cls' you have control, however, from within
% the source .tex file, over both the CopyrightYear
% (defaulted to 200X) and the ACM Copyright Data
% (defaulted to X-XXXXX-XX-X/XX/XX).
% e.g.
% \CopyrightYear{2007} will cause 2007 to appear in the copyright line.
% \crdata{0-12345-67-8/90/12} will cause 0-12345-67-8/90/12 to appear in the copyright line.
%
% ---------------------------------------------------------------------------------------------------------------
% This .tex source is an example which *does* use
% the .bib file (from which the .bbl file % is produced).
% REMEMBER HOWEVER: After having produced the .bbl file,
% and prior to final submission, you *NEED* to 'insert'
% your .bbl file into your source .tex file so as to provide
% ONE 'self-contained' source file.
%
% ================= IF YOU HAVE QUESTIONS =======================
% Questions regarding the SIGS styles, SIGS policies and
% procedures, Conferences etc. should be sent to
% Adrienne Griscti (griscti@acm.org)
%
% Technical questions _only_ to
% Gerald Murray (murray@hq.acm.org)
% ===============================================================
%
% For tracking purposes - this is V2.0 - May 2012

\documentclass{sig-alternate-05-2015}



\usepackage{algorithm}
\usepackage{algorithmic}
\usepackage[utf8]{inputenc}

\renewcommand{\algorithmicrequire}{\textbf{Input:}}
\renewcommand{\algorithmicensure}{\textbf{Output:}}
\newcommand{\algorithmicbreak}{\textbf{break}}
\newcommand{\BREAK}{\STATE \algorithmicbreak}

\begin{document}

% Copyright
%\setcopyright{acmcopyright}
%\setcopyright{acmlicensed}
%\setcopyright{rightsretained}
%\setcopyright{usgov}
%\setcopyright{usgovmixed}
%\setcopyright{cagov}
%\setcopyright{cagovmixed}


% DOI
%\doi{10.475/123_4}

% ISBN
%\isbn{123-4567-24-567/08/06}

%Conference
%\conferenceinfo{PLDI '13}{June 16--19, 2013, Seattle, WA, USA}

%\acmPrice{\$15.00}

%
% --- Author Metadata here ---
%\conferenceinfo{WOODSTOCK}{'97 El Paso, Texas USA}
%\CopyrightYear{2007} % Allows default copyright year (20XX) to be over-ridden - IF NEED BE.
%\crdata{0-12345-67-8/90/01}  % Allows default copyright data (0-89791-88-6/97/05) to be over-ridden - IF NEED BE.
% --- End of Author Metadata ---

%\title{Title\titlenote{Title Note}}
\title{Composite Retrieval}
%\subtitle{[Extended Abstract]\titlenote{Subtitle note}}
\subtitle{[Extended Abstract]}

%
% You need the command \numberofauthors to handle the 'placement
% and alignment' of the authors beneath the title.
%
% For aesthetic reasons, we recommend 'three authors at a time'
% i.e. three 'name/affiliation blocks' be placed beneath the title.
%
% NOTE: You are NOT restricted in how many 'rows' of
% "name/affiliations" may appear. We just ask that you restrict
% the number of 'columns' to three.
%
% Because of the available 'opening page real-estate'
% we ask you to refrain from putting more than six authors
% (two rows with three columns) beneath the article title.
% More than six makes the first-page appear very cluttered indeed.
%
% Use the \alignauthor commands to handle the names
% and affiliations for an 'aesthetic maximum' of six authors.
% Add names, affiliations, addresses for
% the seventh etc. author(s) as the argument for the
% \additionalauthors command.
% These 'additional authors' will be output/set for you
% without further effort on your part as the last section in
% the body of your article BEFORE References or any Appendices.

\numberofauthors{5} %  in this sample file, there are a *total*
% of EIGHT authors. SIX appear on the 'first-page' (for formatting
% reasons) and the remaining two appear in the \additionalauthors section.
%
\author{
% You can go ahead and credit any number of authors here,
% e.g. one 'row of three' or two rows (consisting of one row of three
% and a second row of one, two or three).
%
% The command \alignauthor (no curly braces needed) should
% precede each author name, affiliation/snail-mail address and
% e-mail address. Additionally, tag each line of
% affiliation/address with \affaddr, and tag the
% e-mail address with \email.
%
% 1st. author
\alignauthor
Esteban Feuerstein\\
%\titlenote{The secretary disavows any knowledge of this author's actions.}\\
       \affaddr{Departamento de Computación. FCEyN, UBA}\\
       \affaddr{Pabellón I, Ciudad Universitaria, C1428EGA, CABA}\\
       \affaddr{Buenos Aires, Argentina}\\
       \email{efeuerst@gmail.com}
% 2nd. author
\alignauthor 
Juan Andrés Knebel\\
       \affaddr{Departamento de Computación. FCEyN, UBA}\\
       \affaddr{Pabellón I, Ciudad Universitaria, C1428EGA, CABA}\\
       \affaddr{Buenos Aires, Argentina}\\
       \email{juanknebel@gmail.com}
% 3rd. author
\alignauthor
Isabel Méndez-Díaz\\
%\titlenote{This author is the one who did all the really hard work.}\\
       \affaddr{Departamento de Computación. FCEyN, UBA}\\
       \affaddr{Pabellón I, Ciudad Universitaria, C1428EGA, CABA}\\
       \affaddr{Buenos Aires, Argentina}\\
       \email{imendez@dc.uba.ar}
\and  % use '\and' if you need 'another row' of author names
% 4th. author
\alignauthor
Amit Stein\\
%\titlenote{Dr.~Trovato insisted his name be first.}\\
       \affaddr{Departamento de Computación. FCEyN, UBA}\\
       \affaddr{Pabellón I, Ciudad Universitaria, C1428EGA, CABA}\\
       \affaddr{Buenos Aires, Argentina}\\
       \email{astein@dc.uba.ar}
% 5th. author
\alignauthor 
Paula Zabala\\
       \affaddr{Departamento de Computación. FCEyN, UBA}\\
       \affaddr{Pabellón I, Ciudad Universitaria, C1428EGA, CABA}\\
       \affaddr{Buenos Aires, Argentina}\\
       \email{pzabala@dc.uba.ar}
}
% There's nothing stopping you putting the seventh, eighth, etc.
% author on the opening page (as the 'third row') but we ask,
% for aesthetic reasons that you place these 'additional authors'
% in the \additional authors block, viz.
\date{30 July 1999}
% Just remember to make sure that the TOTAL number of authors
% is the number that will appear on the first page PLUS the
% number that will appear in the \additionalauthors section.

\maketitle
\begin{abstract}
	Las búsquedas tradicionales ofrecen soluciones que solo tiene en cuenta un solo atributo de los elementos y no la relación que éstos tienen con el resto de los elementos. Las mismas nos ofrecen un lista ordenada de resultados que se relacionan con el criterio seleccionado, ocasionando que muchas veces se necesite reformular la consulta original para así lograr una solución adecuado al criterio de búsqueda.\\
	El artículo \textbf{Composite Retrieval of Diverse and Complementary Bundles}\cite{compositeRetrival} propone devolver los elementos resultantes agrupados, dónde cada grupo esta relacionado internamente bajo algún criterio de similitud y a la vez los elementos sean complementarios de forma tal que satisfaga las expectativas del usuario para que no tenga la necesidad de realizar una nueva intervención y así lograr una mejor experiencia de búsqueda.\\
	En este trabajo se aplico este tipo de resultados en consultas realizadas sobre la base de datos provista por ~\cite{dataDrive} que contiene artículos relacionados con la ingeniería de software. Por ejemplo, para la consulta: artículos de distintas universidades el resultado obtenido es una lista de grupos, en la que cada grupo contiene artículos similares que fueron escritos en distintas universidades.\\ 
Para obtener estos resultados se desarrollaron algoritmos de clusterización y se implementaron metaheurística que buscan una solución mas óptima en un conjunto de soluciones.\\
\end{abstract}
%
% The code below should be generated by the tool at
% http://dl.acm.org/ccs.cfm
% Please copy and paste the code instead of the example below. 
%
\begin{CCSXML}
<ccs2012>
<concept>
<concept_id>10002951.10003317.10003338.10003345</concept_id>
<concept_desc>Information systems~Information retrieval diversity</concept_desc>
<concept_significance>300</concept_significance>
</concept>
</ccs2012>
\end{CCSXML}

\ccsdesc[300]{Information systems~Information retrieval diversity}

%
% End generated code
%

%
%  Use this command to print the description
%
\printccsdesc
% We no longer use \terms command
%\terms{Theory}
\keywords{Composite Retrieval, Tabu Searh}
\section{Introduction}
En las búsquedas convencionales el usuario ingresa una consulta esperando que el buscador devuelva una colección de elementos que coincidan con el criterio de búsqueda elegido. En general lo que ocurre es que son varios los elementos del universo que concuerdan pero con grados de relevancia diferentes, los motores de búsqudas utilizan el ranking de resultados para ordenar la colección de elementos devueltos. El ranking de resultados se obtiene a partir de representación lógica de los elementos que incluye los metadatos necesarios para operar sobre ellos. La desventaja de los ranking de resultados es que únicamente se compara la consulta de la búsqueda con los metadatos de los elementos, dejando de lado el análisis de los elementos entre sí y conviritiendo, en ocasiones, al proceso en una acción tediosa y repetitiva ya que el usuario deberá cambiar la consulta original y explorar la colección de elementos hasta lograr encontrar el o los elementos deseado.\\
En el  artículo \textbf{Composite Retrieval of Diverse and Complementary Bundles}\cite{compositeRetrival} se propone presentar una lista de grupos de elementos, en lugar de entregar una lista vertical de los mismos. Cada grupo deberá estar relacionado internamente bajo el criterio de similitud elegido y la lista ordenada de forma lógica con la finalidad de que uno o más conjuntos satisfagan las expectativas del usuario sin necesidad de una nueva intervención para refinar su búsqueda para lograr una mejor experiencia de búsqueda.\\
Planear un viaje típicamente requiere realizar múltiples búsquedas en distintos motores para recabar la información de los diferentes destinos que se quiere visitar, las distancias geográficas, los precios de las atracciones, las actividades que se pueden realizar o leer opiniones acerca de los destinos seleccionados, entre otros.\\
En una búsqueda típica los resultados obtenidos son una larga lista ordenada por la relevancia del criterio de la consulta. Este tipo de soluciones no otorgan respuestas que relacionen el criterio buscado con los demás elementos de la lista resultante.\\
Otro ejemplo es el caso en el que un cliente de una tienda online de venta de discos que le gusta escuchar música de diferentes países, cuenta con un presupuesto limitado y no está interesado en un ningún género musical específico, pero si quiere comprar un conjunto de discos que pertenezcan al mismo género musical. El cliente al comenzar su búsqueda obtendría una lista parecida a la siguiente:
\begin{itemize}
  \item Physical Graffiti - Led Zeppelin
  \item Led Zeppelin - Led Zeppelin
  \item It's Hard - The Who
  \item Perfect Strangers - Deep Purple
  \item El Cielo Puede Esperar - Attaque 77
  \item Wheels of Fire - Cream
  \item Confesiones de Invierno - Sui Generis
  \item The White Album - The Beatles
  \item Innuendo - Queen
  \item Sticky Fingers - The Rolling Stones
  \item Kamikaze - Luis Alberto Spinetta
\end{itemize}

De la lista obtenida el usuario deberá seleccionar aquellos discos que sean de su interes con el posible error de elegir más de un disco del mismo origen. Segundo, deberá ir agregando y eliminando de su lista manualmente en el caso que la elección de un disco superase el presupuesto que él posee. Tercero, no necesariamente elegirá el mejor subconjunto de discos que maximice su presupuesto y a su vez el origen de los discos sean distintos.\\
Para este tipo de búsquedas la solución que se propone está pensada para aquellas consultas que requieren obtener un conjunto de elementos que se relacionan como respuesta. Se podría realizar una clusterización de los resultados pero, en las técnicas tradicionales la agrupación se hace por la similitud entre ítems. En el ejemplo de los discos con una clusterización tradicional, donde la similitud sea el género musical, seguramente se generen tantos cluster como géneros de discos existan y en cada cluster se encontrarán todos los discos de ese género. Una vez obtenido el resultado se deberá explorar todos los clusters para elegir los discos.\\
En cambio si se aplicase las técnicas mencionadas en \textit{``Composite Retrieval of Diverse and Complementary Bundles''} las soluciones obtenidas se ajustarían al presupuesto y cada uno de los ítems dentro del bundle (es el nombre que se le da al agrupamiento de ítems) sean complementarios entre sí, de modo tal, que el usuario pordrá optar por cualquier bundle de la solución y estar seguro que su elección cumple con su objetivo inicial, que pertenece a un mismo género musical y exista variedad en la elección.\\
Si en el ejemplo de la tienda de discos se establece la complementariedad del atributo que refleja el origen de la banda y se establece un presupuesto máximo a cada bundle, una solución posible sería:
\begin{itemize}
  \item Bundle 1:
  \begin{itemize}
    \item Physical Graffiti - Led Zeppelin (Inglaterra)
    \item After chabón - Sumo (Argentina)
    \item Back in Black - AC/DC (Estados Unidos)
  \end{itemize}
  \item Bundle 2:
  \begin{itemize}
    \item Natty Dread - Bob Marley (Jamaica)
    \item El ritual de la Banana - Los Pericos (Argentina)
    \item Labour of Love - UB40 (Inglaterra)
  \end{itemize}
	  \item Bundle 3:
  \begin{itemize}
    \item Ramones - Ramones (Estados Unidos)
    \item El Cielo Puede Esperar - Attaque 77 (Argentina)
    \item Sandinista! - The Clash (Inglaterra)
  \end{itemize}
\end{itemize}
Lo que se quiere lograr en los ejemplos descriptos y en cualquier otro problema similar de búsquedas es otorgarle al usuario un conjunto de bundles que cumplan siempre con las siguientes propiedades: 
\begin{itemize}
  \item \textbf{Cubrimiento}: Maximizar la cantidad de elementos en el bundle.
  \item \textbf{Compatibilidad}: Los elementos del bundle deben ser similares.
  \item \textbf{Validez}: El costo total de los elementos del bundle no debe superar el presupuesto.
  \item \textbf{Diversificada}: Los bundles entre si deben ser diversos.
\end{itemize}
\section{Motivation}
Entendimos que el resultado de una búsqueda este compuesto por bundles de items complementarios sujetos a un presupuesto puede ser de gran utilidad para instancias bibliográficas. Esto facilita al usuario a explorar una amplia diversidad de items bibliográficos ya sean libros, editoriales, autores o tópicos de manera acotada. La diversidad se obtiene gracias a la complementariedad, ya que los items que contiene el bundle son similares pero tienen algun atributo que los diferencia, lo que le da riqueza al bundle por la diversidad. Que el bundle este acotado por alguna condición, por ejemplo cantidad de items por bundle, permite que un usuario pueda analizarlo y que no se pierda en un océano de items.\\
La base de datos con la que se trabajo en este artículo es la proporcionada por "Data-Driven Journey through Software Engineering Research" que contiene artículos relacionados con la ingeniería de software catalogados por autores, tópicos, venues y afiliaciones. 
Al instanciar composite retrival sobre esta base de datos, realizamos las consultas: 'Artículos de diferentes conferencias'; 'Autores de distintas universidades'; 'Instituciones de diferentes regiones' esperando obtener resultados que puedan satisfacer las necesidades de alguien que este realizando investigación sobre algún tema. Por ejemplo de la consulta de 'Artículos de diferentes conferencias' se obtienen bundles que contienen artículos similares con respecto al tópico que fueron presentados en distintas conferencias ...
\section{Related Work}
En ~\cite{compositeRetrival} se sugieren diferentes algoritmos para hallar una solución al problema. Produce and Choose es uno de ellos y en el que se encuetra basado este artículo. En el artículo original se mencionan otras dos alternativas más para la solución de este tipo de problemas, una basada en técnicas de clustering y otra en programación lineal. Es por las los resultados obtenidos de las ejecuciones con PAC que se decidió utilizarlo para realizar los cambios propuestos.\\
La estructura del algoritmo permite optimizar y agregar mejoras para obtener mejores resultados. Se implementó otra heurística con un enfoque diferente como fue una de tipo golosa pero por los resultados obtenidos y el tiempo de ejecución en este trabajo nos enfocaremos unicamente en el algoritmo PAC y las mejoras que se realizaron en él al añadirle búsquedas locales.\\
En la etapa de producción de los bundles puede ocurrir que los mismos no sean los óptimos ya que no se trata de algoritmos exactos. Una vez finalizada la etapa de producción y a diferencia de las soluciones anteriores, se intenta mejorar aquellos bundles que por un tema de ordenamiento y elección de lo items no resultaron siendo mejores. El tema de utilizar el resto de los items del universo que no fueron escogidos para formar parte de la solución final en pos de mejorar la solución, modificando los bundles ya generados, es clave para las mejoras propuestas.
\subsection{Data Model}\label{body-data-model}
El modelo de datos de la instancia de \textit{articulo italianos} contiene las entidades: artículos, autores, venues, affiliations y topics. Los artículos están etiquetados con un topic profile que representa el porcentaje de cada tópico encontrado en el artículo, así por ejemplo el paper \textit{paper1} esta catalogado con los siguientes tópicos: \textit{50\% topic1, 50\% topic2}. A partir del topic profile de cada uno de los artículos se pudo definir la noción de similitud entre los artículos.\\
Como también es interesante hacer consultas sobre los autores, se necesitaba que éstos también tengan un perfil el cuál no se encontraba en la base de datos. Con el objetivo de no depender de ninguna otra fuente se utilizaron los perfiles de los artículos para lograr el objetivo y de de esta manera definir su similitud. En orden de lograr un perfil, para cada autor se tomaron todos los papers en los cuales figura como autor y se sumaron los porcentajes de cada uno de los tópicos y luego normalizaron los valores para que tomen valore válidos (entre 0 y 1). Si bien no es cierto que el perfil del autor es que aquí se calcula ya que solo contiene información acotado, pero a medida que la base de datos se complete esta información será cada vez más precisa.\\
Utilizando la misma técnica anteriormente descripta se puede obtener el perfil del resto de los objetos (Universidad, venue).\\
Para generar los resultados de \textit{composite} se tiene un conjunto de objetos bibliográficos $I$ que son unívocamente identificado y contienen un conjunto de atributos y una función de similitud entre los objetos $ s: I \times I \rightarrow [0;1]$. En este trabajo la función de similitud se definió a partir del coseno del vector del topic profile.\\
\subsection{Problem Statement}
Formalmente el problema consiste en dado un conjunto de items $ I=\left\{i_1 \ldots i_n\right\} $, una función de similitud $ s(u,v) $ para cada par $ (u,v) \in IxI $ un atributo complementario $\alpha$, una función budget, un presupesto y un entero k se debe hallar $ S=\left\{S_1 \ldots S_k\right\} $ que maximice la función:
\begin{equation} \label{des:eq-fnObj}
\sum_{1 \leq i \leq k}{\sum_{u,v \in S_i}{\gamma s(u,v)}} + \sum_{1 \leq i \leq j \leq k}{(1-\gamma) (1-\max_{u \in S_i, v \in S_j}{s(u,v)})}
\end{equation}
Cada elemento $s_i \in S$ es válido si y sólo si satisface las reglas:
\begin{itemize}
	\item \textbf{Complementaridad:} dado el atributo $\alpha$ de los objetos, $\forall u,v \in s_i, u.\alpha \neq v.\alpha$
	\item \textbf{Presupuesto:} dada la función de costo $f$ y el presupuesto $\beta$, entonces $\forall s_i \in S, f(s_i) \leq \beta$, donde $f(s_i)$ es la suma de costos de los elementos pertenecientes al bundle.
\end{itemize}
La formula ~\ref{des:eq-fnObj} es una típica función objetivo de un problema de clustering, donde la calidad del clustering es una combinación entre la calidad de cada cluster (intra-cluster) y de la separación entre clusters (inter-cluster). A través del parámetro $\gamma$ el usuario puede definir el balance entre intra e inter de una solución. Si El usuario prioriza una solución de bundles cohesivos sobre la diversidad el  valor de $\gamma$ será cercano a uno y si lo que prioriza es la diversidad el valor estará cerca de cero.\\
\section{Algorithm}
\subsection{Produce and Choose}
Como se mencionó anteriormente el alogoritmo con el que se obtuvo los mejores resultados fue PAC. PAC consiste en dos partes: primero en generar un conjunto de bundles válidos y luego seleccionar los k mejores que formarán parte de la solución. Para la parte de generación de bundles se plantearon los métodos de clusterización \texttt{Efficient C-HAC} y \texttt{k-BOBO}.\\
\texttt{Efficient C-HAC} (Efficient Constrained hierarchical agglomerative clustering) es un adaptación de un algoritmo de clusterización jerárquico aglomerativo. En este algoritmo inicialmente cada objeto pertenece a un cluster unitario y luego sucesivamente se unen un par de clusters para generar un cluster válido. El algoritmo finaliza cuando no existan el par de bundles $S_1$ y $S_2$ tal que el bundle $S_1 \cup S_2 $ sea válido, esto es porque el costo de $S_1 \cup S_2$ supera el presupuesto o porque algún elemento de $S_1$ tiene un atributo igual que un elemento de $S_2$.\\
El método \texttt{BOBO-k} (Bundles One-By-One), está inspirado en k-means, consiste en generar $k$ cluster del conjunto de $n$ ítems. El algoritmo comienza con todos los items del conjunto $I$ como posibles pivots $P$. Se selecciona un pivote de $P$ y con los elementos de $I$ se genera un bundle válido alrededor de este, en caso que el bundle generado sea suficientemente bueno se agrega al conjunto de bundles candidatos y los ítems del bundle se eliminan de $I$. La generación de bundles continúa hasta que se cumpla el criterio de parada que es la generación de $k$ bundles.\\
Al finalizar la producción de bundles comienza la etapa de selección de bundles en la cual se deben seleccionar los $k$ bundles para la solución. El problema de seleccionar los bundles que maximizan la función objetivo se traduce en encontrar en el grafo completo G con peso en los nodos y vértices (el peso de los nodos representa la calidad de los bundles y el peso de las aristas es la distancia entre los nodos) el k-subgrafo de mayor peso (considerando los nodos y vértices). Para ello se implemento un algoritmo goloso, por el cual se selecciona iterativamente del conjunto de bundles aquel que máximiza la funcion objetivo.\\
\subsection{Tabu Search}
Con el objetivo de encontrar una solución más óptima de la obtenida por PAC, se realizo dos implementaciones de la metaheuristica tabu search. La \texttt{Inter-Bundle} explora soluciones vecinas intercambiando bundles que pertenecen a la solución con los otros bundles producidos en la etapa de producción pero no son parte de la solución. La otra implementación es \texttt{Intra-Bundle} que explora las soluciones vecinas reemplazando el item de menor similitud del centroide del bundle menos cohesivo de la solución por algún ítem que no pertenezca a la solución con mayor similitud al centroide. Para ambos casos el criterio de parada es por la cantidad de soluciones vecinas visitadas.

\section{Conclusions}
This paragraph will end the body of this sample document.
Remember that you might still have Acknowledgments or
Appendices; brief samples of these
follow.  There is still the Bibliography to deal with; and
we will make a disclaimer about that here: with the exception
of the reference to the \LaTeX\ book, the citations in
this paper are to articles which have nothing to
do with the present subject and are used as
examples only.
%\end{document}  % This is where a 'short' article might terminate
%%ACKNOWLEDGMENTS are optional
\section{Acknowledgments}


%
% The following two commands are all you need in the
% initial runs of your .tex file to
% produce the bibliography for the citations in your paper.
\bibliographystyle{abbrv}
\bibliography{sigproc}  % sigproc.bib is the name of the Bibliography in this case
% You must have a proper ".bib" file
%  and remember to run:
% latex bibtex latex latex
% to resolve all references
%
% ACM needs 'a single self-contained file'!
%
%APPENDICES are optional
%\balancecolumns
\appendix
%Appendix A
%\subsection{Additional Authors}
%This section is inserted by \LaTeX; you do not insert it.
%You just add the names and information in the
%\texttt{{\char'134}additionalauthors} command at the start
%of the document.
\section{Used algorithms}

%\balancecolumns % GM June 2007
% That's all folks!
\end{document}
