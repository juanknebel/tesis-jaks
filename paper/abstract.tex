\begin{abstract}
Las búsquedas tradicionales ofrecen soluciones que solo tiene en cuenta un solo atributo de los elementos y no la relación que éstos tienen con el resto del universo. Las suelen ofrecer una lista ordenada de resultados relacionadas con el criterio utilizado, ocasionando muchas veces reformular la consulta original para así lograr una solución adecuada al criterio de búsqueda.\\
Como respuesta a éste último comportamiento surge \textbf{Composite Retrieval} ~\cite{compositeRetrival}, su objetivo es agrupar elementos en bundles bajo un mismo atributo logrando, al mismo tiempo, que éstos sean complementarios entre sí por algún otro atributo definido previamente.\\
En este paper se tomaron las ideas ya desarrolladas previamente en ~\cite{compositeRetrival} y se aplicaron a la resolución de búsquedas sobre una base de datos de artículos científicos pertenecientes a la Ingeniería de Software ~\cite{dataDrive}. Más aún se propusieron cambios para mejorar la complejidad de los algoritmos y también se adicionaron nuevas técnicas de búsquedas pretendiendo refinar la calidad de las soluciones.\\
\end{abstract}