\begin{abstract}
Las búsquedas tradicionales ofrecen soluciones que solo tienen en cuenta un solo atributo de los elementos y no la relación que éstos tienen con el resto del universo. Las mismas suelen ofrecer una lista ordenada de resultados relacionadas con el criterio utilizado ocasionando, muchas veces, precisar reformular la consulta original para así lograr una solución adecuada al criterio de búsqueda.\\
Como respuesta a éste último comportamiento surge \textbf{Composite Retrieval of Diverse and Complementary Bundles}\cite{compositeRetrival}, su objetivo es agrupar elementos en bundles en los cuales los elementos dentro de ellos se encuentran relacionados internamente bajo algún criterio de similitud y a la vez sean complementarios de forma tal que satisfaga las expectativas del usuario y no tenga la necesidad de realizar una nueva intervención logrando así una mejor experiencia de búsqueda.\\
En este paper se tomaron las ideas ya desarrolladas previamente en ~\cite{compositeRetrival} y se aplicaron a la resolución de búsquedas sobre una base de datos de artículos científicos pertenecientes a la Ingeniería de Software ~\cite{dataDrive}. Por ejemplo, para la consulta: artículos de distintas universidades el resultado obtenido es una lista de grupos, en la que cada grupo contiene artículos similares que fueron escritos en distintas universidades.\\
Más aún se propusieron cambios para mejorar la complejidad de los algoritmos y también se adicionaron nuevas técnicas de búsquedas pretendiendo refinar la calidad de las soluciones.\\
\end{abstract}