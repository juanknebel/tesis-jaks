\begin{abstract}
	Las búsquedas tradicionales ofrecen soluciones que solo tiene en cuenta un solo atributo de los elementos y no la relación que éstos tienen con el resto de los elementos. Las mismas nos ofrecen un lista ordenada de resultados que se relacionan con el criterio seleccionado, ocasionando que muchas veces se necesite reformular la consulta original para así lograr una solución adecuado al criterio de búsqueda.\\
	El artículo \textbf{Composite Retrieval of Diverse and Complementary Bundles}\cite{compositeRetrival} propone devolver los elementos resultantes agrupados, dónde cada grupo esta relacionado internamente bajo algún criterio de similitud y a la vez los elementos sean complementarios de forma tal que satisfaga las expectativas del usuario para que no tenga la necesidad de realizar una nueva intervención y así lograr una mejor experiencia de búsqueda.\\
	En este trabajo se aplico este tipo de resultados en consultas realizadas sobre la base de datos provista por ~\cite{dataDrive} que contiene artículos relacionados con la ingeniería de software. Por ejemplo, para la consulta: artículos de distintas universidades el resultado obtenido es una lista de grupos, en la que cada grupo contiene artículos similares que fueron escritos en distintas universidades.\\ 
Para obtener estos resultados se desarrollaron algoritmos de clusterización y se implementaron metaheurística que buscan una solución mas óptima en un conjunto de soluciones.\\
\end{abstract}