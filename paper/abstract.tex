\begin{abstract}
	Las búsquedas tradicionales ofrecen soluciones que solo tiene en cuenta un solo atributo de los elementos y no la relación que éstos tienen con el resto del universo. Las mismas nos ofrecen un lista ordenada de resultados que se relacionan con el criterio seleccionado, ocasionando que muchas veces se necesite reformular la consulta original para así lograr una solución adecuado al criterio de búsqueda.\\
	Es por eso que surge \textbf{Composite Retrieval} (ref al paper), su objetivo es agrupar elementos en bundles bajo un mismo atributo logrando, al mismo tiempo, que éstos sean complementarios entre sí por algún otro atributo definido previamente.\\
	Se tomaron las ideas ya desarrolladas previamente (PAC) y se aplicaron cambios para mejorar la complejidad de los algoritmos y agregaron nuevas técnicas de búsquedas buscando el aumento de la calidad de las soluciones. Los experimentos fueron realizados sobre una base de datos de papers provista por (ref de los papers) y se compararon las soluciones entre obtenidas antes y después de aplicar los cambios.\\
\end{abstract}