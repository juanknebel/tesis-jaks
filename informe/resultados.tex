\section{Comparando soluciones}
Para comparar las la calidad de las distintas soluciones, además del valor objetivo se compara con 
la cantidad de elementos iguales en toda la solución y la cantidad igual de elementos para cada 
bundle. De esta manera se observa que tan \textquotedblleft parecidas\textquotedblright son las 
soluciones y con más detalle que tan \textquotedblleft parecidos\textquotedblright son los bundles. 
Entre los diferentes algoritmos buscamos la cantidad de elementos iguales en cada una de las 
soluciones y también la cantidad de elementos iguales por bundle.
\section{Papers}
Originalmente la base de datos contenía unos 7777 papers, de los cuáles se tuvo que hacer una 
depuración, ya que había papers que no tenían ningún autor asociado o perfil creado. Luego de la 
depuración obtuvimos 4937 que cumplen los requisitos para la búsqueda de las soluciones.\\
Se genraron soluciones con las siguientes características:\\
\Solucion
{}
{simple, por tuplas y proporcional}
{\texttt{HAC} y \texttt{BOBO-x}, con  $x \in$ $(10, 160)$}
{$\in$ $(0,1; 0,3; 0,5; 0,7; 0,9)$}
{10}
{5}
Como primera observación podemos ver la cantidad de bundles que se generan para cada algoritmo de 
producción y su tiempo de ejecución:\\
\begin{table}[h]
  \centering
  \resizebox{\textwidth}{!} {
    \begin{tabular}{|lcc|}
    \hline
    Algoritmo & Bundles Generados & Tiempo de Ejecución (minutos) \\
    \hline
    HAC       & $2378$ & $6$ \\
    BOBO-10   & $100$  & $2$ \\
    BOBO-160  & $1600$ & $5$ \\
    \hline
    \end{tabular}
  }
    \caption {Cantidad de bundles generados antes de la selección final}
\end{table}
A continuación se muestran los valores de la función objetivo obtenidos:\\
\begin{table}[h]
\centering
  \resizebox{\textwidth}{!} {
    \begin{tabular}{|lc|cccc|}
    \hline
    ~  & ~ & \multicolumn{2}{|c}{Valor función objectivo} & \multicolumn{2}{c|}{Duración de la 
ejecución (mm:ss)} \\
    Algoritmo & gamma & Selección simple & Selección proporcional & Selección simple          
         & Selección proporcional \\ 
    \hline
    HAC & $0,1$ & $48,9470$  & $35,1979$ & $10:00$ & $40:00$ \\
    HAC & $0,3$ & $59,1852$  & $58,7049$ & $10:00$ & $40:00$ \\
    HAC & $0,5$ & $70,5931$  & $70,205$ & $10:00$ & $40:00$ \\
    HAC & $0,7$ & $82,0687$  & $81,8331$ & $10:00$ & $40:00$ \\
    HAC & $0,9$ & $93,8227$  & $93,7189$ & $10:00$ & $40:00$ \\
    BOBO-160 & $0,1$ & $33,3762$  & $35,1979$ & $6:00$ & $46:00$ \\
    BOBO-160 & $0,3$ & $33,2741$  & $34,4164$ & $6:00$ & $46:00$ \\
    BOBO-160 & $0,5$ & $37,3484$  & $37,0669$ & $6:00$ & $46:00$ \\
    BOBO-160 & $0,7$ & $40,4186$  & $40,1762$ & $6:00$ & $46:00$ \\
    BOBO-160 & $0,9$ & $49,0972$  & $44,9824$ & $6:00$ & $46:00$ \\
    BOBO-10 & $0,1$ & $29,3038$  & $30,5376$ & $1:30$ & $2:00$ \\
    BOBO-10 & $0,3$ & $25,9363$  & $26,6800$ & $1:30$ & $2:00$ \\
    BOBO-10 & $0,5$ & $20,9841$  & $22,9482$ & $1:30$ & $2:00$ \\
    BOBO-10 & $0,7$ & $22,3052$  & $23,2333$ & $1:30$ & $2:00$ \\
    BOBO-10 & $0,9$ & $18,8381$  & $21,9347$ & $1:30$ & $2:00$ \\
    BOBO-ex & $0,1$ & $35,5786$  & - & $14:00$ & - \\
    BOBO-ex & $0,3$ & $35,4117$  & - & $14:00$ & - \\
    BOBO-ex & $0,5$ & $39,4408$  & - & $14:00$ & - \\
    BOBO-ex & $0,7$ & $45,0940$  & - & $14:00$ & - \\
    BOBO-ex & $0,9$ & $51,2695$  & - & $14:00$ & - \\
    \hline
    \end{tabular}
  }
  \caption {Valor función objetivo y tiempo de ejecución para la búsqueda de papers similares}
\end{table}


Al comparar \texttt{BOBO-10} y \texttt{BOBO-160} usando la selección simple se observa que la 
diferencia entre los valores de la función objetivo de cada solución aumenta a medida que $\gamma$ 
se acerca a $1$. Suponemos que esto se debe a la estrategía \texttt{produce and choose}. El objetivo 
de la primer etapa es producir bundles con máximo valor de similitud. En el caso de que se requiera 
una solución con mayor separación entre bundles, al haber producido menos cantidad de bundles 
existen menos posibilidades para generar una solución más dispersa. Un mismo análisis se podría 
hacer si comparamos \texttt{BOBO-10} y \texttt{HAC}.\\
\colorbox{red}{ANALIZAR \texttt{BOBO-160} y \texttt{HAC}}\\
Comparando los resultados obtenidos al realizar la selección de a un candidato contra la selección 
de a pares obtuvimos que para \texttt{BOBO-160} y \texttt{HAC} los tiempos aumentaron a $40$ 
minutos y para \texttt{BOBO-10} a $2$ minutos. En cuanto al valor de la función objetivo el único 
beneficiado fue \texttt{BOBO-10} ya que para \texttt{HAC} empeoró y para \texttt{BOBO-160} el 
aumento fue muy pequeño en comparación al incremento de tiempo.
\section{Autores}
Se genraron soluciones con las siguientes características:\\
\Solucion
{}
{simple y proporcional}
{\texttt{HAC} y \texttt{BOBO-x}, con  $x \in$ $(10, 160)$ y \texttt{BOBO-ex}}
{$\in$ $(0,1; 0,3; 0,5; 0,7; 0,9)$}
{10 y 20}
{5 y 10}
\begin{table}[h]
\centering
  \resizebox{\textwidth}{!} {
    \begin{tabular}{|lc|cccc|}
    \hline
    ~  & ~ & \multicolumn{2}{|c}{Valor función objectivo} & \multicolumn{2}{c|}{Duración de la 
ejecución (mm:ss)} \\
    Algoritmo & gamma & Selección simple & Selección proporcional & Selección simple          
         & Selección proporcional \\ 
    \hline
    HAC & $0,1$ & $50,5$  & $50,5$ & $8:40$ & $9:00$ \\
    HAC & $0,3$ & $61,5$  & $61,5$ & $8:40$ & $9:00$ \\
    HAC & $0,5$ & $72,5$  & $72,5$ & $8:40$ & $9:00$ \\
    HAC & $0,7$ & $83,5$  & $83,5$ & $8:40$ & $9:00$ \\
    HAC & $0,9$ & $94,5$  & $94,5$ & $8:40$ & $9:00$ \\
    BOBO-160 & $0,1$ & $38,6883$  & $36,8917$ & $10:00$ & $8:00$ \\
    BOBO-160 & $0,3$ & $43,4380$  & $41,4767$ & $10:00$ & $8:00$ \\
    BOBO-160 & $0,5$ & $47,3612$  & $47,9337$ & $10:00$ & $8:00$ \\
    BOBO-160 & $0,7$ & $51,5712$  & $52,1462$ & $10:00$ & $8:00$ \\
    BOBO-160 & $0,9$ & $57,2009$  & $57,5260$ & $10:00$ & $8:00$ \\
    BOBO-10 & $0,1$ & $30,2956$  & $31,6080$ & $2:30$ & $2:30$ \\
    BOBO-10 & $0,3$ & $33,6794$  & $35,4411$ & $2:30$ & $2:30$ \\
    BOBO-10 & $0,5$ & $33,0506$  & $37,5776$ & $2:30$ & $2:30$ \\
    BOBO-10 & $0,7$ & $37,2855$  & $34,5657$ & $2:30$ & $2:30$ \\
    BOBO-10 & $0,9$ & $41,0119$  & $35,2511$ & $2:30$ & $2:30$ \\
    BOBO-ex & $0,1$ & $39,9767$  & $39,9767$ & $27:00$ & $27:00$ \\
    BOBO-ex & $0,3$ & $44,0043$  & $44,0043$ & $27:00$ & $27:00$ \\
    BOBO-ex & $0,5$ & $48,5481$  & $48,5481$ & $27:00$ & $27:00$ \\
    BOBO-ex & $0,7$ & $53,1993$  & $53,1993$ & $27:00$ & $27:00$ \\
    BOBO-ex & $0,9$ & $57,7400$  & $57,7400$ & $27:00$ & $27:00$ \\
    \hline
    \end{tabular}
  }
  \caption {Valor función objetivo y tiempo de ejecución para la búsqueda de autores similares}
\end{table}
Para todas las soluciones obtenidas con la generación \texttt{HAC} y utilizando cualquiera de los 
dos algoritmos de selección de selección, todas las soluciones formaron los mismos bundles a 
pesar de que el $\gamma$ sea distinto.\\ 
Se realizaron otras búsquedas de soluciones, excluyendo a cinco de los autores que están presentes 
en todas las soluciones. En las nuevas soluciones obtenidas se repite nuevamente el comportamiento 
que para todos los $\gamma$, con las soluciones idénticas y valores de la función objetivo coinciden 
con los de las soluciones obtenidas sin excluir autores.\\
Con el algoritmo \texttt{HAC} todas las soluciones fueron idénticas para todos los $\gamma$. Esto 
se debe a que existen $40000$ relaciones de similitud con valor uno. Con la heurística Produce and 
Choose, al momento de producir no se tiene en cuenta el $\gamma$ por lo tanto para todos los 
$\gamma$ en la etapa de producción se producen los mismos bundles.\\
Por otro lado todas las soluciones generadas por el algoritmo \texttt{HAC} prácticamente no 
comparten bundles similares con las demás soluciones, ni siquiera autores similares en toda la 
solución.
