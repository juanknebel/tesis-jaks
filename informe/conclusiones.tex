\section{Comparaciones entre los diferentes algoritmos}\label{conc:compDifAlgo}
\subsection{Papers}
Con los resultados obtenidos podemos hacer dos tipos de análisis y comparaciones, la primera y más 
clara es comparar entre los algoritmos \texttt{SingleHAC} y \texttt{EfficientHAC}, en la que 
observamos que para $\gamma$ bajos obtuvimos los mismos resultados, pero para $\gamma$ altos a 
partir de $0.7$ obtuvimos soluciones diferentes las cuáles tenían un valor con respecto a la 
función objetivo prácticamente iguales pero se obtuvieron mejores relaciones interbundles.\\
La otra comparación que surge es entre \texttt{EfficientHAC} y \texttt{Greedy} en el que se observa 
que para $\gamma$ las soluciones obtenidas son \textquotedblleft similarmente 
buenas\textquotedblright , dado que sus valores en términos de función objetivo están cercanos, 
pero a la vez observamos que la relación interbundles es mejor las soluciones de la implementación 
\texttt{Greedy}. En cambio para $\gamma$ altos, el valor de la función objetivo en la 
implementación \texttt{EfficientHAC} fue muy superior pero la relación interbundles en el algoritmo 
\texttt{Greedy} se ve que fue mejor, logrando soluciones mas interdependientes.

\section{Valores cercanos al óptimo}
Dada la función que se está intentando maximizar $$\displaystyle\sum_{1 \leq i \leq k} 
\displaystyle\sum_{u,v \in S_{i}} \gamma s(u,v)\ 
+\ \displaystyle\sum_{1 \leq i \leq j \leq k} (1-\gamma) (1 - \displaystyle\max_{u \in S_{1}, v 
\in S_{j}} s(u,v))$$ podemos hacer suposiciones para ver que tan lejos o cerca están de algunas de 
los valores óptimos según el $\gamma$.\\
Para las siguientes cotas supongamos escenarios ideales en el cuál la soluciones contiene bundles 
donde todos sus elementos tienen una similitud máxima (igual a 1) y los bundles entre si son 
completamente diferentes, o sea, la similitud entre cada bundle es 0. Con estas hipótesis podemos 
ver que la reemplazar la función por $$\displaystyle\sum_{1 \leq i \leq k} 
\displaystyle\sum_{u,v \in S_{i}} \gamma 1\ 
+\ \displaystyle\sum_{1 \leq i \leq j \leq k} (1-\gamma) (1 - \displaystyle\max_{u \in S_{1}, v 
\in S_{j}} 0)$$ que luego se transforma en $$\displaystyle\sum_{1 \leq i \leq k} 
\displaystyle\sum_{u,v \in S_{i}} \gamma 1\ 
+\ \displaystyle\sum_{1 \leq i \leq j \leq k} (1-\gamma) 1$$ como en nuestro caso $k\ =\ 10$ y la 
cantidad de items por bundle es $5$ entonces la sumatoria se resume en $$\displaystyle\gamma\ 100\ 
+\ (1-\gamma)\ 45$$
Para los $\gamma \in (0.1, 0.3, 0.5, 0.7, 0.9)$ los resultados de la función son los siguientes:\\
\begin{table}[H]
  \centering
  \resizebox{0.5\textwidth}{!} {
    \begin{tabular}{|lc|}
    \hline
    $\gamma$ & Valor función objetivo \\
    \hline
    $0.1$  & $50.5$ \\
    $0.3$  & $61.5$ \\
    $0.5$  & $72.5$ \\
    $0.7$  & $83.5$ \\
    $0.9$  & $94.5$ \\
    \hline
    \end{tabular}
  }
    \caption {Valor óptimo de la función para cada $\gamma$}
\end{table}

En las ejecuciones del algoritmo \textbf{SingleHAC} para artículos en los $\gamma$ más altos los 
resultados de los bundles eran los mismos lo cuál se explica porque a medida que $\gamma$ aumenta se 
acerca cada vez más a la cota superior que establecimos. No sucediendo los mismo con $\gamma$ 
menores.\\
En cambio para las ejecuciones también del algoritmo \textbf{SingleHAC} pero para autores en todas 
las soluciones obtuvimos los mismos bundles y se ve reflejado porque la función objetivo para cada 
$\gamma$ es igual a la cota.
