La búsqueda y recuperación de la información es una de las aplicaciones más utilizadas en Internet. El estudio de esta problemática y la propuesta de nuevos modelos para satisfacer los distintos requerimientos de los usuarios, es una necesidad y una importante motivación para la investigación. En este trabajo se propusieron significativas mejoras sobre los algoritmos originalmente desarrollados en \cite{journals/tkde/Amer-YahiaBCFMZ14}. Al mismo tiempo agregamos: una nueva forma de hallar soluciones (búsqueda golosa) y dos estrategias (búsquedas tabú) que colaboran con las técnicas de búsquedas aquí presentadas.

A la luz de los resultados observados en el capítulo anterior, podemos concluir que los cambios propuestos en los algoritmos del tipo $PAC$ fueron exitosos en términos de los valores de la función objetivo. Si bien en muy pocos casos notamos una disminución del valor total de la solución, en el resto de los escenarios fueron superiores. 

En la etapa de producción de los paquetes y en la selección de los mismos, la elección de utilizar las dos medidas \texttt{inter} e \texttt{intra} resultó acertada cuando los valores de $\gamma$ eran menores. Lo cuál es correcto, como vimos éstos son los casos en los cuales más internviene el valor \texttt{inter}.

El uso de las estrategias tabú resultó ser una opción muy adecuada para todos los tipos de búsquedas realizadas en el desarrollo de esta tesis. Sobre todo la facilidad de poder usarse a partir de cualquier solución inicial sin importar la estrategia elegida al inicio de la pesquisa.

En cuanto a la nueva heurística de la familia de algoritmos golosos, tiene la ventaja de contar con implementaciones simples y fácilmente adaptables a los cambios. Conviretiendola en una alternativa muy interesante, en aquellos casos en los cuales sea aceptable un deterioro en la calidad de la solución en pos de un mayor tiempo de respuesta. Mejor aún si se lo convina con una búsqueda tabú, que probó ser un complemento muy útil, mejorando en la mayoría de los casos la calidad de la solución final, sin pérdida de performance. 

Un punto en el que se deberá trabajar en futuros desarrollos es la performance. Si bien el objetivo principal del trabajo no estuvo enfocado en el rendimiento de los algoritmos, en cambio si, en que los algoritmos tengan una complejidad acorde. De esta manera evitamos realilzar optimazación prematura y centrarnos en nuestro objetivo principal. Igualmente los tiempos obtenidos estuvieron dentro de los rangos necesarios para poder realizar todas las pruebas necesarias.

\paragraph{Trabajo futuro} Como parte final del trabajo, se describen los temas que quedaron abiertos para trabajos futuros. 



Uno de los principales es que el algoritmo pueda generar soluciones en el que un ítem esté presente en distintos paquetes. A pesar de que no es parte del problema, sería interesante que cada paquete haga un cubrimiento de un tema particular. Como por ejemplo, el escenario en cuál un cliente visita una disqueria virtual -presentado en la introducción-, cada paquete dentro del resultado entregado contenga discos de todos los géneros que le gustan al cliente.

Otro tema que surgió es adaptar los algoritmos para que se ejecuten en paralelo.