\section{Búsquedas generales}
A partir de los resultados obtenidos en las pruebas realizadas la aplicación de las búsquedas tabú siempre mejoró los resultados en cualquier circunstancia y sin modificar significativamente el tiempo de ejecución. En el caso de PAC con la generación de bundles con BOBO, aplicando la búsqueda tabú el resultado mejoro considerablemente.\\
Comparando entre el algoritmo goloso y PAC, se obtuvo que el algoritmo goloso dio mejores resultados cuando la producción de bundles se hizo con la estrategia BOBO pero no así con la estrategia de producción jerárquica. Como toda la familia de algoritmos golosos tienen la ventaja de poder ser escritos muy rápidamente, convirtiéndolos en una alternativa muy interesante en los casos que se necesite una solución a este tipo de problemas y sea aceptable que las soluciones no sean las mejores con respecto a otros métodos ya conocidos.\\
Las implementaciones de BOBO fueron en todos los casos las más rápidas resultando ser las ideales para instancias muy grandes sobre todo si se aplica al final una búsqueda tabú, que como se observó mejora mucho las soluciones en estos casos. En cambio HAC en las pruebas realizadas con un universo de elementos menor a los 10000, cuadruplica o quintuplica a los tiempos debido a su complejidad ($\mathcal{O}(N^{2}\lg n)$), pero en instancias más chicas como las atracciones turísticas los tiempos fueron muy cercanos entre ellos.\\
En las ejecuciones realizadas con la implementación del algoritmo \textbf{HAC}, \texttt{SingleHAC} y \texttt{EfficientHAC}, se observó que para valores de $\gamma$ bajos se generaban las mismos soluciones, en cambio para $\gamma$ mayores $0.7$ se obtenían diferentes. Dichas soluciones tenían un valor, con respecto a la función objetivo, prácticamente igual pero se sus relaciones interbundles eran mejores.\\
En los experimentos realizados el uso de una variante u otro no resultó ser significativo, al menos con los datos que se usaron.\\
Las soluciones obtenidas usando HAC y búsquedas golosas para valores de $\gamma$ menores a $0.5$ fueron \textquotedblleft similarmente buenas\textquotedblright , dado que los resutlados en términos de función objetivo fueron cercanos. En cambio para $\gamma$ mayores se observó que las soluciones HAC resultaron mejores con valores significativamente mejores. En todos los casos se identificó que las relaciones interbundles son mejores en las soluciones del algoritmo goloso, logrando soluciones más interdependientes.

\section{Valores cercanos a la cota superior}\label{conc:valoresOptimos}
Dada la función que se está intentando maximizar $$\displaystyle\sum_{1 \leq i \leq k} \displaystyle\sum_{u,v \in S_{i}} \gamma s(u,v)\ +\ \displaystyle\sum_{1 \leq i \leq j \leq k} (1-\gamma) (1 - \displaystyle\max_{u \in S_{1}, v \in S_{j}} s(u,v))$$ se pueden hacer suposiciones para ver que tan lejos o cerca están de algunas de los valores óptimos según el $\gamma$.\\
Para las siguientes cotas se suponen escenarios ideales en el cuál la soluciones contiene bundles donde todos sus elementos tienen una similitud máxima (igual a 1) y los bundles entre si son completamente diferentes, o sea, la similitud entre cada bundle es 0. Con estas hipótesis se puede ver que la reemplazar la función por $$\displaystyle\sum_{1 \leq i \leq k} \displaystyle\sum_{u,v \in S_{i}} \gamma 1\ +\ \displaystyle\sum_{1 \leq i \leq j \leq k} (1-\gamma) (1 - \displaystyle\max_{u \in S_{1}, v \in S_{j}} 0)$$ que luego se transforma en $$\displaystyle\sum_{1 \leq i \leq k} \displaystyle\sum_{u,v \in S_{i}} \gamma 1\ +\ \displaystyle\sum_{1 \leq i \leq j \leq k} (1-\gamma) 1$$ como en nuestro caso $k\ =\ 10$ y la cantidad de items por bundle es $5$ entonces la sumatoria se resume en $$\displaystyle\gamma\ 100\ +\ (1-\gamma)\ 45$$
Para los $\gamma \in (0.1, 0.3, 0.5, 0.7, 0.9)$ los resultados de la función son los siguientes:\\
\begin{table}[H]
  \centering
  \resizebox{0.5\textwidth}{!} {
    \begin{tabular}{|lc|}
    \hline
    $\gamma$ & Valor función objetivo \\
    \hline
    $0.1$  & $50.5$ \\
    $0.3$  & $61.5$ \\
    $0.5$  & $72.5$ \\
    $0.7$  & $83.5$ \\
    $0.9$  & $94.5$ \\
    \hline
    \end{tabular}
  }
    \caption {Cota superior de la función para cada $\gamma$}
\end{table}
