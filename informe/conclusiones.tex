La búsqueda y recuperación de la información es una de las aplicaciones más utilizadas en Internet. El estudio de esa problemática, y la propuesta de nuevos modelos para satisfacer distintos requerimientos de los usuarios es por lo tanto una necesidad y una importante motivación para la investigación. En este trabajo se propusieron significativas mejoras a algoritmos para realizar búsquedas de información alternativas a las tradicionales.

De acuerdo a la experiencia computacional, los algoritmos propuestos resultaron más eficientes que los presentes en la literatura tanto en valor de función objetivo como en su sensibilidad al criterio de valorización de cada objetivo, sin perjuicio en el tiempo de ejecución. En el caso de PAC con la generación de paquetes con BOBO, aplicando la búsqueda tabú el resultado mejoro considerablemente.

En la comparación entre el algoritmo goloso y PAC, se observó que con el algoritmo goloso se obtuvieron mejores resultados qué cuando la producción de los paquetes se hizo con la estrategia BOBO pero no así con la estrategia de producción jerárquica. Como toda la familia de algoritmos golosos tiene la ventaja de la simplicidad de su implementación, convirtiéndolos en una alternativa muy interesante en los casos que se necesite una solución a este tipo de problemas en los que es aceptable el deterioro de la calidad de las soluciones en pos de mejorar la performance.

Para la estrategia PAC las implementaciones de BOBO fueron en todos los casos las más rápidas, resultando ser las ideales para instancias de tamaño muy grandes. La solución obtenida mejora considerablemente Si se aplica al final una búsqueda tabú. En las pruebas realizadas con HAC con un universo de elementos menor a los 10000 se cuadruplica o quintuplica los tiempos de ejecución debido a su complejidad ($\mathcal{O}(N^{2}\lg n)$). Sin embargo en instancias más chicas como las atracciones turísticas los tiempos fueron muy cercanos entre ellos.

Las soluciones obtenidas usando Efficient-HAC y búsquedas golosas para valores de $\gamma$ menores a $0.5$ fueron \textquotedblleft similarmente buenas\textquotedblright , dado que los resultados en términos de función objetivo fueron cercanos. En cambio para $\gamma$ más cercanos a uno se observó que el valor de la función objetivo de las soluciones HAC resultaron significativamente mayores. En todos los casos se identificó que las relaciones interpaquetes son mejores en las soluciones del algoritmo goloso, logrando soluciones más interdependientes.

Al comparar \texttt{BOBO-10} y \texttt{BOBO-160} usando la selección simple se observa que la diferencia entre los valores de la función objetivo de cada solución aumenta a medida que $\gamma$ se acerca a $1$. Se supone que esto se debe a la estrategia \texttt{produce and choose}. El objetivo de la primer etapa es producir paquetes con máximo valor de similitud. En el caso de que se requiera una solución con mayor separación entre paquetes, al haber producido menos cantidad de paquetes existen menos posibilidades para generar una solución más dispersa. Un mismo análisis se podría hacer si se compara \texttt{BOBO-10} y \texttt{HAC}.

Comparando los resultados obtenidos al realizar la selección de a un candidato contra la selección de a pares se obtuvo que para \texttt{BOBO-160} y \texttt{HAC} los tiempos aumentaron a $40$ minutos y para \texttt{BOBO-10} a $2$ minutos. En cuanto al valor de la función objetivo el único beneficiado fue \texttt{BOBO-10} ya que para \texttt{HAC} empeoró y para \texttt{BOBO-160} el aumento fue muy pequeño en comparación al incremento de tiempo.


Como parte final del trabajo, se describen los temas que quedaron abiertos para trabajos futuros. Uno de los principales es que el algoritmo pueda generar soluciones en el que un ítem este presente en distintos paquetes. A pesar de que no es parte del problema, sería interesante que cada paquetes haga un cubrimiento de un tema. Con esto se quiere decir por que en el ejemplo del cliente de la disqueria -presentado en la introducción- que cada paquete del resultado contenga discos de todos los géneros que que le gustan al cliente.

Otro tema que surgió es adaptar los algoritmos para que se ejecuten en paralelo.

 

\section{Desarrollo}
\begin{itemize}
 \item Mejorar las búsquedas tabú para encontrar mejores soluciones.
 \item Nuevas heurísticas para intentar mejorar las soluciones.
\end{itemize}

 
\section{User friendly}
\begin{itemize}
 \item Generar una interfaz de usuario amigable.
 \item Una mejor manera de visualizar los resultados de manera on line.
\end{itemize}

\section{Performance}
\begin{itemize}
 \item Acelerar los tiempos de ejecución.
 \item Ejecución en diferentes hilos.
 \item Tunning de la base de datos.
\end{itemize}

\section{Extensibilidad}
\begin{itemize}
 \item Generar una interfaz genérica para poder trabajar con datos de cualquier origen y poder ser explotados de la misma forma.
\end{itemize}
