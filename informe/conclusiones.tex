\section{Búsquedas generales}
A partir de los resultados obtenidos en las pruebas realizadas la aplicación de las búsquedas tabú siempre mejoró los resultados en cualquier circunstancia y sin modificar significativamente el tiempo de ejecución. En el caso de PAC con la generación de paquetes con BOBO, aplicando la búsqueda tabú el resultado mejoro considerablemente.

Comparando entre el algoritmo goloso y PAC, se obtuvo que el algoritmo goloso dio mejores resultados cuando la producción de paquetes se hizo con la estrategia BOBO pero no así con la estrategia de producción jerárquica. Como toda la familia de algoritmos golosos tienen la ventaja de poder ser escritos muy rápidamente, convirtiéndolos en una alternativa muy interesante en los casos que se necesite una solución a este tipo de problemas y sea aceptable que las soluciones no sean las mejores con respecto a otros métodos ya conocidos.

Las implementaciones de BOBO fueron en todos los casos las más rápidas resultando ser las ideales para instancias muy grandes sobre todo si se aplica al final una búsqueda tabú, que como se observó mejora mucho las soluciones en estos casos. En cambio HAC en las pruebas realizadas con un universo de elementos menor a los 10000, cuadruplica o quintuplica a los tiempos debido a su complejidad ($\mathcal{O}(N^{2}\lg n)$), pero en instancias más chicas como las atracciones turísticas los tiempos fueron muy cercanos entre ellos.

Las soluciones obtenidas usando Efficient-HAC y búsquedas golosas para valores de $\gamma$ menores a $0.5$ fueron \textquotedblleft similarmente buenas\textquotedblright , dado que los resultados en términos de función objetivo fueron cercanos. En cambio para $\gamma$ más cercanos a uno se observó que el valor de la función objetivo de las soluciones HAC resultaron significativamente mayores. En todos los casos se identificó que las relaciones interpaquetes son mejores en las soluciones del algoritmo goloso, logrando soluciones más interdependientes.

Al comparar \texttt{BOBO-10} y \texttt{BOBO-160} usando la selección simple se observa que la diferencia entre los valores de la función objetivo de cada solución aumenta a medida que $\gamma$ se acerca a $1$. Se supone que esto se debe a la estrategia \texttt{produce and choose}. El objetivo de la primer etapa es producir paquetes con máximo valor de similitud. En el caso de que se requiera una solución con mayor separación entre paquetes, al haber producido menos cantidad de paquetes existen menos posibilidades para generar una solución más dispersa. Un mismo análisis se podría hacer si se compara \texttt{BOBO-10} y \texttt{HAC}.

Comparando los resultados obtenidos al realizar la selección de a un candidato contra la selección de a pares se obtuvo que para \texttt{BOBO-160} y \texttt{HAC} los tiempos aumentaron a $40$ minutos y para \texttt{BOBO-10} a $2$ minutos. En cuanto al valor de la función objetivo el único beneficiado fue \texttt{BOBO-10} ya que para \texttt{HAC} empeoró y para \texttt{BOBO-160} el aumento fue muy pequeño en comparación al incremento de tiempo.
