%\documentclass[11pt,a4paper,twoside]{tesis}
% SI NO PENSAS IMPRIMIRLO EN FORMATO LIBRO PODES USAR
\documentclass[11pt,a4paper]{tesis}

\usepackage{graphicx}
\usepackage[utf8]{inputenc}
\usepackage[spanish]{babel}
\usepackage[left=3cm,right=3cm,bottom=3.5cm,top=3.5cm]{geometry}

\usepackage{color}
\usepackage{solucion}
\usepackage{multicol}
\usepackage{algorithm}
\usepackage{algorithmic}
\usepackage{xcolor}
\usepackage{colortbl}
\usepackage{lscape}
\usepackage{longtable}
\usepackage{amsmath}
\usepackage{amssymb}
\usepackage{textcomp}
\usepackage{caption}
\usepackage{subcaption}
\usepackage{hyperref}
\usepackage{todonotes}
\hypersetup{
    colorlinks,
    citecolor=black,
    filecolor=black,
    linkcolor=black,
    urlcolor=black
}
\renewcommand{\algorithmicrequire}{\textbf{Input:}}
\renewcommand{\algorithmicensure}{\textbf{Output:}}
\newcommand{\algorithmicbreak}{\textbf{break}}
\newcommand{\BREAK}{\STATE \algorithmicbreak}
\DeclareMathOperator*{\argmin}{arg\,min}
\DeclareMathOperator*{\argmax}{arg\,max}
\begin{document}

%%%% CARATULA
% Comentar y descomentar según corresponda
\def\titulo{Licenciado }
\def\autor{Amit Stein, Juan Andrés Knebel}
\def\tituloTesis{Nuevos algoritmos para recuperación de ``ítems empaquetados'': \mbox{Algo}}
\def\runtitulo{Nuevos algoritmos para recuperación de ``ítems empaquetados''}
\def\runtitle{New Algorithms for composite retrival}
\def\director{Obi-Wan Kenobi}
\def\codirector{Master Yoda}
\def\lugar{Buenos Aires, 2014}
%\input{caratula}

%%%% ABSTRACTS, AGRADECIMIENTOS Y DEDICATORIA
\frontmatter
\pagestyle{empty}
%\begin{center}
%\large \bf \runtitulo
%\end{center}
%\vspace{1cm}
\chapter*{\runtitulo}

\noindent Las búsquedas tradicionales ofrecen soluciones que solo tienen en cuenta un solo atributo de los elementos y no la relación que estos tienen con el resto del universo. Las mismas suelen ofrecer una lista ordenada de resultados relacionadas con
el criterio utilizado. Ocasionando, muchas veces, la necesidad de reformular la consulta original para así lograr una solución adecuada al criterio de búsqueda elegido previamente.\\
El artículo Composite Retrieval of Diverse and Complementary Bundles \cite{compositeRetrival}, propone que el resultado de la búsqueda este compuesta por conjuntos de elementos. Estos conjuntos, conocidos como bundles, tienen la particularidad que los elementos que contienen se encuentran relacionados internamente bajo algún criterio de similitud y a la vez son complementarios. De forma tal que la relación de los elementos de un bundle satisfagan las expectativas del usuario y no tenga la necesidad de realizar una nueva intervención logrando así una mejor experiencia de búsqueda.\\
En este trabajo se implementa este tipo de resultados para consultas realizadas sobre la base de datos de \cite{dataDrive}, que contiene artículos científicos pertenecientes a la Ingeniería de Software. Para ello se utilizan los algoritmos de \cite{compositeRetrival} a los que se le aplican modificaciones para optimizar el rendimiento y se proponen nuevos enfoques en orden de mejorar los resultados que se obtuvieron originalmente.

\bigskip

\noindent\textbf{Palabras claves:} Composite Retrieval, Agrupamiento, Similitud, Búsqueda Tabú, Algoritmo Goloso.

\cleardoublepage
%%\begin{center}
%\large \bf \runtitle
%\end{center}
%\vspace{1cm}
\chapter*{\runtitle}

\noindent 

Los usuarios de Internet, u otros medios informáticos, constantemente realizan búsquedas para hallar elementos de su interés, generalmente a través de palabras o frases. A lo largo del desarrollo de Internet las búsquedas fueron adquiriendo cada vez más importancia por la enorme cantidad de información que día a día se almacena en los distintos servidores del planeta.

Las búsquedas tradicionales ofrecen soluciones que únicamente tienen en cuenta un atributo de los elementos y no la relación que éstos tienen con el resto del universo. Las mismas suelen ofrecer una lista ordenada de resultados relacionados con el criterio de búsqueda utilizado. Muchas veces esto ocasiona la necesidad de reformular la consulta original para así lograr una solución adecuada al criterio de búsqueda elegido previamente.

Como respuesta a este último comportamiento se propone que el resultado de la búsqueda esté compuesta por conjuntos de elementos. Estos conjuntos, conocidos como {\em paquetes}, tienen la particularidad de que los elementos conformantes se encuentran relacionados internamente bajo algún criterio de similitud y a la vez son complementarios. El propósito es que la relación de los elementos de un paquete satisfaga las expectativas del usuario y no tenga la necesidad de realizar una nueva intervención, logrando así una mejor experiencia de búsqueda.

Este tipo de búsqueda de elementos complementarios y compatibles, conectados mediante algún tipo de relación, generó el estudio de la {\em recuperación compuesta de información}. Es decir, el estudio de métodos para crear, recuperar y valorizar respuestas compuestas, organizando los resultados en paquetes de elementos. Esto ayuda a los usuarios a explorar un gran número de elementos relevantes de una forma más eficiente.

En este trabajo se proponen algoritmos heurísticos para realizar este tipo de búsquedas. Los mismos son experimentalmente evaluados en dos bases de datos una de artículos relacionados con ingeniería de software y otra de atracciones turísticas, a partir de esta evaluación podemos concluir que tienen una muy buena performance, mejorando los existentes en la literatura.
\bigskip


\noindent\textbf{Palabras claves:} Recuperación de la información, Agrupamiento, Heuristícas.

%\cleardoublepage
%\chapter*{Agradecimientos}

\noindent A Mirta que nos desleita con sus almuerzos y preguntas sobresalientes.
\noindent A Laura que estuvo a mi lado
\noindent A mis padres, a Abigail y René y a Edith que me acompañaron y ayudaron durante toda la carrera y A Mila y Florián que lo hicieron este último tramo. 
\noindent a todos los profesores y ayudantes del DC y en especial a Paula, Isabel y Estebán por todo lo enseñado.
\noindent A Dario, Julio y Sebastián con los que además de compañeros nos hicimos buenos amigos.
 % OPCIONAL: comentar si no se quiere

%\cleardoublepage
%\hfill \textit{A Nestor y Chavez que nos cuidan y guian desde arriba}  % OPCIONAL: comentar si no se quiere

\cleardoublepage
\tableofcontents

\mainmatter
\pagestyle{headings}

%%%% ACA VA EL CONTENIDO DE LA TESIS

\chapter{Introducción}
\label{chap:introduccion}
\section{Motivación}
{\begin{small}%
\begin{flushright}%
\it An algorithm must be seen to be believed.\\Donald Knuth.
\end{flushright}%
\end{small}%
\vspace{.5cm}}
La recuperación de información (Information Retrieval en inglés) es la actividad de obtener información relevante a partir de una necesidad sobre una colección de recursos de información.  Los criterios de las búsquedas pueden estar basados en metadatos o en contenido.    
Este trabajo se basa en el artículo \textit{\textquotedblleft Composite Retrieval of Diverse and Complementary Bundles\textquotedblright}\cite{compositeRetrival} donde se propone que los resultados de las búsqueda se devuelva en grupos de ítems, con la finalidad de que uno de estos grupos satisfaga las expectativas del usuario. Cada uno de los grupos se forma por elementos similares que difieren en algún atributo sin excederse del presupuesto. El objetivo es generar grupos de ítems complementarios lo más cohesivos posibles y que los grupos entre si sean distintos. De este modo al usuario se le ofrece una gran variedad de ítems ordenados de forma lógica para facilitar la selección.\\
En una búsqueda típica los resultados obtenidos son lineales al criterio seleccionado y no otorgan respuestas que relacionen el criterio buscado con otros elementos que se relacionan. Un posible resultado en un búsqueda de discos de música podría ser:\\
\begin{itemize}
  \item Physical Graffiti - Led Zeppelin
  \item Led Zeppelin - Led Zeppelin
  \item It's Hard - The Who
  \item Perfect Strangers - Deep Purple
  \item El Cielo Puede Esperar - Attaque 77
  \item wheels of fire - Cream
  \item Confesiones de Invierno - Sui Generis
  \item The White Album - The Beatles
  \item Innuendo - Queen
  \item Sticky Fingers - The Rolling Stones
  \item Kamikaze - Luis Alberto Spinetta
\end{itemize}

Si el usuario está interesado en comprar discos, cuenta con un presupuesto limitado y le interesa la diversidad de los artistas a adquirir, con este resultado lineal, el usuario debería realizar reiteradas búsquedas para encontrar los discos de los artistas que le interesen.\\


El resultado que se propone está diseñado para aquellas consultas que requieren obtener un conjunto de elementos que se relacionan como respuesta. En las técnicas tradicionales de clusterización la agrupación se hace por la similitud entre ítems. En el ejemplo de los discos con una clusterización tradicional, donde la similitud sea el género musical, seguramente se generen tantos clúster como géneros de discos existan y en cada clúster estén todos los discos de ese género. Con este resultado se deberá explorar todo el clúster para elegir los discos; donde la información es redúndate porque algunos discos son tan parecidos que no se comprarían juntos.\\
El resultado que se obtiene con ``Composite Retrieval of Diverse and Complementary Bundles'' se ajuste al presupuesto y a que los ítems sean complementarios, de modo tal que el usuario seleccionando un bundle (es el nombre que se le da al agrupamiento de ítems) cubra su necesidad. Con los discos se establece la complementariedad de algún atributo, como puede ser el origen de la banda, cada bundle es una opción de discos que el usuario puede comprar porque cumplen con el presupuesto, son del mismo género de música y hay variedad entre ellos. Una solución posible es:
\begin{itemize}
  \item Bundle 1:
  \begin{itemize}
    \item Physical Graffiti - Led Zeppelin (Inglaterra)
    \item After chabón - Sumo (Argentina)
    \item Back in Black - AC/DC (Estados Unidos)
  \end{itemize}
  \item Bundle 2:
  \begin{itemize}
    \item Natty Dread - Bob Marley (Jamaica)
    \item El ritual de la Banana - Los Pericos (Argentina)
    \item Labour of Love - UB40 (Inglaterra)
  \end{itemize}
	  \item Bundle 3:
  \begin{itemize}
    \item Ramones - Ramones (Estados Unidos)
    \item El Cielo Puede Esperar - Attaque 77 (Argentina)
    \item Sandinista! - The Clash (Inglaterra)
  \end{itemize}
\end{itemize}

Lo que se propone es otorgar un conjunto de \textbf{bundles} que cumplen con las siguientes propiedades:
\begin{itemize}
  \item \textbf{Cubrimiento}: Maximizar la cantidad de elementos en el bundle.
  \item \textbf{Compatibilidad}: Los elementos del bundle deben ser similares.
  \item \textbf{Validez}: El costo total de los elementos del bundle no debe superar el presupuesto.
  \item \textbf{Diversificada}: Los bundles entre si deben ser diversos.
\end{itemize}

\section{Instancia elegida}
Este trabajo consiste en implementar el diseño de resultados de \textit{\textquotedblleft Composite Retrieval of Diverse and Complementary Bundles\textquotedblright}\cite{compositeRetrival} de consultas sobre la base de datos de los artículos de \textit{\textquotedblleft A Data-Driven Journey through Software Engineering Research\textquotedblright}\cite{dataDrive}. La decisión de utilizar esta base de datos es por la completitud de la información y que el tamaño de la cantidad de elementos que contiene requiere de optimizar los algoritmos propuestos.\\
La base de datos contiene cerca de $7800$ artículos, de los que se tiene los autores que participaron en ellos, en la conferencia que fueron presentados y lo más importante es que se cuenta con una clasificación ya realizada de los tópicos a los que hacen referencia cada uno de ellos. Ésta última característica es llamada \texttt{topicProfile} y esta expresada en porcentajes para cada uno de tópicos que son tratados.\\
Cuenta con $9800$ autores y de ellos tenemos la información de que a universidad pertenecen, que será útil como criterio de diversidad y además de cada una de las universidades se sabe a que región pertenece cada una.\\
El \texttt{topicProfile} es muy importante porque, como veremos más adelante, nos permitirá definir la similitud no sólo entre los artículos, sino que también entre los autores y las universidades de la base da datos de una manera prácticamente directa.

\chapter{Desarrollo}
\label{chap:desarrollo}

\section{Modelo}
Dado el conjunto de items I y la función de similitud \textbf{S}: IxI$\rightarrow$R[0;1]. El problema se puede representar como un grafo 
con peso en las aristas, en el cuál los vertices son los items y el peso de las aristas es la similitud entres estos.
\section{Problema}
El problema de obtener el conjunto de bundles que máximiza la función objetivo se puede considerar como un problema
de clusterización en el cual la calidad de los cluster esta dado por la cohesión de los items que lo componen y la
separación entre cluster con menos similitud.
Las diferencias con los problemas tipicos de clusterización son:
\begin{enumerate}
 \item La cantidad de items en un bundle esta limitada por el presupuesto.
 \item En el bundle no puede existir más de un item con el mismo atributo de complementaridad.
\end{enumerate}
\section{Función de similitud}
La similitud se emplea para comparar dos documentos y determinar que tan parecido sonentre si. La comparación se realiza en el \textit{modelo de espacio vectorial} donde los artículos se representan con vectores, en la que cada dimensión corresponde a un tópico. El valor del tópico del artículo en la base de datos, es el valor en el vector.\\
El artículo $a$ se representa con el vector $V_a = [v_1,v_2,...,v_n]$, donde $N$ es la cantidad de tópicos. Los vectores cumplen las siguientes propiedades:
\begin{enumerate}
 \item $v_i \leqslant 1$
 \item $\sum{v_i} = 1$
\end{enumerate}

Con los vectores es posible calcular el coeficiente de similitud a través de la función $ S (V_i, V_j)$ la cual refleja el grado de similitud de los pesos de los tópicos correspondientes. Existen varias medidas de similitud, En este trabajo se utilizó una de las más utilizadas que es la medida de similitud basada en el ángulo de los vectores.\\
En la medida de similitud del coseno la dirección de los vectores es lo que se utiliza para determinar que tan parecidos son entre si. Cuanto más parecido sean los documentos el ángulo de los vectores se acerca a cero. Siendo que dos vectores con identica distribución de los términos, el ángulo es cero, produciendo la máxima medida de similitud.\\
Conforme a \cite{newSimilarity} el ángulo de dos vectores proporcionales tienen la misma dirección con lo que el ángulo es cero. Entonces estos vectores son exactamente similares en cuanto a la medida de similitud basada en el ángulo. Esta medida no considera el peso de cada tópico. Por lo tanto no diferencia entre un articulo profesional y un articulo de un diario que cubre el mismo tópico. Esta debilidad de la medida basada en el ángulo no interfiere en este trabajo, por las propiedades de los vectores los documentos. Porque los documentos con tópicos similares se representan con vectores que se encuentran muy cerca en el espacio euclediano.

Dado los vectores $V_i$ y $V_j$ la función de similitud basada en el ángulo se obtiene por la definición del producto escalar.\\
\begin{equation} \label{eq:angulovectorial}
\cos(\theta) =  \dfrac{\overrightarrow{U} . \overrightarrow{V}}{\overrightarrow{\lVert V\lVert}.\overrightarrow{\lVert U\lVert}}
\end{equation}

Como los componentes de todos los vectores son mayor o igual a cero se obtien que $0\leq\cos(\theta)\geq1$. Que corresponde con la función de similitud.

\begin{figure}[H]
\includegraphics[width=0.8\textwidth]{img/coseno.png}
\caption{Comportamiento de la función $\cos$. En rojo la región que involucra los resutlados de la función de similitud}
\label{bus:img-coseno}
\end{figure}


\section{Resolución}
Encontrar una solución al problema planteado es reducible 

El algoritmo que se utilizó para obtener las soluciones es \texttt{Produce-and-Choose}, cuenta con 
dos fases.
En la primer fase se generan cierta cantidad de bundles, e la fase siguiente se seleccionan los 
bundles que serán parte de la solución.\\
A continuación se explican los algoritmos utilizados para cada fase.
\section{Generación de bundles}
Dado un conjunto de papers el objetivo es generar clusters en los cuales los papers suficientemente 
similares pertenezcan al mismo cluster y en cluster distintos los disímiles. Cuanto mayor es la 
similitud en el cluster (intra) y mayor la diferencia entre los cluster (inter) es mejor la 
clusterización.\\
Definir como se constituye un cluster es complejo. Por ejemplo para los 20 puntos que se muestran a 
continuación existen tres (o más) formas de clusterizar que son validas. Entonces la mejor 
definición depende del tipo de dato y del resultado esperado.

\begin{figure}[H]
  \centering
    \includegraphics[width=0.8\textwidth]{img/howToCluster.png}
  \caption{Agregar descripcion}
  \label{res:img-howToCluster}
\end{figure}

En la clusterización para Composite Retrival, la definición es generar cluster máximizando el costo que esta acotado por el budget.
Ya que lo esperado es obtener cluster que utilicen el máximo del presupuesto.\\

El problema de la clusterización es NP-hard (agregar ref, explicar problema de clusterizacion),
por lo que se utilizaron dos técnicas ya conocidas para aproximarse a una solución.\\
La principal diferencia entre las estrategias de clusterización es entre la jerárquica y de 
partición.\\
La primer técnica produce un árbol de particiones, la raíz es un cluster que contiene todos los items
y las hojas son clusters con un único ítem. Cada nivel intermedio, puede ser visto, como la 
combinación entre dos clusters del nivel inferior inmediato. Mientras que la segunda genera solo un 
nivel de las particiones de los items de una vez.\\ 
Se implementaron las dos técnicas para la clusterización jerárquica con el algoritmo
Hierarchical clustering y la de partición con Bundles One-By-One.\\

\subsection{Bundles One-By-One}
El método \texttt{BOBO-x} esta inspirado en k-nn. Consiste en cada paso seleccionar, de manera azarosa, un item del conjunto de pivotes (en el inicio este conjunto contiente todos los items),
con el que se genera un bundle valido a al rededor de este. En el caso de que el intra del bundle sea bueno se agrga al conjunto de candidatos de bundle.
La iteración finaliza cuando se generan la cantidad de candidatos definidos en el parámetro $x$ o cuando el conjunto de bundles es vacío. El caso de que $x$ sea 'Ex' todos los 
items son pivotes.\\
Para generar el bundle a partir del pivote, se realiza de manera golosa de eligiendo en cada iteración el item que
máximiza el intra y que cumple con las resticciones.\\
\subsection{Hierarchical clustering}
La heurística Hierarchical clustering \texttt{HAC} comienza con tantos clusters como cantidad de elementos, cada uno está 
conformado por un solo ítem y en cada paso se unen los dos clusters más cercanos que respetan las restricciones. 
Para ello se define la función de distancia para los items $u$ y $v$ como:\\
\begin{equation}
d_{1}(u,v) = 1 - s(u, v)
\end{equation}

Con la función de distancia $d_{1}$ en la clusterización se generan los cluster lo más cohesivos posibles,
En la figura 1 se observa que el algoritmo selecciona los items más cercanos. En las búsquedas que 
se realizan en ``Composite ...''\cite{compositeRetrival} se tiene el parámetro $\gamma$ que indica 
que tipo de resultado es el esperado. 

\begin{figure}[H]
  \centering
    \includegraphics[width=0.3\textwidth]{img/cluster2.png}
  \caption{Selección de bundles usando $d_{1}$}
  \label{res:img-usingEfficientHAC}
\end{figure}

En caso de que el $\gamma$ sea pequeño, la 
clusterización esperada para la misma instancia es la que se visualiza en la imágen 2, clusters no 
tan cohesivos pero más variados. Por lo que se define una función de distancia que considera el 
$\gamma$.\\
\begin{equation}
d2(u,v) = 1 - FO(\{u\} \cup \{v\})
\end{equation}
Donde $FO$ es la función definida en \eqref{eq:fnObj} \\

\begin{figure}[H]
  \centering
    \includegraphics[width=0.3\textwidth]{img/cluster1.png}
  \caption{Selección de bundles usando $d_{2}$}
  \label{res:img-usingSingleHAC}
\end{figure}

Para la distancia $d_{1}$ se implementó el algoritmo \texttt{EfficientHAC} que es tiene una 
complejidad $\mathcal{O}(n^{2})$, mientras que para $d_{2}$ el algoritmo es \texttt{SingleHAC} que 
la complejidad es $\mathcal{O}(n^{2} * \ln{n})$. Según se demostró en el capítulo 17 de 
\cite{informationRetrival}


\section{Selección de bundles}
Al finalizar la producción de bundles, se deben seleccionar los $k$ bundles para la solución.
El problema de seleccionar los bundles que maximizan la función objetivo, se traduce en 
encontrar en el grafo G el k-subgrafo de mayor peso de nodos y aristas.

Se implementó el algoritmo que propone ``Composite ...''\cite{compositeRetrival} que transforma el 
grafo G en el grafo G', con los mismos nodos y aristas, redefiniendo el peso de las aristas con la 
función:\\

\begin{equation}
\omega_{1}(u,v) = \dfrac{\gamma}{2( k - 1)} (\omega(u) + \omega(v)) + (1 - \gamma)\psi(u,v) 
\end{equation}

(aca va el algoritmo)\\

En la función $\omega_{1}(u,v)$ el valor de la función $\psi(u,v)$  es considerablemente menor al de
$\omega(u) + \omega(v)$, (en el intra se suma los ejes de todos los nodos, el inter es el máximo entre los items)
implica que $\gamma$ no cumple con el objetivo de balancear entre una solución cohesiva y una variada.
Para esto se funciones $\omega$ alternativas.\\

Con $\omega_{2}(u,v,w,y)$ la búsqueda se realiza con la combinación de cuatro nodos, por lo que el orden de complejidad
aumenta a ...

\begin{equation}
\begin{split}
\omega_{2}(u,v,w,y) &= \dfrac{\gamma}{2( k - 1)} (\omega(u) + \omega(v) + \omega(w) + \omega(y)) \\
&+ (1 - \gamma)(\psi(u,v) + \psi(u,w) + \psi(u,y)  + \psi(v,w) + \psi(v,y) + \psi(w,y))
\end{split}
\end{equation}

La función $\omega_{3}(u, k)$ recibe $u$ el conjunto de clusters seleccionados hasta el momento 
y $k$ que es la cantidad de clusters para la solución. Con $k$ y el tamaño de $u$ se calcula el 
coeficiente con el propósito en que en cada paso se pondere el inter y el intra. Para eso se multiplica
con coeficientes cada parte de la función. Con esto se mantiene la relación de inter e intra durante la selección de bundles

\begin{equation}
\begin{split}
\omega_{3}(u,k) &= \dfrac{k}{u.size} * (\gamma \sum_{v \in U}(w(v))) \\
&+ \dfrac{(k * (k-1))}{2} * \dfrac{2}{(u.size() * (u.size() - 1))} * 1 - \gamma \sum_{v,w \in U}(\psi(v,w))
\end{split}
\end{equation}

\begin{algorithm}[H]
\begin{algorithmic}[1]
\REQUIRE $produced:Vector<SnowFlake>, numRequested:Integer$
\ENSURE $selected:Vector<SnowFlake>$.
\STATE $selected:Vector<SnowFlake> \leftarrow []$
\STATE $originalSize:Integer \leftarrow produced.size$
\WHILE {$selected.size < numRequested\ AND\ selected.size < originalSize$}
\STATE $selectedTemp \leftarrow selected$
\STATE $(candidateOne, candidateTwo) \leftarrow (i, j)\ where$ \\ 
$\displaystyle\max_{i \neq j} (FO(selectedTemp.push(produced_{i} \cup produced_{j})))$
\STATE $produced.erase(i)$
\STATE $produced.erase(j)$
\STATE $selected.push(candidateOne)$
\STATE $selected.push(candidateTwo)$
\ENDWHILE
\RETURN $selected$
\end{algorithmic}
\caption{Selección de bundles de a pares}\label{alg:algSelTuple}
\end{algorithm}
\subsection{Selección proporcional}
Además como tercera opción de selección se implementó un algoritmo proporcional que en cada paso se 
ponderan los resultados de la función que calcula el intra y el inter restante, intentando 
\textquotedblleft adivinar\textquotedblright  el valor de las próximas iteraciones y de esta manera 
dar más importancia en los primeras iteraciones al valor del inter.
\begin{algorithm}[H]
\begin{algorithmic}[1]
\REQUIRE $produced:Vector<SnowFlake>, numRequested:Integer$
\ENSURE $selected:Vector<SnowFlake>$
\STATE $w(u) = \dfrac{k}{u.size} * (\gamma \sum_{v \in U}(w(v))) + \dfrac{(k * (k-1))}{2} * \dfrac{2}{(u.size() * (u.size() - 1))} * 1 - \gamma \sum_{v,w \in U}(\psi(v,w))$
\STATE $available \leftarrow produced$
\STATE $selected \leftarrow []$
\WHILE {$selected.size < numRequested\ AND\ selected.size < produced.size$}
\STATE $candidate \leftarrow max_{i}$
\ENDWHILE
\RETURN $selected$
\end{algorithmic}
\caption{Selección de bundles proporcional}\label{alg:algSelProp}
\end{algorithm}

\section{Modificación de PAC para búsquedas específicas}
Para la obtención de la solución se modificó la producción de bundles como así también la 
selección de los mismos (Produce and Choose). \\
En la producción de bundles en el algoritmo jerárquico, se utilizó la similitud del perfil 
específico con los papers en cada paso que intenta unificar dos clusters. A diferencia del cálculo 
original que la compatibilidad de dos nodos esta dada por su distancia previamente obtenida, en 
este nuevo caso se agrega a ese resultado la compatibilidad de cada uno de ellos con el perfil 
específico. \\
Para la producción del algoritmo BOBO, se agregó junto al pivote en todos los clusters. \\
Para la selección de los bundles que formarán parte de la solución, a cada cluster se le 
calculo el valor intra también se tuvo en cuenta la similitud de todos los elementos con el perfil  
específico, de esta manera a los clusters que contenían papers con los mismos tópicos que el del 
vector especifico se le dio mayor peso.

\section{Algoritmo goloso}
En los algoritmos previos se centraron en construir bundles máximizando el intra y una vez generado 
una cantidad suficiente seleccionar un conjunto de estos para la solución.
En el algoritmo goloso propuesto, se generan unicamente los bundles que pertenecen a la solución.
Agregando iterativamente el item al bundle que máximiza la función objetivo.\\
El algoritmo comienza con los bundles de la solución vacíos. En cada paso selecciona, de los items que no son parte
de la solución, aquel que máximiza la función objetivo agregandolo a alguno de los bundles sin violar las restricciones
del problema.
\begin{algorithm}[H]
\begin{algorithmic}[1]
\REQUIRE $numOfSnowFlakes:Integer$
\ENSURE $selected:Vector<SnowFlake>$
\STATE $selected_{i}:Vector<SnowFlake> \leftarrow \emptyset_{0\leq i<numOfSnowFlakes}$
\STATE $isComplete:Bool \leftarrow False$
\STATE $elements:Set<Element> \leftarrow ElementsOfTheProblem$
\WHILE {$isComplete == False$}
  \STATE $bestScore:Double \leftarrow -\infty$
  \STATE $bestElement:Element \leftarrow \varnothing$
  \STATE $bestBundle:SnowFlake \leftarrow \varnothing$
  \FOR {$elem:Element \in elements$}
    \FOR {$bundle:SnowFlake \in selected$}
      \IF {$isValidBundle(bundle \cup \{elem\}) == True$}
        \STATE $score:Double \leftarrow FO(selected.replace(bundle, bundle \cup \{elem\}))$
        \IF {$score > bestScore$}
          \STATE $bestScore \leftarrow score$
          \STATE $bestBundle \leftarrow bundle$
          \STATE $bestElement \leftarrow elem$
        \ENDIF
      \ENDIF
    \ENDFOR
  \ENDFOR
  \STATE $selected \leftarrow selected.replace(bundle, bundle \cup \{elem\})$
  \STATE $elements.erase(elem)$
  \STATE $isComplete \leftarrow bestElement == \varnothing$
\ENDWHILE
\RETURN $selected$
\end{algorithmic}
\caption{Algoritmo heurística golosa}\label{alg:algHeuGol}
\end{algorithm}

\section{Búsquedas Tabú}
Las búsquedas locales consisten en moverse de solución en solución, aplicando cambios a la solución candidata hasta encontrar una mejor solución o satisfacer un criterio de parada. Los algoritmos consisten en comenzar con una solución e iterativamente moverse a una solución vecina, esto es posible solo si se pude definir una relación de vecindad en el espacio de búsqueda. Como una solución puede tener muchas soluciones vecinas se elige siempre la que maximice o minimice (según el problema elegido) el criterio seleccionado, esto produce que el algoritmo pueda estancarse en un mínimo (ó máximo) local y nunca pueda salir de él.\\
\textbf{Tabú search} es una metaheurística, de la familia de las búsquedas locales, que relaja la primer regla de las búsquedas locales tradicionales y permite moverse a una solución vecina que no cumple con el criterio de búsqueda. De esta manera se permite al algorimo escapar de máximos o mínimos locales y encontrar una mejor solución (en caso que existiese). Otras de las modificaciones que se agregan es que una vez que una solución determinada es visitada, se la marca como tabú para que no vuelva a ser visitada por una determinada cantidad de iteraciones para también de esta manera evitar caer en ciclos y mínimos o máximos locales.\\
Una de las ventajas que tienen este tipo de metaheurísticas es que no son muy costosas en tiempo de ejecución siempre que la cantidad máxima de iteraciones no sea excesiva, con lo cuál se puede ejecutar sin problemas y sin importar de que algoritmo de generación y selección provenga la solución orginal con el fin de intentar mejorarla.\\
\subsection{Tabú mejorando la intra similitud}
Antes de comenzar con las búsqueda tabú, definimos como item centroide de un bundle, al item más cercano a todos los demás del mismo bundle, de esta manera también definimos al ``peor item'' (o item menos coehesivo) a aquel que se encuentra más alejado del centroide.\\
La implementación de la búsqueda tabú para mejorar la parte intra de la función objetivo tiene como intención que los bundles generados sean más cohesivos. Se entiende que un bundle es más cohesivos sí la relación de similitud entre los items es mayor, entonces se intenta mejorar aquellos bundles que contienen items que no son lo suficientemente cohesivos.\\
El movimiento de la solución s a la s' es en el bundle de la solución con menor valor intra. Entonces el paso de la búsqueda tabú es reemplazar el ``peor item'' con otro item del conjunto de los disponibles más cercano al centroide. El item reemplazado y el bundle se marcan como tabú, para que en el próximo paso ese bundle no sea el que se modifique y el item para que no sea el seleccionado para incluirse a la solución.
\begin{algorithm}[H]
\begin{algorithmic}[1]
\REQUIRE $numOfSnowFlakes:Integer$
\ENSURE $selected:Vector<SnowFlake>$
\STATE $selected_{i}:Vector<SnowFlake> \leftarrow \emptyset_{0\leq i<numOfSnowFlakes}$
\STATE $isComplete:Bool \leftarrow False$
\STATE $elements:Set<Element> \leftarrow ElementsOfTheProblem$
\WHILE {$isComplete == False$}
  \STATE $bestScore:Double \leftarrow -\infty$
  \STATE $bestElement:Element \leftarrow \varnothing$
  \STATE $bestBundle:SnowFlake \leftarrow \varnothing$
  \FOR {$elem:Element \in elements$}
    \FOR {$bundle:SnowFlake \in selected$}
      \IF {$isValidBundle(bundle \cup \{elem\}) == True$}
        \STATE $score:Double \leftarrow FO(selected.replace(bundle, bundle \cup \{elem\}))$
        \IF {$score > bestScore$}
          \STATE $bestScore \leftarrow score$
          \STATE $bestBundle \leftarrow bundle$
          \STATE $bestElement \leftarrow elem$
        \ENDIF
      \ENDIF
    \ENDFOR
  \ENDFOR
  \STATE $selected \leftarrow selected.replace(bundle, bundle \cup \{elem\})$
  \STATE $elements.erase(elem)$
  \STATE $isComplete \leftarrow bestElement == \varnothing$
\ENDWHILE
\RETURN $selected$
\end{algorithmic}
\caption{Algoritmo búsqueda tabú sobre elementos}\label{alg:algBusTabuIntra}
\end{algorithm}
\subsection{Tabú realizando cambios de bundles completos}
Lo primero que hicimos fue definir un ``peor bundle'' y un bndle centroide diferente a él. Como ``peor bundle'' decidimos tomar a aquel que la suma de la compatibilidad entre los demás bundles sea menor. Al centroide como aquel que maximice la suma de la compatibilidad entre los demás bundles.\\
Recordemos que las búsquedas originales son siempre de la misma forma, producir muchos bundles y luego seleccionar los mejores de acuerda a la estretegía seleccionada. Luego del último paso nos quedan bundles que por algún motivo no fueron seleccionados, entonces la idea es tomar la solución obtenida e ir intercambiando sus bundles con los que en el paso de selección quedaron sin uso.\\
EL movimiento de la solución s a s' es intercambiar el ``peor bunde'' por aquel que maximice la suma de intercompatibilidades entre los bundles. En caso de no exisitir un bundle que logre esto, decidimos que si un elemento que ya había sido tenido en cuenta para intentar mejorar la función y es el que ``menos la empeora'', entonces realizamos el cambio igualmente para intentar salir del mínimo o máximo que nos encontramos. El bundle que se elimina queda marcado como tabú en cualquiera de las dos circunstancias.
\begin{algorithm}[H]
\begin{algorithmic}[1]
\REQUIRE $numOfSnowFlakes:Integer$
\ENSURE $selected:Vector<SnowFlake>$
\STATE $selected_{i}:Vector<SnowFlake> \leftarrow \emptyset_{0\leq i<numOfSnowFlakes}$
\STATE $isComplete:Bool \leftarrow False$
\STATE $elements:Set<Element> \leftarrow ElementsOfTheProblem$
\WHILE {$isComplete == False$}
  \STATE $bestScore:Double \leftarrow -\infty$
  \STATE $bestElement:Element \leftarrow \varnothing$
  \STATE $bestBundle:SnowFlake \leftarrow \varnothing$
  \FOR {$elem:Element \in elements$}
    \FOR {$bundle:SnowFlake \in selected$}
      \IF {$isValidBundle(bundle \cup \{elem\}) == True$}
        \STATE $score:Double \leftarrow FO(selected.replace(bundle, bundle \cup \{elem\}))$
        \IF {$score > bestScore$}
          \STATE $bestScore \leftarrow score$
          \STATE $bestBundle \leftarrow bundle$
          \STATE $bestElement \leftarrow elem$
        \ENDIF
      \ENDIF
    \ENDFOR
  \ENDFOR
  \STATE $selected \leftarrow selected.replace(bundle, bundle \cup \{elem\})$
  \STATE $elements.erase(elem)$
  \STATE $isComplete \leftarrow bestElement == \varnothing$
\ENDWHILE
\RETURN $selected$
\end{algorithmic}
\caption{Algoritmo búsqueda tabú sobre bundles}\label{alg:algBusTabuBundle}
\end{algorithm}
\chapter{Experimentación computacional}
\label{chap:experimentacion}
\section{Instancias de pruebas}\label{sect:busquedas}
En esta sección se presentan las consultas que se utilizaron para evaluar las propuestas algorítmicas presentadas en este trabajo. Las consultas se hicieron sobre dos bases de datos. Una de ellas corresponde a la base de datos de artículos provista por \textit{\textquotedblleft A Data-Driven Journey through Software Engineering Research\textquotedblright}\cite{dataDrive}. La otra base de datos corresponde a atracciones turísticas de Europa. A continuación se describe para cada consulta de cada base da datos la función de similitud, el atributo de complementariedad y el presupuesto. 

\subsection{Base de datos de artículos}
La base de datos utilizada es la proporcionada por \textit{\textquotedblleft A Data-Driven Journey through Software Engineering Research\textquotedblright}\cite{dataDrive}. La misma contiene unos $7800$ artículos relacionados con la ingeniería de software presentados en diferentes conferencias entre los años 1975 y 2011 catalogados por autores, tópicos relevantes, conferencia donde fue presentado el trabajo y afiliaciones. Además cada artículo tiene asociado un \texttt{topicProfile} de los autores, que expresa el porcentaje de la relevancia de cada tópico considerado dentro del artículo. Este valor fue calculado en función de los temas de las conferencias o revistas de los trabajos citados en el artículo. La base considera 37 tópicos para definir el \texttt{topicProfile}. De los $9800$ autores se tiene la información de la universidad a la que cada uno pertenece y la región donde se encuentra dicha universidad.

El \texttt{topicProfile} es lo que permitirá definir la similitud, no sólo entre los artículos, sino también entre los autores y las universidades de la base de datos de una manera prácticamente directa.

Los criterios de las búsquedas realizadas sobre la base de datos se concibieron a partir de lo que se considera que es de interés general. Por ejemplo, una posible consulta sobre esta base de datos podría ser la búsqueda de artículos sobre núcleos temáticos característicos en las distintas conferencias, de manera que observando un paquete se podría conocer qué se dice sobre este tema en cada una de las conferencias. En otro escenario, con el propósito de armar paneles de expertos, puede resultar de interés la búsqueda de investigadores que trabajan en tópicos similares con afiliación en distintas universidades. También se puede querer conocer universidades de distintas regiones con grupos de investigación trabajando en tópicos similares.

Por lo establecido en \cite{compositeRetrival} para las búsquedas se deben realizar las siguientes definiciones:
\begin{itemize}
  \item \textbf{Similitud}: Función que dado dos ítems devuelve la similitud entre estos.
  \item \textbf{Costo}: Función que dado un ítem devuelve el costo del mismo.
  \item \textbf{Presupuesto}: El presupuesto que se tiene, el cual no podrá ser excedido por ningún paquete.
  \item \textbf{Complementariedad}: Propiedad del ítem que es único en cada paquete.
\end{itemize}

Para todas las búsquedas, sin importar el ítem que sea (artículo, autor o universidad), se definió que el costo de cada ítem sea de una unidad y que el presupuesto para cada búsqueda sea de cinco unidades. En consecuencia, todos los paquetes de todos los resultados contienen como máximo cinco ítems. Se tomó esta decisión para que cada paquete contenga como máximo un número fijo de elementos. Además se estableció que sean diez los paquetes devueltos en cada búsqueda. El motivo para tomar esta decisión es que un humano pueda valorizar el resultado propuesto fácilmente. Entonces, de aquí en adelante, para cada criterio de búsqueda se deben definir únicamente la función de similitud y la propiedad de complementariedad.

Como se mencionó anteriormente, en la base de datos cada artículo cuenta con su \texttt{Topic Profile}. El \texttt{Topic Profile} define el perfil de un artículo asignándole un porcentaje a cada tópico que se hace referencia. En el caso del artículo \texttt{A Cognitive-Based Mechanism for Constructing Software Inspection Teams} el \texttt{Topic Profile} se compone por los tópicos  REQUIREMENTS, RELIABILITY, TESTING y SOFTWARE QUALITY. El porcentaje de cada uno de estos es 71.43 \%, 17.86 \%, 7.14 \% y 3.57 \% respectivamente. Esto significa que el 71.43\% de los trabajos citados en este artículo fueron presentados en conferencias o publicados en revistas vinculadas al área REQUIREMENTS.

El modelo computacional del perfil de cada artículo es un vector que la dimensión corresponde a la cantidad de tópicos y cada posición representa un tópico diferente. El valor de cada posición del vector es el porcentaje del tópico que le corresponde a ese artículo según el \textit{Topic Profile} de la base de datos. Más adelante se explica como estos vectores se utilizan para comparar la similitud entre los artículos.

Para los autores no se cuenta con información más allá de los artículos que escribieron, pero sólo con eso alcanza para poder generar un perfil de autores. Para cada autor se hace la suma vectorial de cada uno de los \texttt{Topic Profile} de los artículos en los cuales participó y con eso se obtiene el \texttt{Topic Profile de Autores}. Para obtener el perfil de las universidades se aplicó el mismo criterio. Se realiza la suma vectorial de cada uno de los \texttt{Topic Profile de Autores} pertenecientes a la misma universidad y así se genera el \texttt{Topic Profile de Universidades}. En ambos casos se aplica la normalización sobre los vectores resultantes.

Para clarificar mostramos un ejemplo de los perfiles de los elementos:

\begin{table}[H]
\begin{tabular}{lll}
	Artículo & Topic Profile & Autores \\
	Artículo 1 & $[$0.20, 0.40, 0.40, 0.00$]$ & Autor 1, Autor 2, Autor 3 \\
	Artículo 2 & $[$0.30, 0.70, 0.00, 0.00$]$ & Autor 2, Autor 3 \\
	Artículo 3 & $[$0.00, 0.10, 0.00, 0.90$]$ & Autor 2 \\
	Artículo 4 & $[$0.00, 0.00, 1.00, 0.00$]$ & Autor 1, Autor 3 \\
\end{tabular}
\label{tabla:topicProfileEj1}
\end{table}

\begin{table}[H]
\begin{tabular}{lll}
	Autor & Topic Profile & Universidad \\
	Autor 1 & $[$0.14, 0.27, 0.95, 0.00$]$ & Universidad 1 \\
	Autor 2 & $[$0.30, 0.74, 0.25, 0.55$]$ & Universidad 2 \\
	Autor 3 & $[$0.27, 0.60, 0.76, 0.0$]$ & Universidad 2 \\
\end{tabular}
\label{tabla:topicProfileEj2}
\end{table}

\begin{table}[H]
\begin{tabular}{ll}
	Universidad & Topic Profile \\
	Universidad 1 & $[$0.14, 0.27, 0.95, 0.00$]$ \\
	Universidad 2 & $[$0.31, 0.72, 0.54, 0.30$]$ \\
\end{tabular}
\label{tabla:topicProfileEj3}
\end{table}

Para la evaluación las consultas realizadas son:
\begin{enumerate}
	\item
		Artículos con tópicos similares presentados en distintas conferencias. \label{busqueda:articulos}
		\begin{itemize}
			\item \textbf{Similitud}: Función que compara el perfil de cada artículo.
			\item \textbf{Complementariedad}: Lugar dónde fue presentado.
		\end{itemize}

	\item
	Autores que escribieron artículos con tópicos similares afiliados a universidades distintas. \label{busqueda:autores}
	\begin{itemize}
		\item \textbf{Similitud}: Función que compara el perfil de los autores.
		\item \textbf{Complementariedad}: Universidad de pertenencia del autor.
	\end{itemize}

	\item 
	Universidades en donde se escribieron artículos de tópicos similares que se encuentran en distintas regiones. \label{busqueda:universidades}
	\begin{itemize}
		\item \textbf{Similitud}: Función que compara el perfil de las universidades.
		\item \textbf{Complementariedad}: Región de la institución.
	\end{itemize}
\end {enumerate}

Para obtener resultados de mayor calidad, se depuró de la base de datos aquellos artículos que no contengan la información del autor, de los tópicos (\textbf{topic profile}) o del lugar de publicación (\textbf{venue}). Quedando, luego de la depuración $5500$ artículos.  

\paragraph{Función de similitud}
La similitud se emplea para comparar dos objetos y determinar qué tan parecido son entre si. En este trabajo definimos la similitud entre los objetos de la base de datos de artículos mediante la \textbf{similitud coseno}. Esta es una medida de similitud entre dos vectores en un espacio vectorial provisto de un producto escalar que mide el coseno del ángulo comprendido entre ellos.

Entonces se define la función de similitud $S(U_i, V_j)$ para los vectores $U_i$ y $V_j$ a partir del producto escalar\\

\begin{equation} \label{eq:angulovectorial}
\cos(\theta) =  \dfrac{\overrightarrow{U_i} . \overrightarrow{V_j}}{\overrightarrow{\lVert U_i\lVert}.\overrightarrow{\lVert V_j\lVert}}
\end{equation}

Para esta instancia los objetos (ahora artículos, autores o universidades) están representados por vectores, donde cada dimensión corresponde a un tópico cuyo valor se corresponde con el valor del tópico del objeto según la base de datos \cite{dataDrive}. Por lo tanto el objeto $a$ se representa con el vector $V_a = [v_1,v_2,...,v_3]$ que cumple con las siguientes propiedades:
\begin{enumerate}
 \item $v_i \geq 0$
 \item $\sum{v_i} = 1$
\end{enumerate}

Como los componentes de todos los vectores son mayor o igual a cero se obtiene que $0\leq\cos(\theta)\leq1$, que implica que $S(V_i, V_j) \in \left[0, 1\right]$.

\begin{figure}[H]
\includegraphics[width=0.8\textwidth]{img/coseno.png}
\caption{Comportamiento de la función $\cos$. En rojo la región que pertenece a la función de similitud}
\label{bus:img-coseno}
\end{figure}

%Para \textbf{similitud coseno} dos vectores proporcionales con la misma dirección la similitud es 1 (ya que es 0 el ángulo que se forma). Por lo que esta similitud no diferencia entre un artículo profesional y un artículo de un diario que cubre el mismo tópico. Por ejemplo si dos artículos que cubren un mismo y único tópico, pero para uno el valor del tópic Esta debilidad de la medida basada en el ángulo no interfiere en este trabajo por la segunda propiedad de los vectores del problema, porque para que dos vectores sean proporcionalmente iguales tienen que ser idénticos y en tal caso es correcto que la similitud entre ellos sea 1.

Con el objetivo de simplificar la ejecución de los algoritmos, considerando que el costo de calcular $cos()$ de los vectores es alto, se decidió realizar el cáculo de la similitud de los artículos, autores y universidades previamente a la ejecución de los algoritmos de búsquedas

\subsection{Base de dato de atracciones turísticas}
Se utilizó una instancia de datos correspondiente a 200 atracciones turísticas de Europa, con datos relevados del sitio \textit{TripAdvisor}. De cada atracción se tiene la información del precio, del tipo (parque, museo, edificio) y de la distancia geográfica con el resto de las atracciones.

El propósito de la búsqueda es darle al usuario distintas opciones de circuitos turísticos que contienen atracciones, con las siguientes requerimientos: evitar realizar largos traslados; que haya variedad en el tipo de atracción y que que el costo del circuito no supere el presupuesto del turista. Por lo tanto el modelo de la búsqueda quedo diseñada de la siguiente manera: \label{busqueda:atracciones}

\begin{itemize}
	\item \textbf{Similitud}: La inversa de la distancia entre las atracciones. 
	\item \textbf{Costo}: Precio de la atracción. 
	\item \textbf{Presupuesto}: Presupuesto del turista. 
	\item \textbf{Complementariedad}: Tipo de atracción.
\end{itemize}

\section{Análisis de resutados}\label{sect:resultados}

En esta sección se analizan los resultados computacionales comparando la calidad de las soluciones obtenidos por los algoritmos discutidos en \autoref{chap:nuevas-propuestas}, sobre las bases de datos de \autoref{sect:busquedas}. Con el objetivo de evaluar las propuestas algorítmicas, se consideraron los siguientes métodos.

Para la experimentación se utilizó una máquina Desktop Intel(R) Core(TM) i5-4570T CPU @ 2.90GHz, 5.7G Ram, con DB: 5.5.46-MariaDB-1ubuntu0.14.04.2.

\begin{itemize}
\item{$alg1$} PAC(C-HAC / selección simple)
\item{$alg2$} PAC(BOBO-10 / selección simple)
\item{$alg3$} PAC(BOBO-10 / selección proporcional)
\item{$alg4$} PAC(BOBO-10 / selección proporcional) + tabú
\item{$alg5$} PAC(Intra-Inter C-HAC / selección proporcional)
\item{$alg6$} PAC(Intra-Inter C-HAC / selección proporcional) + tabú
\item{$alg7$} Construcción golosa
\item{$alg8$} Construcción golosa + tabú
\end{itemize}

Tanto $alg1$ como $alg2$ corresponden a las metodologías propuestas en \cite{compositeRetrival}. Los demás algoritmos involucran las mejoras propuestas en este trabajo. En \texttt{PAC}, la búsqueda tabú \texttt{Inter-Paquete} se realiza al finalizar la etapa de producción y la \texttt{Intra-Paquete} luego de la selección. En la \texttt{Búsqueda Golosa} se intenta mejorar la solución obtenida mediante la búsqueda tabú \texttt{Intra-Paquete}. Cabe señalar que no se tienen en consideración BOBO-Ex y CAP ya que para el tamaño de la instancia los tiempos de ejecución de esos algoritmos resultaron prohibitivos. A partir de experimentación preliminar con BOBO para valores $c=1, 5$ y $10$, resultó $c=10$ la opción más competitiva. Para la búsquedas tabú se definió que la cantidad de iteraciones de permanencia de un elemento en la lista tabú sea el promedio de elementos que tiene un paquete en la solución inicial.

Para realizar una comparación entre la calidad de las soluciones obtenidas por los diferentes algoritmos, se ha evaluado para los $\gamma \in \left\{0,1; 0,3; 0,4; 0,5; 0,6; 0,7; 0,8; 0,9\right\}$ el porcentaje de deterioro de cada solución respecto de la mejor solución obtenida por alguno de los ocho algoritmos.

En el caso de la búsqueda de artículos, que es el escenario que contiene la mayor cantidad de objetos, los tiempos de ejecución  para los algoritmos C-HAC ($alg1$, $algo5$ y $alg6$) son del orden de los 5 minutos, mientras que para los algoritmos BOBO ($alg2$, $alg3$ y $alg4$) de 2 minutos y los golosos ($alg7$ y $alg8$) de los 6 minutos. Los incrementos de tiempo debido a la ejecución de las metaheruísticas de mejora son despreciables, están entre los 5 y 7 segundos. Por lo cual no se considera que el tiempo sea un factor que valga la pena analizar.

\subsection{Base de datos de artículos}
Para comprender el comportamiento de los resultados de las búsquedas se diseñaron dos tipos de gráficos que permiten visualizar la cohesión de los paquetes y la dispersión entre ellos. De esta forma se podrá analizar la calidad del resultado obtenido más allá del valor de la función objetivo.

Los gráficos del estilo de la figura \ref{res:img-explain-bars} permiten analizar la distribución de los tópicos de una solución a nivel de paquete y de la relación con otros. Las filas corresponden a los 10 paquetes obtenidos y las columnas a los 38 tópicos considerados. El tamaño del círculo representa la proporción del tópico en el perfil del artículo y el color hace referencia al paquete al cual artículo pertenece. Por lo tanto, dos artículos tendrán gran similitud cuando los patrones de sus círculos coincidan, tanto en tamaño como en distribución. Si para un paquete la distribución entre los tópicos y el tamaño de los círculos es similar entre sus artículos se puede deducir que este paquete es cohesivo (tiene buen valor intra). Por otro lado, si los patrones de los círculos de los dos artículos más similares entre distintos paquetes no coinciden, esto indica que el resultado es diverso.
\begin{figure}[H]
  \centering
    \includegraphics[width=1\textwidth]{img/explain-bars.png}
  \caption{}
  \label{res:img-explain-bars}
\end{figure}

Los gráficos de tipo burbuja de la figura \ref{res:img-explain-bubbles} son útiles para concluir el nivel de acoplamiento entre los paquetes de una solución, observando la relación entre los tópicos y los paquetes. Cada burbuja representa un tópico; cada circulo dentro de esa burbuja es un artículo donde el tamaño es la proporción del articulo con el tópico y el color el paquete al que pertenece. Entonces si las burbujas contienen círculos de tamaños parecidos de más de un color se puede decir que ese resultado no es muy diverso, mientras que el color de los círculos de las burbujas sea más homogéneo el resultado será más diverso. En cuanto a la cohesión de los paquetes, es más cohesivo cuando el tamaño de cada circulo dentro de las burbujas es similar (para el mismo color) y cada una de ellas contiene la misma cantidad, o ninguno, de círculos del mismo color.

\begin{figure}[H]
  \centering
    \includegraphics[width=0.5\textwidth]{img/explain-bubbles.png}
  \caption{}
  \label{res:img-explain-bubbles}
\end{figure}

\subsubsection{Búsqueda de artículos}
Para la búsqueda de artículos con tópicos similares en la tabla \ref{tabla:comp1} se observa los porcentajes de deterioro de cada solución respecto de la mejor solución obtenida por alguno de los ocho algoritmos. Una primera observación es que los algoritmos  Intra-Inter C-HAC reflejan el efecto buscado: a menores valores de $\gamma$ donde el valor inter tiene mayor peso, se obtienen mejores soluciones. Es decir, haber considerado en el proceso de generación de paquetes la funcion Intra-Inter benefició a la calidad de las soluciones obtenidas. Los algoritmos BOBO, no obtienen soluciones de la calidad de los algoritmos C-HAC y el proceso de selección proporcional no logra una mejora consistente para todos los valores de $\gamma$. Las soluciones obtenidas con el algoritmo de construcción golosa, no alcanzaron a mejorar las soluciones de C-HAC pero fueron ampliamente mejores que las de BOBO. En promedio el porcentaje de deterioro de BOBO fue de un $50\%$ mientras que el goloso fue de un $12\%$. 

Cabe resaltar el muy buen rendimiento de la búsqueda tabú, tanto en escenarios donde la solución inicial no es de buena calidad (algoritmo BOBO) así como también considerando soluciones de mejor calidad (algoritmo Intra-Inter C-HAC). En el primer caso, se obtienen porcentajes de mejora por encima del $70\%$. En el segundo caso, para varios valores de $\gamma$ la solución obtenida por la búsqueda tabú resultó ser la mejor opción y en otros con deterioros inferiores al $0.5\%$.

Si bien el algoritmo goloso no alcanzó los valores obtenidos por las soluciones generadas por C-HAC, tiene como ventaja su fácil y rápida implementación. Sus tiempos de ejecución fueron levemente mayores a las que se obtuvieron con BOBO y menores a las ejecutadas por C-HAC. Por la forma en la que fue construido siempre genera paquetes completos. Si bien la definición formal del problema no obliga a ésto último, se implementó de esta manera ya que a efectos de un usuario final es más interesante obtener paquetes completos, aunque esto signifique sacrificar la búsqueda de la solución óptima.

\begin{table}[H]
\begin{center}
\begin{tabular}{|c|c|c|c|c|c|c|c|c|}
\hline
$\gamma$&$alg1$&$alg2$&$alg3$&$alg4$&$alg5$&$alg6$&$alg7$&$alg8$ \\ \hline
0.1 & -2.05 & -32.70 & -36.18 & -7.45 & -0.42 & 0.00 & -4.53 & -3.53 \\
0.2 & -2.11 & -38.06 & -41.19 & -8.23 & 0.00 & 0.00 & -4.92 & -3.85 \\
0.3 & -2.31 & -45.21 & -47.35 & -8.01 & 0.00 & 0.00 & -9.17 & -7.66 \\
0.4 & -0.14 & -49.08 & -51.22 & -15.51 & 0.00 & 0.00 & -10.40 & -9.35 \\
0.5 & 0.00 & -52.35 & -54.23 & -17.87 & -0.31 & -0.31 & -12.97 & -10.38 \\
0.6 & 0.00 & -55.16 & -56.04 & -14.43 & -0.05 & -0.05 & -14.78 & -13.69 \\
0.7 & 0.00 & -56.88 & -56.57 & -17.02 & -0.41 & -0.41 & -16.21 & -15.08 \\
0.8 & 0.00 & -57.86 & -57.86 & -16.11 & -0.56 & -0.30 & -18.10 & -17.60 \\
0.9 & 0.00 & -58.92 & -58.92 & -15.91 & -0.48 & -0.35 & -20.47 & -17.61 \\ \hline 
\end{tabular}
\caption{Comparación de calidad de soluciones entre algoritmos para la \hyperref[busqueda:articulos]{búsqueda de artículos}} 
\label{tabla:comp1}
\end{center}
\end{table}

Para comprender la semantica de las soluciones comparamos las soluciones con $\gamma = 0,1$ y $\gamma = 0,9$. En la figura \ref{res:comp1} se observa que para $\gamma = 0,1$ los tópicos (representados por burbujas) para la solución de \textit{alg 3} están presentes en varios paquetes (representados por círculos de colores). En cambio en el $alg6$ la mayoría de las burbujas contiene círculos de un solo color, De esta manera se visualiza que la solución de $alg6$ es más diversa. En $\gamma = 0,9$ el resultado obtenido con $alg3$ no se visualiza que cada burbuja contenga cinco circulos del mismo color, en cambio para la solución de $alg1$ si. Eso significa que los paquetes de la solución de $alg1$ son más cohesivos, ya que los cinco elementos de cada paquete tienen los mismos tópicos.

\begin{figure}[H]
	\centering
	\begin{tabular}{cc}
		$alg3$ & $alg6$\\
		\multicolumn{2}{c}{$\gamma=0.1$}\\ 
			\includegraphics[width=0.5\linewidth]{img/gamma-01-burbujas-alg-3.png}&
			\includegraphics[width=0.5\linewidth]{img/gamma-01-burbujas-alg-6.png} 		\vspace{1cm}\\
			$alg3$ & $alg1$\\
		\multicolumn{2}{c}{$\gamma=0.9$}\\
			\includegraphics[width=0.5\linewidth]{img/gamma-09-burbujas-alg-3.png}&
			\includegraphics[width=0.5\linewidth]{img/gamma-09-burbujas-alg-1.png}\\
	\end{tabular}
	\caption{Comparación entre las soluciones con menor(izq) y mayor(der) función objetivos  para $\gamma\ =\ 0,1\ y \gamma\ =\ 0,9$}
	\label{res:comp1}
\end{figure}

La búsqueda tabú tiene su mayor impacto cuando es aplicada a la solución brindada por el algoritmo $alg3$, como puede observarse en la tabla \autoref{tabla:comp1} y en este contexto resulta una buena alternativa por su bajo costo computacional. En la Figura~\ref{res:bobo} se observa que para $\gamma=0.1$, la solución dada por el $alg3$ tiene artículos de distintos paquetes con patrones muy similares indicando un bajo valor inter-paquete. Por el contrario, luego de aplicar la búsqueda tabú los patrones de los artículos más similares entre distintos paquetes se volvieron más dispares, demostrando el aumento de la diversidad entre paquetes. Para $\gamma=0.9$, como puede observarse en la misma Figura~\ref{res:bobo}, en la solución brindada por el $alg4$ todos los paquetes tienen al menos un artículo cuyo patrón consiste en muchos círculos pequeños distribuidos en la mayoría de los tópicos en contraposición al resto de los artículos del mismo paquete con pocos círculos de gran tamaño, demostrando un bajo valor intra-paquete. En contraposición a los paquetes obtenidos luego de aplicar la búsqueda tabú, los cuáles son mucho más cohesivos. La última afirmación puede observarse en los círculos de los artículos dentro de un mismo paquete, quienes siguen patrones mucho más parecidos. Se puede afirmar que la búsqueda tabú es capaz de mejorar las características de la solución en función del parámetro $\gamma$.

\begin{figure}[H]
	\centering
	\begin{tabular}{cc}
		$alg3$ & $alg4$\\
		\multicolumn{2}{c}{$\gamma=0.1$}\\
			\includegraphics[width=0.5\linewidth]{img/gamma-01-alg-3.png}&
			\includegraphics[width=0.5\linewidth]{img/gamma-01-alg-4.png} 		\vspace{1cm}\\
		\multicolumn{2}{c}{$\gamma=0.9$}\\
			\includegraphics[width=0.5\linewidth]{img/gamma-09-alg-3.png}&
			\includegraphics[width=0.5\linewidth]{img/gamma-09-alg-4.png}\\
	\end{tabular}
	\caption{Comparación de soluciones para BOBO-10 con y sin heurística de mejoramiento}
	\label{res:bobo}
\end{figure}

La decisión acerca del valor de $\gamma$ para priorizar un objetivo sobre el otro es realizada por el usuario. Una curva de frontera eficiente podría ayudar para examinar el balance (trade-off) entre los dos objetivos. La intención es que el usuario pueda analizar si una mejora significativa en el valor intra-paquete implica una degradación sustancial en el valor inter-paquetes y viceversa. Para ilustrar este análisis en la Figura \ref{res:inter_intra} se comparan las soluciones obtenidas por los algoritmos $alg1$ y $alg5$ variando en pasos de $0,1$. Una primera evaluación muestra que la mayoría de las soluciones provistas por ambos algoritmos son no dominadas, es decir ninguna solución es mejor en ambos objetivos que cualquier otra solución.

Como se mencionó anteriormente, el algoritmo $C-HAC$ de \cite{compositeRetrival} ($alg1$) utiliza la función $Score$ para decidir el par de {\em clusters} a unir en la etapa de producción de paquetes y en la selección simple en la segunda fase. Estos dos criterios omiten la diversidad de los paquetes. De acuerdo a los resultados de la tabla \ref{tabla:comp1}, se pudo concluir que haber considerado la función Intra-Inter y la selección proporcional benefició la calidad de las soluciones cuando la misma es medida a partir de la función objetivo $w(S)$. Con el fin de evaluar que el $alg5$ es capaz de captar efectivamente la diversidad en la solución, se analiza las soluciones para $\gamma=0.5$. En este caso el $alg1$ obtuvo un valor de $intra=93,82$ y de $inter=35,49$, mientras que el $alg5$ logró valores de $intra=92,94$ y de $inter=35,96$. A pesar de que la solución de $alg1$ es levemente superior, la solución del $alg5$ aumentó el valor $inter$ el $1,32\%$ con un deterioro del valor $intra$ del $0,93\%$. Observando la Figura~\ref{res:inter_intra} se puede ver que la cohesión intra-paquete de ambas soluciones es equivalente. Sin embargo, la diversidad en la segunda solución es mayor, ya que los patrones entre los artículos más similares de distintos paquetes son más heterogéneos. 

En la misma figura se compara las soluciones obtenidas por los algoritmos $alg2$ y el goloso $alg7$. En promedio supera a las soluciones provistas por $alg2$ en $76\%$. De la comparación, se destaca que el valor de $\gamma$ impacta más en las soluciones de $alg2$.  

\begin{figure}[H]
	\centering
	\begin{tabular}{cc}
			\includegraphics[width=0.5\linewidth]{img/alg1_vs_alg5.png}&
			\includegraphics[width=0.5\linewidth]{img/alg2_vs_alg7.png}\\
			$alg1$ vs $alg5$ & $alg2$ vs $alg7$\\
	\end{tabular}
	\caption{Trade-off entre objetivos para los distinos valores de $\gamma$}
	\label{res:inter_intra}
\end{figure}

\subsubsection{Búsqueda de autores}
En la tabla \ref{tabla:comp2} se muestran los porcentajes de deterioro de cada solución con respecto a la mejor obtenida para la búsqueda de autores que escribieron artículos de tópicos similares que están afiliados a distintas universidades. Se observa que en este caso, a diferencia de la búsqueda de artículos, el algoritmo tabú halló en todos los casos la mejor solución. Con la particularidad que el algoritmo goloso tuvo muy poco deterioro con respecto a la mejor solución, alcanzando en algunos casos la misma solución.

Los algoritmos jerárquicos $C-HAC$ e $Inter-Intra C-HAC$ siguen demostrando que obtienen las mejores soluciones en comparación a $BOBO$, sobre todo $alg6$ que contiene la búsqueda tabú. Éstas búsquedas vuelven a indicar que las soluciones pueden ser mejoradas en porcentajes muy significativos como ocurrió en el caso del algoritmo $alg3$, mejorando las soluciones en un promedio de al rededor del $30\%$ y tambien para $alg5$ y $alg7$ que en todos los casos mejoraron sus valores iniciales. Con respecto a la estrategía de selección propocional aplicada al algoritmo $BOBO$ no se observaron mejoras, aunque las soluciones obtenidas fueron cercanas en términos de función objetivo. 

\begin{table}[H]
\begin{center}
\begin{tabular}{|c|c|c|c|c|c|c|c|c|}
\hline
$\gamma$&$alg1$&$alg2$&$alg3$&$alg4$&$alg5$&$alg6$&$alg7$&$alg8$ \\ \hline
0.1 & -0.33 & -21.59 & -26.05 & -1.74 & -0.13 & 0.00 & -0.61 & 0.00 \\
0.2 & -0.63 & -27.46 & -29.71 & -0.52 & -0.36 & 0.00 & -1.10 & -0.25 \\
0.3 & -0.44 & -30.57 & -32.47 & -0.20 & -0.53 & 0.00 & -1.50 & -0.34 \\
0.4 & -0.25 & -32.63 & -34.29 & -0.32 & -0.25 & 0.00 & -1.88 & -0.73 \\
0.5 & -0.22 & -34.42 & -35.84 & -0.04 & 0.00 & 0.00 & -2.15 & -0.79 \\
0.6 & -0.18 & -35.86 & -37.05 & -2.01 & 0.00 & 0.00 & -2.45 & -1.39 \\
0.7 & 0.00 & -37.10 & -37.93 & -1.71 & -0.12 & 0.00 & -2.59 & -0.96 \\ 
0.8 & 0.00 & -38.19 & -38.70 & -1.44 & -0.08 & 0.00 & -2.72 & -1.20 \\
0.9 & -0.03 & -39.15 & -39.38 & -1.21 & -0.10 & 0.00 & -3.35 & -1.72 \\ \hline 
\end{tabular}
\caption{Comparación de calidad de soluciones entre algoritmos para la \hyperref[busqueda:autores]{búsqueda de autores}} 
\label{tabla:comp2}
\end{center}
\end{table}

Para analizar el trade-off entre la parte inter y la intra en la figura \ref{res:aut_alg1_vs_alg5_vs_alg7} del lado izquierdo se muestra los valores inter e intra de las soluciones obtenidas por los algoritmos $alg1$, $alg2$, $alg5$ y $alg7$ para todos los $\gamma$. Puede apreciarse, como es de esperar, que los valores del $alg2$ son significativamente inferiores al resto de los algorimots y en cambio, $alg1$, $alg5$ y $alg7$ sus valores se concentran en una región reducida ya que sus soluciones fueron muy similares respecto al valor de la función objetivo.

A la derecha de la figura \ref{res:aut_alg1_vs_alg5_vs_alg7} se encuetran los valores inter e intra de los algoritmos $alg1$ y $alg5$. Puede verse como los valores de las soluciones obtenidas con $alg5$ no están dominadas por la parte inter y si más concentradas en la parte intra, a diferencia de los que ocurre con $alg1$.

\begin{figure}[H]
	\centering
	\begin{tabular}{cc}
			\includegraphics[width=0.5\linewidth]{img/aut-alg1-alg2-alg5-alg7.png}&
			\includegraphics[width=0.5\linewidth]{img/aut-alg1-alg5.png}\\
	\end{tabular}
	\caption{}
	\label{res:aut_alg1_vs_alg5_vs_alg7}
\end{figure}


En la figura \ref{res:aut-alg-6} de la solución generada por el algoritmo $alg6$ para $\gamma = 0.1$, se tienen las siguientes observaciones:
\begin{enumerate}
	\item Todos los paquetes contienen autores que pertenecen a los mismos tópicos. 
	\item No existe un tópico que este presente en más de un paquete.
	\item Todos los paquetes utilizan el máximo del presupuesto.
\end{enumerate}
Por (1) y (2) la calidad intra e inter paquete es máxima respectivamente. Con (3) se cumple que la solución obtenida es la de máxima calidad para todo $\gamma$. Para todas las búsquedas realizadas, las soluciones que se obtuvieron cumplían con las observaciones mencionadas.

\begin{figure}[H]
  \centering
    \includegraphics[width=1\textwidth]{img/aut-alg-6.png}
  \caption{}
  \label{res:aut-alg-6}
\end{figure}



\subsubsection{Búsqueda de universidades}
En este escenario lo que más se destaca de la tabla \ref{tabla:comp3} es el comportamiento del algoritmo $alg1$ que generó una mejor solución para los valores de $\gamma\ =\ 0.1$ y $\gamma\ =\ 0.2$ con respecto al algoritmo $alg6$. En el resto de las soluciones obtenidas (de $alg1$ y $alg6$) se aprecia un deterioro cada vez más significativo a medida que crece el valor del $\gamma$.

Por otro lado la función objetivo de las soluciones provista por los algoritmos $alg2$, $alg3$ y $alg7$ se encuentran muy alejadas de las soluciones provistas por los jerárquicos. En el caso del algoritmo goloso $alg7$ las soluciones mejoran con respecto a los algoritmos $alg2$ y $alg3$ cuando el valor de $\gamma$ disminuye.

En este escenario, al igual que en el resto, la búsqueda tabú mejoró las soluciones iniciales del $BOBO$ y del algoritmo goloso considerablemente. Se destaca que para las soluciones jerárquicas la búsqueda tabú siempre mejora la solución inicial sin importar que tan buena.
\begin{table}[H]
\begin{center}
\begin{tabular}{|c|c|c|c|c|c|c|c|c|}
\hline
$\gamma$&$alg1$&$alg2$&$alg3$&$alg4$&$alg5$&$alg6$&$alg7$&$alg8$ \\ \hline
0.1 & 0.00 & -30.89 & -31.26 & -15.05 & -1.39 & -1.33 & -10.97 & -9.73 \\
0.2 & 0.00 & -40.35 & -40.16 & -22.09 & -1.72 & -1.07 & -22.83 & -20.42 \\
0.3 & -2.22 & -50.85 & -49.98 & -27.01 & -2.37 & 0.00 & -36.06 & -28.55 \\
0.4 & -5.72 & -60.65 & -58.96 & -33.76 & -1.36 & 0.00 & -48.44 & -30.38 \\
0.5 & -8.59 & -69.76 & -67.66 & -31.46 & -1.92 & 0.00 & -60.18 & -34.32 \\
0.6 & -12.09 & -77.50 & -74.71 & -33.85 & -2.53 & 0.00 & -70.16 & -35.80 \\
0.7 & -12.59 & -82.97 & -80.09 & -31.52 & 0.00 & 0.00 & -78.04 & -29.36 \\
0.8 & -15.52 & -88.12 & -84.93 & -32.90 & 0.00 & 0.00 & -85.63 & -36.34 \\
0.9 & -17.46 & -88.10 & -87.79 & -31.16 & 0.00 & 0.00 & -92.13 & -28.26 \\
 \hline 
\end{tabular}
\caption{Comparación de calidad de soluciones entre algoritmos para la \hyperref[busqueda:universidades]{búsqueda de univerdades}} 
\label{tabla:comp3}
\end{center}
\end{table}

\subsection{Búsqueda de atracciones turísticas}\label{res:busAtracciones}
En las búsquedas realizadas sobre la base de datos de atracciones turísticas las soluciones obtenidas a partir de las modificaciones propuestas en este trabajo de los algoritmos PAC, como puede observarse en la tabla \ref{tabla:comp4}, mejoran a los originales en todos los casos. Es para destacar que en en este escenario el algoritmo goloso supera a $alg1$. La búsqueda tabú consiguió mejores resultados para los algoritmos PAC. Por otro lado en el algoritmo goloso, en contraste a lo sucedido en las demás consultas, la heurística de búsqueda no obtuvo mejores soluciones.

\begin{table}[H]
\begin{center}
\begin{tabular}{|c|c|c|c|c|c|c|c|c|}
\hline
$\gamma$&$alg1$&$alg2$&$alg3$&$alg4$&$alg5$&$alg6$&$alg7$&$alg8$ \\ \hline
0.1 & -6.49 & -22.89 & -22.30 & -9.99 & -0.05 & 0.00 & -0.52 & -0.52 \\
0.2 & -13.50 & -27.60 & -26.57 & -16.51 & -0.09 & 0.00 & -1.09 & -1.09 \\
0.3 & -21.10 & -32.76 & -29.73 & -20.58 & -0.15 & 0.00 & -1.70 & -1.70 \\
0.4 & -29.38 & -36.68 & -33.51 & -25.10 & -0.20 & 0.00 & -2.37 & -2.37 \\
0.5 & -38.43 & -42.05 & -38.44 & -30.94 & -0.27 & 0.00 & -3.10 & -3.10 \\
0.6 & -48.35 & -47.73 & -43.86 & -34.75 & -0.34 & 0.00 & -3.90 & -3.90 \\
0.7 & -59.29 & -51.47 & -50.54 & -43.46 & -0.41 & 0.00 & -4.78 & -4.78 \\
0.8 & -71.36 & -57.07 & -56.14 & -45.61 & 0.00 & 0.00 & -5.62 & -5.62 \\
0.9 & -84.97 & -60.59 & -59.81 & -44.04 & 0.00 & 0.00 & -7.33 & -7.33 \\
\hline 
\end{tabular}
\caption{Comparación de calidad de soluciones entre algoritmos para la \hyperref[busqueda:atracciones]{búsqueda de atracciones turísticas}}
\label{tabla:comp4}
\end{center}
\end{table}

%En la figura \ref{res:cit_bobo_diff} se visualiza el porcentaje de mejora del algoritmo $alg3$ para cada valor de $\gamma$ con respecto a $alg2$. En %promedio el porcentaje de mejora es de $3.5\%$ y la mayor diferencia se da para los valores de $\gamma$ $0.3$, $0.4$ y $0.5$
%y en los extremos la diferencia disminuye.
%\begin{figure}[H]
%  \centering
%    \includegraphics[width=1\textwidth]{img/cit_bobo_diff.png}
%  \caption{}
%  \label{res:cit_bobo_diff}
%\end{figure}

\chapter{Conclusiones}
\label{chap:conclusiones}
\section{Búsquedas generales}
A partir de los resultados obtenidos en las pruebas realizadas la aplicación de las búsquedas tabú siempre mejoró los resultados en cualquier circunstancia y sin modificar significativamente el tiempo de ejecución. En el caso de PAC con la generación de paquetes con BOBO, aplicando la búsqueda tabú el resultado mejoro considerablemente.

Comparando entre el algoritmo goloso y PAC, se obtuvo que el algoritmo goloso dio mejores resultados cuando la producción de paquetes se hizo con la estrategia BOBO pero no así con la estrategia de producción jerárquica. Como toda la familia de algoritmos golosos tienen la ventaja de poder ser escritos muy rápidamente, convirtiéndolos en una alternativa muy interesante en los casos que se necesite una solución a este tipo de problemas y sea aceptable que las soluciones no sean las mejores con respecto a otros métodos ya conocidos.

Las implementaciones de BOBO fueron en todos los casos las más rápidas resultando ser las ideales para instancias muy grandes sobre todo si se aplica al final una búsqueda tabú, que como se observó mejora mucho las soluciones en estos casos. En cambio HAC en las pruebas realizadas con un universo de elementos menor a los 10000, cuadruplica o quintuplica a los tiempos debido a su complejidad ($\mathcal{O}(N^{2}\lg n)$), pero en instancias más chicas como las atracciones turísticas los tiempos fueron muy cercanos entre ellos.

Las soluciones obtenidas usando Efficient-HAC y búsquedas golosas para valores de $\gamma$ menores a $0.5$ fueron \textquotedblleft similarmente buenas\textquotedblright , dado que los resultados en términos de función objetivo fueron cercanos. En cambio para $\gamma$ más cercanos a uno se observó que el valor de la función objetivo de las soluciones HAC resultaron significativamente mayores. En todos los casos se identificó que las relaciones interpaquetes son mejores en las soluciones del algoritmo goloso, logrando soluciones más interdependientes.

Al comparar \texttt{BOBO-10} y \texttt{BOBO-160} usando la selección simple se observa que la diferencia entre los valores de la función objetivo de cada solución aumenta a medida que $\gamma$ se acerca a $1$. Se supone que esto se debe a la estrategia \texttt{produce and choose}. El objetivo de la primer etapa es producir paquetes con máximo valor de similitud. En el caso de que se requiera una solución con mayor separación entre paquetes, al haber producido menos cantidad de paquetes existen menos posibilidades para generar una solución más dispersa. Un mismo análisis se podría hacer si se compara \texttt{BOBO-10} y \texttt{HAC}.

Comparando los resultados obtenidos al realizar la selección de a un candidato contra la selección de a pares se obtuvo que para \texttt{BOBO-160} y \texttt{HAC} los tiempos aumentaron a $40$ minutos y para \texttt{BOBO-10} a $2$ minutos. En cuanto al valor de la función objetivo el único beneficiado fue \texttt{BOBO-10} ya que para \texttt{HAC} empeoró y para \texttt{BOBO-160} el aumento fue muy pequeño en comparación al incremento de tiempo.

%\chapter{Implementación}
%\section{Cálculo de la similitud}
Previo a la ejecución de los algoritmos de búsquedas realizamos los cálculos entre todos los elementos involucrados, papers ya autores. Lo realizamos de esta manera con el objetivo de optimizar los tiempos de ejecución de las búsquedas ya que en cada paso cuando tiene que comparar dos items no necesita calcular el $\cos(\theta)$ de los vectores de sus perfiles y en cambio solo accede a una posición de una matriz que mantenemos en memoria.\\
Si bien de esta manera ganamos en tiempo de ejecución, lo que perdemos es la complejidad espacial ya que estamos manteniendo una matriz de entre $8$ y $22$ millones de registros. En nuestro caso al contar con instancias de no mas de $7000$ elementos no es un problema porque en ningún momento alcanza a pasar los $4$ GB de memoria, lo cuál encontramos en cualquier pc medianamente moderna.
\chapter{Trabajo futuro}
\label{chap:trabajos-futuros}
\section{Desarrollo}
\begin{itemize}
 \item Nuevas heurísticas para intentar mejorar las soluciones.
\end{itemize}


\section{User friendly}
\begin{itemize}
 \item Generar una interfaz de usuario amgigable.
 \item Una mejor manera de visualizar los resultados de manera on line.
\end{itemize}

\section{Performance}
\begin{itemize}
 \item Acelerar los tiempos de ejecución.
 \item Depuración del código.
 \item Ejecución en diferentes hilos.
 \item Tunning de la base de datos.
\end{itemize}

\section{Extensibilidad}
\begin{itemize}
 \item Generar una interfaz genérica para poder trabajar con datos de cualquier origen y poder ser explotados de la misma forma.
 \item Adaptar el código para poder lograr 
\end{itemize}

%\chapter{Apéndice}
%\section{Resultados de las ejecuciones de los algoritmos de búsqueda}
A continuación se muestra el valor de la función objetivo para cada uno de los distintos 
algoritmos ejecutados y el tiempo de ejecución final.
\subsection{Papers}
\Solucion
{}
{simple, por tuplas y proporcional}
{\texttt{HAC} y \texttt{BOBO-x}, con  $x \in$ $(10, 160)$}
{$\in$ $(0,1; 0,3; 0,5; 0,7; 0,9)$}
{10}
{5}
A continuación se muestran los valores de la función objetivo obtenidos:\\
\begin{table}[H]
\centering
  \resizebox{\textwidth}{!} {
    \begin{tabular}{|lc|cccc|}
    \hline
    ~  & ~ & \multicolumn{2}{|c}{Valor función objetivo} & \multicolumn{2}{c|}{Duración de la 
ejecución (mm:ss)} \\
    Algoritmo & gamma & Selección simple & Selección proporcional & Selección simple          
         & Selección proporcional \\ 
    \hline
    HAC & $0,1$ & $48,9470$  & $35,1979$ & $10:00$ & $40:00$ \\
    HAC & $0,3$ & $59,1852$  & $58,7049$ & $10:00$ & $40:00$ \\
    HAC & $0,5$ & $70,5931$  & $70,205$ & $10:00$ & $40:00$ \\
    HAC & $0,7$ & $82,0687$  & $81,8331$ & $10:00$ & $40:00$ \\
    HAC & $0,9$ & $93,8227$  & $93,7189$ & $10:00$ & $40:00$ \\
    BOBO-160 & $0,1$ & $33,3762$  & $35,1979$ & $6:00$ & $46:00$ \\
    BOBO-160 & $0,3$ & $33,2741$  & $34,4164$ & $6:00$ & $46:00$ \\
    BOBO-160 & $0,5$ & $37,3484$  & $37,0669$ & $6:00$ & $46:00$ \\
    BOBO-160 & $0,7$ & $40,4186$  & $40,1762$ & $6:00$ & $46:00$ \\
    BOBO-160 & $0,9$ & $49,0972$  & $44,9824$ & $6:00$ & $46:00$ \\
    BOBO-10 & $0,1$ & $29,3038$  & $30,5376$ & $1:30$ & $2:00$ \\
    BOBO-10 & $0,3$ & $25,9363$  & $26,6800$ & $1:30$ & $2:00$ \\
    BOBO-10 & $0,5$ & $20,9841$  & $22,9482$ & $1:30$ & $2:00$ \\
    BOBO-10 & $0,7$ & $22,3052$  & $23,2333$ & $1:30$ & $2:00$ \\
    BOBO-10 & $0,9$ & $18,8381$  & $21,9347$ & $1:30$ & $2:00$ \\
    BOBO-ex & $0,1$ & $35,5786$  & - & $14:00$ & - \\
    BOBO-ex & $0,3$ & $35,4117$  & - & $14:00$ & - \\
    BOBO-ex & $0,5$ & $39,4408$  & - & $14:00$ & - \\
    BOBO-ex & $0,7$ & $45,0940$  & - & $14:00$ & - \\
    BOBO-ex & $0,9$ & $51,2695$  & - & $14:00$ & - \\
    \hline
    \end{tabular}
  }
  \caption {Valor función objetivo y tiempo de ejecución para la búsqueda de papers similares}
\end{table}
Al comparar \texttt{BOBO-10} y \texttt{BOBO-160} usando la selección simple se observa que la 
diferencia entre los valores de la función objetivo de cada solución aumenta a medida que $\gamma$ 
se acerca a $1$. Suponemos que esto se debe a la estrategia \texttt{produce and choose}. El 
objetivo de la primer etapa es producir bundles con máximo valor de similitud. En el caso de que se 
requiera una solución con mayor separación entre bundles, al haber producido menos cantidad de 
bundles existen menos posibilidades para generar una solución más dispersa. Un mismo análisis se 
podría hacer si comparamos \texttt{BOBO-10} y \texttt{HAC}.\\
Comparando los resultados obtenidos al realizar la selección de a un candidato contra la selección 
de a pares obtuvimos que para \texttt{BOBO-160} y \texttt{HAC} los tiempos aumentaron a $40$ 
minutos y para \texttt{BOBO-10} a $2$ minutos. En cuanto al valor de la función objetivo el único 
beneficiado fue \texttt{BOBO-10} ya que para \texttt{HAC} empeoró y para \texttt{BOBO-160} el 
aumento fue muy pequeño en comparación al incremento de tiempo.
\subsection{Autores}
\Solucion
{}
{simple y proporcional}
{\texttt{HAC} y \texttt{BOBO-x}, con  $x \in$ $(10, 160)$ y \texttt{BOBO-ex}}
{$\in$ $(0,1; 0,3; 0,5; 0,7; 0,9)$}
{10 y 20}
{5 y 10}
\begin{table}[H]
\centering
  \resizebox{\textwidth}{!} {
    \begin{tabular}{|lc|cccc|}
    \hline
    ~  & ~ & \multicolumn{2}{|c}{Valor función objetivo} & \multicolumn{2}{c|}{Duración de la 
ejecución (mm:ss)} \\
    Algoritmo & gamma & Selección simple & Selección proporcional & Selección simple          
         & Selección proporcional \\ 
    \hline
    HAC & $0,1$ & $50,5$  & $50,5$ & $8:40$ & $9:00$ \\
    HAC & $0,3$ & $61,5$  & $61,5$ & $8:40$ & $9:00$ \\
    HAC & $0,5$ & $72,5$  & $72,5$ & $8:40$ & $9:00$ \\
    HAC & $0,7$ & $83,5$  & $83,5$ & $8:40$ & $9:00$ \\
    HAC & $0,9$ & $94,5$  & $94,5$ & $8:40$ & $9:00$ \\
    BOBO-160 & $0,1$ & $38,6883$  & $36,8917$ & $10:00$ & $8:00$ \\
    BOBO-160 & $0,3$ & $43,4380$  & $41,4767$ & $10:00$ & $8:00$ \\
    BOBO-160 & $0,5$ & $47,3612$  & $47,9337$ & $10:00$ & $8:00$ \\
    BOBO-160 & $0,7$ & $51,5712$  & $52,1462$ & $10:00$ & $8:00$ \\
    BOBO-160 & $0,9$ & $57,2009$  & $57,5260$ & $10:00$ & $8:00$ \\
    BOBO-10 & $0,1$ & $30,2956$  & $31,6080$ & $2:30$ & $2:30$ \\
    BOBO-10 & $0,3$ & $33,6794$  & $35,4411$ & $2:30$ & $2:30$ \\
    BOBO-10 & $0,5$ & $33,0506$  & $37,5776$ & $2:30$ & $2:30$ \\
    BOBO-10 & $0,7$ & $37,2855$  & $34,5657$ & $2:30$ & $2:30$ \\
    BOBO-10 & $0,9$ & $41,0119$  & $35,2511$ & $2:30$ & $2:30$ \\
    BOBO-ex & $0,1$ & $39,9767$  & $39,9767$ & $27:00$ & $27:00$ \\
    BOBO-ex & $0,3$ & $44,0043$  & $44,0043$ & $27:00$ & $27:00$ \\
    BOBO-ex & $0,5$ & $48,5481$  & $48,5481$ & $27:00$ & $27:00$ \\
    BOBO-ex & $0,7$ & $53,1993$  & $53,1993$ & $27:00$ & $27:00$ \\
    BOBO-ex & $0,9$ & $57,7400$  & $57,7400$ & $27:00$ & $27:00$ \\
    \hline
    \end{tabular}
  }
  \caption {Valor función objetivo y tiempo de ejecución para la búsqueda de autores similares}
\end{table}



%%%% BIBLIOGRAFIA
\backmatter
\bibliographystyle{plain}
\bibliography{tesis} 

\end{document}
