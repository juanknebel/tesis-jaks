\documentclass[11pt,a4paper,twoside]{tesis}
% SI NO PENSAS IMPRIMIRLO EN FORMATO LIBRO PODES USAR
%\documentclass[11pt,a4paper]{tesis}

\usepackage{graphicx}
\usepackage[utf8]{inputenc}
\usepackage[spanish]{babel}
\usepackage[left=3cm,right=3cm,bottom=3.5cm,top=3.5cm]{geometry}

\begin{document}

%%%% CARATULA
% Comentar y descomentar según corresponda
\def\titulo{Licenciada }
%\def\titulo{Licenciado }

\def\autor{Amit Stein}
\def\autor{Juan Andres Knebel}
\def\tituloTesis{Composite Retrival: \mbox{Algo}}
\def\runtitulo{Composite Retrival en español}
\def\runtitle{Composite Retrival en ingles}
\def\director{Obi-Wan Kenobi}
\def\codirector{Master Yoda}
\def\lugar{Buenos Aires, 2014}
\input{caratula}

%%%% ABSTRACTS, AGRADECIMIENTOS Y DEDICATORIA
\frontmatter
\pagestyle{empty}
%\begin{center}
%\large \bf \runtitulo
%\end{center}
%\vspace{1cm}
\chapter*{\runtitulo}

\noindent Las búsquedas tradicionales ofrecen soluciones que solo tienen en cuenta un solo atributo de los elementos y no la relación que estos tienen con el resto del universo. Las mismas suelen ofrecer una lista ordenada de resultados relacionadas con
el criterio utilizado. Ocasionando, muchas veces, la necesidad de reformular la consulta original para así lograr una solución adecuada al criterio de búsqueda elegido previamente.\\
El artículo Composite Retrieval of Diverse and Complementary Bundles \cite{compositeRetrival}, propone que el resultado de la búsqueda este compuesta por conjuntos de elementos. Estos conjuntos, conocidos como bundles, tienen la particularidad que los elementos que contienen se encuentran relacionados internamente bajo algún criterio de similitud y a la vez son complementarios. De forma tal que la relación de los elementos de un bundle satisfagan las expectativas del usuario y no tenga la necesidad de realizar una nueva intervención logrando así una mejor experiencia de búsqueda.\\
En este trabajo se implementa este tipo de resultados para consultas realizadas sobre la base de datos de \cite{dataDrive}, que contiene artículos científicos pertenecientes a la Ingeniería de Software. Para ello se utilizan los algoritmos de \cite{compositeRetrival} a los que se le aplican modificaciones para optimizar el rendimiento y se proponen nuevos enfoques en orden de mejorar los resultados que se obtuvieron originalmente.

\bigskip

\noindent\textbf{Palabras claves:} Composite Retrieval, Agrupamiento, Similitud, Búsqueda Tabú, Algoritmo Goloso.

\cleardoublepage
%\begin{center}
%\large \bf \runtitle
%\end{center}
%\vspace{1cm}
\chapter*{\runtitle}

\noindent 

Los usuarios de Internet, u otros medios informáticos, constantemente realizan búsquedas para hallar elementos de su interés, generalmente a través de palabras o frases. A lo largo del desarrollo de Internet las búsquedas fueron adquiriendo cada vez más importancia por la enorme cantidad de información que día a día se almacena en los distintos servidores del planeta.

Las búsquedas tradicionales ofrecen soluciones que únicamente tienen en cuenta un atributo de los elementos y no la relación que éstos tienen con el resto del universo. Las mismas suelen ofrecer una lista ordenada de resultados relacionados con el criterio de búsqueda utilizado. Muchas veces esto ocasiona la necesidad de reformular la consulta original para así lograr una solución adecuada al criterio de búsqueda elegido previamente.

Como respuesta a este último comportamiento se propone que el resultado de la búsqueda esté compuesta por conjuntos de elementos. Estos conjuntos, conocidos como {\em paquetes}, tienen la particularidad de que los elementos conformantes se encuentran relacionados internamente bajo algún criterio de similitud y a la vez son complementarios. El propósito es que la relación de los elementos de un paquete satisfaga las expectativas del usuario y no tenga la necesidad de realizar una nueva intervención, logrando así una mejor experiencia de búsqueda.

Este tipo de búsqueda de elementos complementarios y compatibles, conectados mediante algún tipo de relación, generó el estudio de la {\em recuperación compuesta de información}. Es decir, el estudio de métodos para crear, recuperar y valorizar respuestas compuestas, organizando los resultados en paquetes de elementos. Esto ayuda a los usuarios a explorar un gran número de elementos relevantes de una forma más eficiente.

En este trabajo se proponen algoritmos heurísticos para realizar este tipo de búsquedas. Los mismos son experimentalmente evaluados en dos bases de datos una de artículos relacionados con ingeniería de software y otra de atracciones turísticas, a partir de esta evaluación podemos concluir que tienen una muy buena performance, mejorando los existentes en la literatura.
\bigskip


\noindent\textbf{Palabras claves:} Recuperación de la información, Agrupamiento, Heuristícas.

\cleardoublepage
\chapter*{Agradecimientos}

\noindent A Mirta que nos desleita con sus almuerzos y preguntas sobresalientes.
\noindent A Laura que estuvo a mi lado
\noindent A mis padres, a Abigail y René y a Edith que me acompañaron y ayudaron durante toda la carrera y A Mila y Florián que lo hicieron este último tramo. 
\noindent a todos los profesores y ayudantes del DC y en especial a Paula, Isabel y Estebán por todo lo enseñado.
\noindent A Dario, Julio y Sebastián con los que además de compañeros nos hicimos buenos amigos.
 % OPCIONAL: comentar si no se quiere

\cleardoublepage
\hfill \textit{A Nestor y Chavez que nos cuidan y guian desde arriba}  % OPCIONAL: comentar si no se quiere

\cleardoublepage
\tableofcontents

\mainmatter
\pagestyle{headings}

%%%% ACA VA EL CONTENIDO DE LA TESIS

\chapter{Introducción}
\section{Explicación}
{\begin{small}%
\begin{flushright}%
\it
Explicar el tema en que consiste
\end{flushright}%
\end{small}%
\vspace{.5cm}}

\section{Que se va a hacer}
{\begin{small}%
\begin{flushright}%
\it
Las cosas que se van hacer
\end{flushright}%
\end{small}%
\vspace{.5cm}}

\chapter{Desarrollo}
\chapter{Resultados}
\chapter{Conclusiones}

%%%% BIBLIOGRAFIA
\backmatter
%\bibliography{tesis}

\end{document}
