\section{Definición de similitud y complementariedad}
Como paso preliminar a la búsqueda de las soluciones se definió la noción de similitud y 
complementariedad para los grupos de elementos.
\subsection{Papers}
Ya se contaba con los perfiles de los papers, con lo cuál decidimos definir la similitud entre 
papers a través de la distancia entre los perfiles de cada uno de ellos. Cada perfil se interpreta 
como un vector de \texttt{n} posiciones, dónde cada posición pertenece a un tópico particular. 
Entonces la similitud entre dos papers la interpretamos como el ángulo vectorial entre dos 
vectores.\\
La complementariedad de dos papers se definió al lugar de presentación de dicho paper.
\subsection{Autores}
Para determinar que \textquotedblleft tan iguales\textquotedblright son dos autores tuvimos que 
crear primero un perfil de autor para luego poder compararlos. Para lograrlo lo primero que se hizo 
fue tener en cuenta todos los perfiles de papers en los que participaron cada uno de ellos. Como 
cada paper tenía ya calculado su perfil el cuál representamos como un vector de \texttt{n} 
posiciones, dónde cada posición pertenece a un tópico particular. De esta manera tomamos todos 
perfiles (vectores) de los papers en los cuáles el autor participó e hicimos una suma componente a 
componente para así obtener un perfil de cada uno de los autores.\\
El paso siguiente fue determinar la similitud de los autores, la primer aproximación fue similar al 
cálculo de similitud de los papers y calculamos los ángulos entre todos los vectores. Como similitud
alternativa decidimos restar el perfil de cada autor componente a componente y calcular el valor de 
su norma, así de esta forma a mayor norma mayor similitud entre autores.\\
La complementariedad de dos papers se definió al lugar de pertenencia del autor en cuestión.
\section{Generación de soluciones}
\subsection{Papers}
Originalmente la base de datos contenía unos 7777 papers, de los cuáles se tuvo que hacer una 
depuración, ya que había papers que no tenían ningún autor asociado o perfil creado. Luego de la 
depuración obtuvimos 4937 que cumplen los requisitos para la búsqueda de las soluciones.\\
Se generaron soluciones con distintos valores de $\gamma$ ($0,1$; $0,3$; $0,5$; $0,7$; $0,9$), las 
cuáles fueron de 10 bundles con 5 items cada una.\\
Las heurísticas utilizadas fueran la generación jerárquica (\textbf{HAC}) y la aleatoria 
(\textbf{BOBO}) para la producción de bundles y una estrategia en forma de algoritmo goloso para la 
selección de los bundles.