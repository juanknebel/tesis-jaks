%\begin{center}
%\large \bf \runtitulo
%\end{center}
%\vspace{1cm}
\chapter*{\runtitulo}

\noindent 

Los usuarios de Internet, mediante el uso de motores de búsqueda, realizan búsquedas de diferentes elementos, como pueden ser, páginas productos o mensajes.

El crecimiento de Internet provocó que las búsquedas fueran adquiriendo cada vez mayor protagonismo y relevancia. Como consecuencia, los motores de búsqueda comenzaron a ser cada vez más específicos y complejos.

En esta tesis se estudia el caso en que el usuario requiere una respuesta que contiene más de un elemento. En este escenario, con los motores de búsqueda actuales, es el usuario el responsable de generar su propia solución realizando varias búsquedas hasta encontrar lo buscado. Éste proceso puede requerir muchos intentos de búsquedas combinando palabras, frases o atributos de los elementos buscados. La calidad de la solución estará directamente relacionada con la imaginación y criterios utilizados por el usuario.

Como respuesta a las intenciones de búsquedas aquí mencionadas y los problemas que presentan, se propone que el resultado de la búsqueda esté formado por conjuntos de elementos. Estos conjuntos, llamados {\em paquetes}, tendrán la particularidad de estar integrados por elementos relacionados bajo algún criterio de similitud y, al mismo tiempo, complementarios. El objetivo es lograr una mejor experiencia de búsqueda para el usuario. La propuesta es que el usuario intervenga al comienzo de la búsqueda y luego evalúe qué paquetes satisfacen sus expectativas, sabiendo que los elementos a su disposición se encuentran relacionados entre sí.

Este tipo de búsqueda de elementos complementarios y compatibles, conectados mediante algún tipo de relación, generó el estudio de la {\em Recuperación de la Información Compuesta}. Es decir, el estudio de métodos para crear, recuperar y valorizar respuestas compuestas, organizando los resultados en paquetes de elementos. Esto ayuda a los usuarios a explorar un gran número de elementos relevantes de una forma más eficiente.

En este trabajo se proponen algoritmos heurísticos para realizar este tipo de búsquedas. Los mismos son experimentalmente evaluados en dos bases de datos, una de artículos académicos relacionados con Ingeniería de Software y la otra sobre atracciones turísticas. A partir de las evaluaciones realizadas se concluirá que los algoritmos propuestos se comportan de acuerdo a las expectativas y mejorando los existentes en la literatura.
\bigskip



\noindent\textbf{Palabras claves:} Recuperación de la información, Agrupamiento, Heurísticas.