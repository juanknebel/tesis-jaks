%\begin{center}
%\large \bf \runtitulo
%\end{center}
%\vspace{1cm}
\chapter*{\runtitulo}

\noindent Las búsquedas tradicionales ofrecen soluciones que solo tienen en cuenta un solo atributo de los elementos y no la relación que estos tienen con el resto del universo. Las mismas suelen ofrecer una lista ordenada de resultados relacionadas con
el criterio utilizado. Ocasionando, muchas veces, la necesidad de reformular la consulta original para así lograr una solución adecuada al criterio de búsqueda elegido previamente.\\
El artículo Composite Retrieval of Diverse and Complementary Bundles \cite{compositeRetrival}, propone que el resultado de la búsqueda este compuesta por conjuntos de elementos. Estos conjuntos, conocidos como bundles, tienen la particularidad que los elementos que contienen se encuentran relacionados internamente bajo algún criterio de similitud y a la vez son complementarios. De forma tal que la relación de los elementos de un bundle satisfagan las expectativas del usuario y no tenga la necesidad de realizar una nueva intervención logrando así una mejor experiencia de búsqueda.\\
En este trabajo se implementa este tipo de resultados para consultas realizadas sobre la base de datos de \cite{dataDrive}, que contiene artículos científicos pertenecientes a la Ingeniería de Software. Para ello se utilizan los algoritmos de \cite{compositeRetrival} a los que se le aplican modificaciones para optimizar el rendimiento y se proponen nuevos enfoques en orden de mejorar los resultados que se obtuvieron originalmente.

\bigskip

\noindent\textbf{Palabras claves:} Composite Retrieval, Agrupamiento, Similitud, Búsqueda Tabú, Algoritmo Goloso.