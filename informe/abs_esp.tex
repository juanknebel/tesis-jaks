%\begin{center}
%\large \bf \runtitulo
%\end{center}
%\vspace{1cm}
\chapter*{\runtitulo}

\noindent Las búsquedas tradicionales ofrecen soluciones que únicamente tienen en cuenta un solo atributo de los elementos y no la relación que estos tienen con el resto del universo. Las mismas suelen ofrecer una lista ordenada de resultados relacionadas con el criterio de búsqueda utilizado. Muchas veces ésto ocasiona la necesidad de reformular la consulta original para así lograr una solución adecuada al criterio de búsqueda elegido previamente.

El artículo Composite Retrieval of Diverse and Complementary Bundles \cite{compositeRetrival}, propone que el resultado de la búsqueda esté compuesta por conjuntos de elementos. Estos conjuntos, conocidos como paquetes (o bundles en inglés), tienen la particularidad de que los elementos conformantes se encuentran relacionados internamente bajo algún criterio de similitud y a la vez son complementarios. El propósito es que la relación de los elementos de un paquete satisfagan las expectativas del usuario y no tenga la necesidad de realizar una nueva intervención, logrando así una mejor experiencia de búsqueda.

En este trabajo se implementa este tipo de búsquedas para consultas realizadas sobre una base de datos que contiene artículos científicos pertenecientes a la Ingeniería de Software \cite{dataDrive}. Para ello se desarrollan modificaciones a los algoritmos presentados en \cite{compositeRetrival} para optimizar el rendimiento. Adicionalmente se proponen nuevos enfoques en orden de mejorar los resultados que se obtuvieron originalmente.

\bigskip


\noindent\textbf{Palabras claves:} Recuperación de la información, Agrupamiento, Heuristícas.