%\begin{center}
%\large \bf \runtitulo
%\end{center}
%\vspace{1cm}
\chapter*{\runtitulo}

\noindent 

Los usuarios de los motores de búsquedas de Internet, como por ejemplo los buscadores genéricos Google, Yahoo y otros similares o de sitios de compras de on-line similares a Amazon, Mercado Libre, constantemente realizan búsquedas para hallar elementos de su interés, generalmente a través de palabras o frases. A lo largo del desarrollo de Internet las búsquedas fueron adquiriendo cada vez más importancia por la enorme cantidad de información que día a día se almacena en los distintos servidores del planeta.

Las búsquedas tradicionales ofrecen soluciones que únicamente tienen en cuenta un atributo de los elementos y no la relación que éstos tienen con el resto del universo. Las mismas suelen ofrecer una lista ordenada de resultados relacionados con el criterio de búsqueda utilizado. 

Cuando se ofrece este tipo de resultados y el caso particular requiera que la solución deba contener más de un elemento, será necesario que el usuario genere su solución realizando varias búsquedas hasta completar su deseo inicial. Éste último proceso puede requerir muchos intentos de búsquedas combinando palabras, frases o atributos de los elementos buscados.

Como respuesta al último comportamiento mencionado en el párrafo anterior, se propone que el resultado de la búsqueda esté dado por conjuntos de elementos. Estos conjuntos, llamados {\em paquetes}, tienen la particularidad de estar formados por elementos que se encuentran relacionados internamente bajo algún criterio de similitud y al mismo tiempo ser complementarios. El propósito es que la relación de los elementos de un paquete satisfaga las expectativas del usuario y no tenga la necesidad de realizar una nueva intervención, logrando así una mejor experiencia de búsqueda.

Este tipo de búsqueda de elementos complementarios y compatibles, conectados mediante algún tipo de relación, generó el estudio de la {\em Recuperación de la Información Combinada}. Es decir, el estudio de métodos para crear, recuperar y valorizar respuestas compuestas, organizando los resultados en paquetes de elementos. Esto ayuda a los usuarios a explorar un gran número de elementos relevantes de una forma más eficiente.

En esta tesis se proponen algoritmos heurísticos para realizar este tipo de búsquedas. Los mismos son experimentalmente evaluados en dos bases de datos, una de artículos académicos relacionados con Ingeniería de Software y otra de atracciones turísticas. A partir de esta evaluación podemos concluir que los algoritmos propuestos tienen una muy buena performance, mejorando los existentes en la literatura.
\bigskip


\noindent\textbf{Palabras claves:} Recuperación de la información, Agrupamiento, Heuristícas.