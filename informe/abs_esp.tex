%\begin{center}
%\large \bf \runtitulo
%\end{center}
%\vspace{1cm}
\chapter*{\runtitulo}

\noindent 

Los usuarios de Internet, u otros medios informáticos, constantemente realizan búsquedas para hallar elementos de su interés, generalmente a través de palabras o frases. A lo largo del desarrollo de Internet las búsquedas fueron adquiriendo cada vez más importancia por la enorme cantidad de información que día a día se almacena en los distintos servidores del planeta.

Las búsquedas tradicionales ofrecen soluciones que únicamente tienen en cuenta un atributo de los elementos y no la relación que éstos tienen con el resto del universo. Las mismas suelen ofrecer una lista ordenada de resultados relacionados con el criterio de búsqueda utilizado. Muchas veces esto ocasiona la necesidad de reformular la consulta original para así lograr una solución adecuada al criterio de búsqueda elegido previamente.

Como respuesta a este último comportamiento se propone que el resultado de la búsqueda esté compuesta por conjuntos de elementos. Estos conjuntos, conocidos como {\em paquetes}, tienen la particularidad de que los elementos conformantes se encuentran relacionados internamente bajo algún criterio de similitud y a la vez son complementarios. El propósito es que la relación de los elementos de un paquete satisfagan las expectativas del usuario y no tenga la necesidad de realizar una nueva intervención logrando así una mejor experiencia de búsqueda.

Este tipo de búsqueda de elementos complementarios y compatibles, conectados mediante algún tipo de relación, generó el estudio de la {\em recuperación compuesta de información}, es decir, el estudio de métodos para crear, recuperar y valorizar respuestas compuestas, organizando los resultados en paquetes de elementos. Esto ayuda a los usuarios a explorar un gran número de elementos relevantes de una forma más eficiente.

En este trabajo se proponen algoritmos heurísticos para realizar este tipo de búsquedas. Los mismos son experimentalmente evaluados en dos bases de datos una de artículos relacionados con ingeniería de software y otra de atracciones turísticas. A partir de esta evaluación podemos concluir que tienen una muy buena performance, mejorando los existentes en la literatura.
\bigskip


\noindent\textbf{Palabras claves:} Recuperación de la información, Agrupamiento, Heuristícas.