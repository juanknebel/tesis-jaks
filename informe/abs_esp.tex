%\begin{center}
%\large \bf \runtitulo
%\end{center}
%\vspace{1cm}
\chapter*{\runtitulo}

\noindent Las búsquedas tradicionales nos ofrecen una solución que solo tiene en cuenta un atributo de los elementos y no la relación que éstos tienen con el resto del universo. Ocasionando que muchas veces se necesiten reformular la consulta original para así lograr la solución que el usuario quiere encontrar.\\
Es por eso que surge \textbf{Composite Retrieval}, teniendo como objetivo agrupar conjuntos de elementos bajo un mismo atributo, al mismo tiempo que éstos son complementarios entre sí. Facilitando al usuario la elección de un grupo de elementos que satisfagan su consulta original.

\bigskip

\noindent\textbf{Palabras claves:} Composite Retrieval, Similitud, Complementaridad, Búsqueda Tabú, Generación de Bundles.