%\begin{center}
%\large \bf \runtitulo
%\end{center}
%\vspace{1cm}
\chapter*{\runtitulo}

\noindent 

Los usuarios Internet constantemente realizan búsquedas de diferentes elementos como pueden ser páginas, productos o mensajes entre otros. Estas búsquedas se realizan mediante el uso de motores de búsquedas.

El crecimiento de Internet provocó que las búsquedas fueran adquiriendo cada vez mayor protagonismo y al mismo tiempo relevancia. Como consecuencia, los motores de búsqueda comenzaron a ser cada vez más específicos y complejos.

En esta tesis se estudia el caso que el usuario requiera una respuesta que contenga más de un elemento, las prácticas mencionadas en el párrafo anterior no cumplirán el objetivo del problema. El nuevo escenario propuesto obliga al usuario a generar su propia solución realizando varias búsquedas hasta completar su deseo inicial. Éste último proceso puede requerir muchos intentos de búsquedas combinando palabras, frases o atributos de los elementos buscados y la calidad de la solución estará directamente relacionada con la imaginación y criterio utilizado por el usuario.

Como respuesta a las intenciones de búsquedas aquí mencionadas y los problemas que presentan, se propone que la solución de la búsqueda esté formada por conjuntos de elementos. Estos conjuntos, llamados {\em paquetes}, tendrán la particularidad de estar formados por elementos que se encuentran relacionados internamente bajo algún criterio de similitud y al mismo tiempo serán complementarios. El propósito final es lograr una mejor experiencia de búsqueda para el usuario, lo cuál incluye que solo intervenga al comienzo de la acción y luego evalúe si los paquetes satisfacen sus expectativas, sabiendo que los elementos a su disposición se encuentran relacionados entre sí.

Este tipo de búsqueda de elementos complementarios y compatibles, conectados mediante algún tipo de relación, generó el estudio de la {\em Recuperación de la Información Compuesta}. Es decir, el estudio de métodos para crear, recuperar y valorizar respuestas compuestas, organizando los resultados en paquetes de elementos. Esto ayuda a los usuarios a explorar un gran número de elementos relevantes de una forma más eficiente.

En este trabajo se proponen algoritmos heurísticos para realizar este tipo de búsquedas. Los mismos son experimentalmente evaluados en dos bases de datos, una de artículos académicos relacionados con Ingeniería de Software y la restante sobre atracciones turísticas. A partir de las evaluaciones realizadas se concluirá que los algoritmos propuestos se comportan de acuerdo a las expectativas y mejorando los existentes en la literatura.
\bigskip



\noindent\textbf{Palabras claves:} Recuperación de la información, Agrupamiento, Heurísticas.