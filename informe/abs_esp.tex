%\begin{center}
%\large \bf \runtitulo
%\end{center}
%\vspace{1cm}
\chapter*{\runtitulo}

\noindent 

Los motores de búsqueda son la herramienta utilizada por los usuarios de Internet para obtener información acerca de diferentes elementos, como pueden ser sitios web, productos o mensajes.

El crecimiento de Internet provocó el incremento y protagonismo de las búsquedas. Como consecuencia, los motores comenzaron a ser cada vez más específicos y complejos. 

Esta Tesis se focaliza en el caso del usuario que requiera más de un elemento como respuesta a su búsqueda. Con las herramientas actuales el usuario debe repetir las búsquedas reiteradas veces combinando palabras, frases o atributos de los elementos. La calidad de las soluciones estará sujetas a la imaginación y criterios del usuario.

Como solución a esta inquietud y los problemas que presenta, se propone que el resultado de la búsqueda esté formado por conjuntos de elementos. Estos conjuntos, llamados {\em paquetes}, tendrán la particularidad de estar integrados por elementos relacionados bajo algún criterio de similitud y, al mismo tiempo, complementarios. El objetivo es que el usuario pueda lograr una mejor experiencia de búsqueda interviniendo en el inicio de la consulta y luego evaluando qué paquetes satisfacen sus inquietudes.

En este trabajo se presentan algoritmos heurísticos para devolver este tipo de resultados. Los mismos son evaluados en dos bases de datos (una corresponde a artículos académicos relacionados con Ingeniería de Software y la otra a artículos sobre atracciones turísticas).

Se llegará a la conclusión de que los algoritmos propuestos se comportan de acuerdo a las expectativas y mejorando los existentes en la literatura.


\bigskip



\noindent\textbf{Palabras claves:} Recuperación de la información, Agrupamiento, Heurísticas.