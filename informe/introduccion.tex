\section{Estado Actual de las Búsquedas}
{\begin{small}%
\begin{flushright}%
\it An algorithm must be seen to be believed.\\Donald Knuth.
\end{flushright}%
\end{small}%
\vspace{.5cm}}
La \textbf{Recuperación de la Información} (IR por Information Retrieval en inglés), es la actividad de obtener información relevante a partir de una inmensa colección de datos y con criterios de lo más variados, desde el resultado de la final del mundial de fútbol, los libros de un autor y hasta el mail de la confirmación de una compra.\\
Los motores de búsquedas de la web como Google, Yahoo, etc son los clásicos ejemplos de una aplicación de IR. El proceso comienza cuando el usuario ingresa una consulta en el sistema y espera obtener un resultado en base a su búsqueda. El problema radica en que dificilmente un único elemento del universo concuerde con la consulta elegida, en cambio lo que ocurre es que se cuenta con un conjunto de elementos que coinciden pero con grados de relevancia diferentes (ranking de resultados). Para obtener el ranking de resultados los sistemas de IR trabajan con una representación lógica de los elementos que incluye los metadatos necesarios para operar sobre ellos.\\
El principal problema con los ranking de resultados es que únicamente se compara la consulta de la búsqueda con los metadatos de los objetos dejando de lado el análisis de los elementos entre sí. Conviritiendo, en ocasiones, al proceso en una acción tediosa y repetitiva ya que deberá cambiar la consulta original hasta lograr encontrar el resultado deseado.\\
El término \textbf{Composite Retrieval} mencionado en el paper \textit{\textquotedblleft Composite Retrieval of Diverse and Complementary Bundles\textquotedblright}\cite{compositeRetrival} propone, en lugar de entregar una lista vertical de elementos, agruparlos bajo algún criterio de similitud. Al usuario se le presentará una lista de conjuntos de elementos relacionados con su búsqueda
ordenados de forma lógica con la finalidad de que uno o más conjuntos satisfaga las expectativas del mismo sin necesidad de una nueva intervención para refinar su búsqueda.

\section{Motivación}
Planear un viaje típicamente requiere realizar múltiples búsquedas para juntar información de los diferentes lugares, leer opiniones acerca de las atracciones turísticas y validar las distancias geográficas de los lugares a visitar.\\
En una búsqueda típica los resultados obtenidos son una larga lista ordenada por la relevancia del criterio de la consulta. Este tipo de resultados no otorgan respuestas que relacionen el criterio buscado con otros elementos que se relacionan. Por ejemplo el caso de que un usuario de una tienda online de venta de discos, realice una búsqueda discos de rock el resultado puede ser una lista como la siguiente:\\
\begin{itemize}
  \item Physical Graffiti - Led Zeppelin
  \item Led Zeppelin - Led Zeppelin
  \item It's Hard - The Who
  \item Perfect Strangers - Deep Purple
  \item El Cielo Puede Esperar - Attaque 77
  \item Wheels of Fire - Cream
  \item Confesiones de Invierno - Sui Generis
  \item The White Album - The Beatles
  \item Innuendo - Queen
  \item Sticky Fingers - The Rolling Stones
  \item Kamikaze - Luis Alberto Spinetta
\end{itemize}

Si este usuario está interesado en comprar discos, cuenta con un presupuesto limitado y quiera adquirir diversos artistas, el usuario debería realizar reiteradas búsquedas para encontrar el conjunto de discos que le interesen.\\

El resultado que se propone está diseñado para aquellas consultas que requieren obtener un conjunto de elementos que se relacionan como respuesta. En las técnicas tradicionales de clusterización la agrupación se hace por la similitud entre ítems. En el ejemplo de los discos con una clusterización tradicional, donde la similitud sea el género musical, seguramente se generen tantos clúster como géneros de discos existan y en cada clúster estén todos los discos de ese género. Con este resultado se deberá explorar todo el clúster para elegir los discos; donde la información es redundante porque algunos discos son tan parecidos que no se comprarían juntos.\\

El resultado que se obtiene con ``Composite Retrieval of Diverse and Complementary Bundles'' se ajuste al presupuesto y a que los ítems sean complementarios, de modo tal que el usuario seleccionando un bundle (es el nombre que se le da al agrupamiento de ítems) cubra su necesidad. Con los discos se establece la complementariedad de algún atributo, como puede ser el origen de la banda, cada bundle es una opción de discos que el usuario puede comprar porque cumplen con el presupuesto, son del mismo género de música y hay variedad entre ellos. Una solución posible es:
\begin{itemize}
  \item Bundle 1:
  \begin{itemize}
    \item Physical Graffiti - Led Zeppelin (Inglaterra)
    \item After chabón - Sumo (Argentina)
    \item Back in Black - AC/DC (Estados Unidos)
  \end{itemize}
  \item Bundle 2:
  \begin{itemize}
    \item Natty Dread - Bob Marley (Jamaica)
    \item El ritual de la Banana - Los Pericos (Argentina)
    \item Labour of Love - UB40 (Inglaterra)
  \end{itemize}
	  \item Bundle 3:
  \begin{itemize}
    \item Ramones - Ramones (Estados Unidos)
    \item El Cielo Puede Esperar - Attaque 77 (Argentina)
    \item Sandinista! - The Clash (Inglaterra)
  \end{itemize}
\end{itemize}

Lo que se propone es otorgar un conjunto de \textbf{bundles} que cumplen con las siguientes propiedades:
\begin{itemize}
  \item \textbf{Cubrimiento}: Maximizar la cantidad de elementos en el bundle.
  \item \textbf{Compatibilidad}: Los elementos del bundle deben ser similares.
  \item \textbf{Validez}: El costo total de los elementos del bundle no debe superar el presupuesto.
  \item \textbf{Diversificada}: Los bundles entre si deben ser diversos.
\end{itemize}

\section{Instancia elegida}
Este trabajo consiste en implementar el diseño de resultados de \textit{\textquotedblleft Composite Retrieval of Diverse and Complementary Bundles\textquotedblright}\cite{compositeRetrival} de consultas sobre la base de datos de los artículos de \textit{\textquotedblleft A Data-Driven Journey through Software Engineering Research\textquotedblright}\cite{dataDrive}. La decisión de utilizar esta base de datos es por la completitud de la información y que el tamaño de la cantidad de elementos que contiene requiere de optimizar los algoritmos propuestos.\\

La base de datos contiene cerca de $7800$ artículos, de los que se tiene los autores que participaron en ellos, en la conferencia que fueron presentados y lo más importante es que se cuenta con una clasificación ya realizada de los tópicos a los que hacen referencia cada uno de ellos. Ésta última característica es llamada \texttt{topicProfile} y esta expresada en porcentajes para cada uno de tópicos que son tratados.\\

Cuenta con $9800$ autores y de ellos tenemos la información de que a universidad pertenecen, que será útil como criterio de diversidad y además de cada una de las universidades se sabe a que región pertenece cada una.\\

El \texttt{topicProfile} es muy importante porque, como veremos más adelante, nos permitirá definir la similitud no sólo entre los artículos, sino que también entre los autores y las universidades de la base da datos de una manera prácticamente directa.
