\section{Motivación}
{\begin{small}%
\begin{flushright}%
\it An algorithm must be seen to be believed.\\Donald Knuth.
\end{flushright}%
\end{small}%
\vspace{.5cm}}
La motivación para explorar otro tipo de búsqueda es encontrar resultados que provean al 
usuario una experiencia más satisfactoria y una visión más amplia de sus búsquedas que los 
resultados lineales no entregan.\\
Para generar estas soluciones utilizamos \textbf{Composite Retrieval} que necesita de tres 
ejes fundamentales que son: \textit{similitud}, \textit{complementariedad} y \textit{costo}. 
Cada uno de estos elementos debe poder ser calculado para todos los elementos del dominio en el que 
se realicen las búsquedas. Por ejemplo, si nos encontramos en el dominio de atracciones turísticas 
de una ciudad, la similitud podría ser la distancia entre ellos, la complementariedad que tipo de 
atracción (parque, museo, monumento, etc) y el costo sería el valor de cada visita. Para el dominio 
de discos de música, la similitud el género musical, la complementariedad el país de origen y el 
costo el valor del disco.\\
En una búsqueda tradicional los resultados obtenidos son lineales al criterio seleccionado y no 
otorgan respuestas que relacionen el criterio buscado con otros elementos del mismo tipo o 
con elementos que probablemente están relacionados pero no son del mismo tipo. Lo que ocasiona 
tener que realizar repetidas búsquedas usando diferentes criterios para obtener los resultados 
esperados.\\
Lo que se propone con este tipo de búsquedas es otorgar un conjunto de \textbf{bundles}. Cada 
bundle esta formado por elementos relacionados entre sí a través de una función de similitud y que 
será único para algún atributo definido.
\section{Descripción}
Los \textbf{bundles} obtenidos de la búsqueda deben cumplir con las siguientes propiedades:
\begin{itemize}
  \item \textbf{Cubrimiento}: debe maximizar la cantidad de elementos de diferente propiedad tal 
que .
  \item \textbf{Compatibilidad}: todos los elementos internos deben ser similares en relación a 
algún atributo definido previamente.
  \item \textbf{Validez}: el costo total de los elementos no debe superar el presupuesto 
especificado.
  \item \textbf{Máximo}: debe ser el bundle más grande posible.
\end{itemize}
Una propiedad que debe cumplir la solución completa es:
\begin{itemize}
  \item \textbf{Diversificada}: Cada bundle debe ser \textquotedblleft diferente\textquotedblright  
de otro (más adelante se verá que se puede variar este parámetro) para de esta manera poder 
proporcionar una mayor exposición de los elementos que se encuentran en el dominio. Y además evitar 
de esa manera bundles iguales en la solución.
\end{itemize}
Como se mostró en la sección dos de \textit{\textquotedblleft Composite Retrieval of Diverse and 
Complementary Bundles\textquotedblright}\cite{compositeRetrival}, el problema puede ser reducido a 
uno más conocido \textbf{MAXIMUM EDGE SUBGRAPH} con lo cuál nos encontramos en un caso en el cuál 
encontrar una solución exacta se encuentra dentro de los problemas \textbf{NP-hard}.\\
Por tal motivo las técnicas usadas para generar soluciones a nuestras búsquedas serán heurísticas 
aproximadas.
\section{Primeros pasos}
Con los resultados obtenidos del paper \textit{\textquotedblleft Composite Retrieval of Diverse 
and Complementary Bundles\textquotedblright}\cite{compositeRetrival} se decidió implementar 
búsquedas de soluciones para la base de datos \textit{\textquotedblleft A Data-Driven Journey 
through Software Engineering Research\textquotedblright}\cite{dataDrive} utilizando su base de 
datos de papers con el objetivo de obtener conjunto de bundles complementarios.\\
La base de datos contiene información sobre cerca de $7777$ papers con $9865$ autores, además cuenta 
con un perfil para cada paper, llamado \texttt{topicProfile} que ofrece una clasificación en 
porcentajes de que tema se refiere cada paper. Los temas son:
\begin{multicols}{4}
  \begin{itemize}
  \item Adaptive Systems
  \item Algorithm
  \item Architectures
  \item Artificial Intelligence
  \item Autonomic Systems
  \item Concurrency
  \item Database
  \item Distributed Systems
  \item Education
  \item Embedded
  \item Empirical Software Engineering
  \item Evolution
  \item Formal Methods
  \item Hardware
  \item Human Computer Interaction
  \item Information Systms
  \item Knowledge Engineering
  \item Languages
  \item Maintenance
  \item Methodologies
  \item Metrics
  \item Models
  \item Operating Systems
  \item Performance
  \item Process
  \item Product Linesutilizar
  \item Program Analysis
  \item Program Comprehension
  \item Real Time Systems
  \item Reliability
  \item Requirements
  \item Reverse Engineering
  \item Security
  \item Services
  \item Software Quality
  \item Synthesis
  \item Testing
  \item Visualization
  \end{itemize}
  \label{int:tblTopicos}
\end{multicols}
La generación de soluciones de bundles se realizó, como propone el paper \textit{\textquotedblleft 
Composite Retrieval of Diverse and Complementary Bundles\textquotedblright}, maximizando la función 
objetivo: 
\begin{equation}\label{eq:fnObj}
\displaystyle FO = \sum_{1 \leq i \leq k} \displaystyle\sum_{u,v \in S_{i}} \gamma s(u,v)\ 
+\ \displaystyle\sum_{1 \leq i \leq j \leq k} (1-\gamma) (1 - \displaystyle\max_{u \in S_{1}, v 
\in S_{j}} s(u,v))
\end{equation} 
Dónde la función \textit{s} representa la similitud entre dos items y 
$\gamma$ ($0 < \gamma < 1$) es un parámetro para ponderar entre la calidad de un bundle (intra) y 
la separación entre ellos (inter).\\
A partir de información provista por la base de datos y con la definición de funciones de 
similitud, que se verán más adelante, para autores, papers e instituciones se generaron soluciones de 
conjuntos de bundles para los siguientes criterios:
\begin{itemize}
 \item Papers similares de distintos lugares.
 \item Autores similares de distintos universidades.
 \item Búsqueda de papers a partir de perfil específico.
 \item Instituciones similares de diferentes regiones
\end{itemize}
