\section{Descripción}
{\begin{small}%
\begin{flushright}%
\it
\end{flushright}%
\end{small}%
\vspace{.5cm}}
Partiendo del tema planteado en el paper \textit{\textquotedblleft Composite Retrieval of Diverse 
and Complementary Bundles\textquotedblright} en el cuál se buscan elementos complementarios que 
persiguen un mismo objetivo.

\section{Primeros pasos}
{\begin{small}%
\begin{flushright}%
\it
Un gran poder conlleva una gran responsabilidad.\\
Tío Ben.
\end{flushright}%
\end{small}%
\vspace{.5cm}}
Se utilizó la base de datos de \textit{\textquotedblleft A Data-Driven Journey through Software 
Engineering Research\textquotedblright} con el objetivo de obtener conjunto de bundles 
complementarios. La base de datos contiene información sobre cerca de 5000 papers y sus 7000 
autores, además cuenta con un perfil para cada paper, llamado \texttt{topicProfile} que nos 
ofrece una clasificación en porcentajes de que tema se refiere cada paper. Los temas incluidos son:
\begin{itemize}
 \item Adaptive Systems
 \item Algorithm
 \item Architectures
 \item Artificial Intelligence
 \item Autonomic Systems
 \item Concurrency
 \item Database
 \item Distributed Systems
 \item Education
 \item Embedded
 \item Empirical Software Engineering
 \item Evolution
 \item Formal Methods
 \item Hardware
 \item Human Computer Interaction
 \item Information Systms
 \item Knowledge Engineering
 \item Languages
 \item Maintenance
 \item Methodologies
 \item Metrics
 \item Models
 \item Operating Systems
 \item Performance
 \item Process
 \item Product Lines
 \item Program Analysis
 \item Program Comprehension
 \item Real Time Systems
 \item Reliability
 \item Requirements
 \item Reverse Engineering
 \item Security
 \item Services
 \item Software Quality
 \item Synthesis
 \item Testing
 \item Visualization
\end{itemize}
Para generar la soluciones de bundles se debe definir algún atributo de los items para comparar su 
similitud y otro para conocer su complementariedad. Además para la búsqueda de soluciones se 
define una variable \texttt{$\gamma$} ($0 < \gamma < 1$) usada para ponderar la selección de 
bundles y de esta manera obtener soluciones más cohesivas en las que cada bundle esta conformado 
por items muy parecidos entre sí o soluciones con bundles más dispersos en dónde es más importante 
que la relación entre los bundles sea lo más alejada posible\\.
A partir de información provista por la base de datos y con la definición de funciones de 
similitud, que se verán más adelante, tanto para autores y papers se generaron conjuntos de bundles 
para los siguientes criterios:
\begin{itemize}
 \item Papers de tópicos similares que se presentaron en conferencias de 
distintos lugares.
 \item Autores similares de distintos lugares.
\end{itemize}