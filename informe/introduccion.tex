\section{Descripción}
{\begin{small}%
\begin{flushright}%
\it
\end{flushright}%
\end{small}%
\vspace{.5cm}}
Con los resultados obtenidos del paper \textit{\textquotedblleft Composite Retrieval of Diverse 
and Complementary Bundles\textquotedblright} se decidió implementar búsquedas de soluciones para la 
base de datos \textit{\textquotedblleft A Data-Driven Journey through Software 
Engineering Research\textquotedblright}
\section{Primeros pasos}
{\begin{small}%
\begin{flushright}%
\it
Todo Pasa.\\
Julio G.
\end{flushright}%
\end{small}%
\vspace{.5cm}}
Se utilizó la base de datos de \textit{\textquotedblleft A Data-Driven Journey through Software 
Engineering Research\textquotedblright} con el objetivo de obtener conjunto de bundles 
complementarios. La base de datos contiene información sobre cerca de $7777$ papers y sus $9865$ 
autores, además cuenta con un perfil para cada paper, llamado \texttt{topicProfile} que ofrece una 
clasificación en porcentajes de que tema se refiere cada paper. Los temas son:
\begin{multicols}{4}
  \begin{itemize}
  \item Adaptive Systems
  \item Algorithm
  \item Architectures
  \item Artificial Intelligence
  \item Autonomic Systems
  \item Concurrency
  \item Database
  \item Distributed Systems
  \item Education
  \item Embedded
  \item Empirical Software Engineering
  \item Evolution
  \item Formal Methods
  \item Hardware
  \item Human Computer Interaction
  \item Information Systms
  \item Knowledge Engineering
  \item Languages
  \item Maintenance
  \item Methodologies
  \item Metrics
  \item Models
  \item Operating Systems
  \item Performance
  \item Process
  \item Product Lines
  \item Program Analysis
  \item Program Comprehension
  \item Real Time Systems
  \item Reliability
  \item Requirements
  \item Reverse Engineering
  \item Security
  \item Services
  \item Software Quality
  \item Synthesis
  \item Testing
  \item Visualization
  \end{itemize}
\end{multicols}
La generación de soluciones de bundles se realizó, como propone el paper \textit{\textquotedblleft 
Composite Retrieval of Diverse and Complementary Bundles\textquotedblright}, maximizando la función 
objetivo: $$\displaystyle\sum_{1 \leq i \leq k} \displaystyle\sum_{u,v \in S_{i}} \gamma s(u,v))\ 
+\ \displaystyle\sum_{1 \leq i \leq j \leq k} (1-\gamma) (1 - \displaystyle\max_{u \in S_{1}, v 
\in S_{j}} s(u,v))$$ Dónde la función \textit{s} representa la similitud entre dos items y 
$\gamma$ ($0 < \gamma < 1$) es un parámetro para ponderar entre la calidad de un bundle (intra) y 
la separación entre ellos (inter).\\
A partir de información provista por la base de datos y con la definición de funciones de 
similitud, que se verán más adelante, tanto para autores y papers se generaron conjuntos de bundles 
para los siguientes criterios:
\begin{itemize}
 \item Papers de tópicos similares que se presentaron en conferencias de 
distintos lugares.
 \item Autores similares de distintos lugares.
 \item Búsqueda de papers a partir de perfil específico.
\end{itemize}