\section{Motivación}
{\begin{small}%
\begin{flushright}%
\it An algorithm must be seen to be believed.\\Donald Knuth.
\end{flushright}%
\end{small}%
\vspace{.5cm}}
Este trabajo esta basado en el artículo \textit{\textquotedblleft Composite Retrieval of Diverse 
and Complementary Bundles\textquotedblright}\cite{compositeRetrival} en el que se propone 
que el resultado de las búsuqeda se devuelva en grupos de items, con la finalidad de que uno de estos grupos satisafaga
las expectativas del usuario. Cada uno de los grupos esta formado por elementos similares que difieren en algún atributo sin excederse del presupuesto.
El objetivo de las soluciones es generar grupos con items complementarios lo más cohesivos posibles y que los grupos entre si sean distintos.
De este modo al usuario se le ofrecen una gran variedad de items ordenados de forma lógica para facilitar la selección.\\

En una búsqueda típica los resultados obtenidos son lineales al criterio seleccionado y no otorgan respuestas que relacionen 
el criterio buscado con otros elementos que se relacionan. Si se buscan discos de música un resultado posible es:\\
\begin{itemize}
  \item Disco1 - Led Zeppelin.
  \item Disco2 - Led Zeppelin.
  \item Disco1 - The Who.
  \item Disco1 - Deep Purple.
  \item Disco1 - Cream.
  \item Disco1 - The Beatles.
  \item Disco1 - Queen.
  \item Disco1 - Rolling Stone.
\end{itemize}

Si se cuenta con un presupuesto para comprar discos, con este resultado lineal, el usuario debe realizar reiteradas búsquedas para 
encontrar los discos que que le interesen.\\

El resultado que se propone en ``Composite Retrieval of Diverse and Complementary Bundles'' esta diseñado para aquellas consultas 
que requieren obtener un conjunto de elementos que se relacionan como respuesta. Las técnicas tradicionales de 
clusterización la agrupación se hace por la similitud entre items. En el ejemplo de los discos con una clusterización tradicional,
donde la similitud sea el genero músical, seguramente se generen tantos clusters como géneros de discos existan y en cada cluster estén todos 
los discos de ese género. Con este resultado se deberá explorar todo el cluster para elegir los discos; donde la información es redundate 
porque algunos discos son tan parecidos que no se comprarían juntos.\\

En ``Composite Retrieval of Diverse and Complementary Bundles'' la clusterización se ajuste al presupuesto y los itmes sean complementarios, 
para que el usuario seleccionando un bundle (es el nombre que se le da al agrupamiento de items) cubra su necesidad. Con los discos 
se establece la complementariedad de algún atributo, como puede ser el origen de la banda, cada bundle es una opción de discos que el usuario 
puede comprar porque cumplen con el presupuesto, son del mismo género de música y hay variedad entre ellos. Una solución posible es:
\begin{itemize}
  \item Bundle 1:
  \begin{itemize}
    \item Disco1 - Led Zeppelin (Inglaterra).
    \item Disco1 - The Doors (Estados Unidos).
    \item Disco1 - Charly García (Argentina).
  \end{itemize}
  \item Bundle 2:
  \begin{itemize}
    \item Disco1 - The Who (Inglaterra).
    \item Disco1 - Kiss (Estados Unidos).
    \item Disco1 - Luis Alberto Spinetta (Argentina).
  \end{itemize}
\end{itemize}

Lo que se propone es otorgar un conjunto de \textbf{bundles} que cumplen con las siguientes propiedades:
\begin{itemize}
  \item \textbf{Cubrimiento}: Maximizar la cantidad de elementos en el bundle.
  \item \textbf{Compatibilidad}: Los elementos del bundle deben ser similares.
  \item \textbf{Validez}: El costo total de los elementos del bundle no debe superar el presupuesto.
  \item \textbf{Diversificada}: Los bundles entre si deben ser diversos.
\end{itemize}

\section{Instanciación}
Este trabajo consiste en implementar el diseño de resultados de ``Composite Retrieval of Diverse and Complementary Bundles'' 
para la base de datos de los papers de \textit{\textquotedblleft A Data-Driven Journey through Software Engineering Research\textquotedblright}\cite{dataDrive}.
La decisión de utilizar esta base de datos es por la completitud de la información y que el tamaño de la cantidad de elementos
que contiene requiere de una optimización de los algoritmos propuestos.\\

La base de datos contiene cerca de $7800$ papers con información de quienes son sus autores y dónde fueron presentados. 
Cada paper tiene un \texttt{topicProfile} que es una clasificación en porcentajes de cada tópico. De los $9800$ autores se conoce de que universidad 
es y de cada universidad a que región pertenece.
