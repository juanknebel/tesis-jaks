A cada búsquedas se le define los componentes:
\begin{itemize}
  \item \textbf{Similitud}: Función que dado dos items devuelve la similitud entre estos.
  \item \textbf{Costo}: Función que dado un item devuelve el costo del mismo.
  \item \textbf{Presupuesto}: El presupuesto que se tiene, el cual no podrá ser excedido por ningún bundle.
  \item \textbf{Complmentariedad}: Propiedad del item que es único en cada bundle.
\end{itemize}

Se utilizaron los siguientes criterios a las consultas que se hicieron en la base de datos, 
en los cuales la función costo para todos los elementos se defino con la constante 1 y el presupuesto de 5.
Por lo que en cada criterio resta definir la función de similitud y la propiedad de complementariedad.

\section{Papers de diferentes conferencias}\label{bus:papSimDisLug}
Generar una solución, en el que cada bundle contenga 
papers similares que se hayan presentado en distintas conferencias.\\
\begin{itemize}
  \item \textbf{Similitud}: Función que compara el perfil de cada paper.
  \item \textbf{Complmentariedad}: Lugar dónde fue presentado.
\end{itemize}

\section{Autores de distintas universidades}
Solución de bundles en el que cada uno contiene autores 
similares de distinta universidad de afiliación.\\
\begin{itemize}
  \item \textbf{Similitud}: Función que compara el perfil de los autores.
  \item \textbf{Complmentariedad}: Universidad de pertenencia del autor.
\end{itemize}

\section{Instituciones de diferentes regiones}
En este caso, la búsqueda es para instituciones de diferentes regiones. 
\begin{itemize}
  \item \textbf{Similitud}: Función que compara el perfil de las instituciones.
  \item \textbf{Complmentariedad}: Región de la institución.
\end{itemize}

El perfil de cada paper se calcula con el Topic Profile, el de los autores con la unión de los perfiles de los papers que participaron
y el perfil de las instituciones por la unión de los perfiles de los autores.
