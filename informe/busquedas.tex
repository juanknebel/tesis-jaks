Se propusieron las siguientes búsquedas.
\section{Papers similares, diferentes conferencias}\label{bus:papSimDisLug}
El objetivo de esta búsqueda es generar una solución de bundles, en el que cada bundle contenga 
papers de tópicos similares pero que se hayan presentado en distintas conferencias.\\
Para realizar esta búsqueda se debe definir que se entiende por papers de tópicos similares y cual 
es la presentación de la conferencia. Para ello en la base de datos ~\ref{int:baseDeDatos} cuenta 
con la información de la presentación del paper, \texttt{venue} de ahora en adelante, y los 
perfiles de cada paper ,\texttt{topicProfile} a partir de ahora.\\
La primer característica se utilizó para la complementariedad de dos papers, por lo que en cada 
bundle no habrá dos papes de un mismo \texttt{venue}. La segunda característica se utilizó para 
la similitud entre papers, la cuál se definió como la distancia entre los \texttt{topicProfile}.\\
Se decidió utilizar como distancia entre \texttt{topicProfile} al ángulo que forman dos 
vectores, los cuáles en cada posición de ellos pertenece a un porcentaje de un tópico particular.
\section{Autores similares, distintas universidades}
Esta búsqueda consiste en encontrar una solución de bundles en el que cada bundle contiene autores 
similares pero de distinta universidad de afiliación.\\
Para determinar la similitud entre los autores se creó un perfil de cada uno. Para lograrlo lo 
primero que se hizo fue tener en cuenta todos los perfiles de papers en los que participaron cada 
uno de ellos. Al igual que con los papers, el perfil de los autores se representó con un vector. 
Este vector se cálculo sumarizando los vectores de cada uno de los papers en el que participó.\\
El paso siguiente fue determinar la similitud de los autores, la primer aproximación fue similar al 
cálculo de similitud de los papers y calculamos los ángulos entre todos los vectores. Como similitud
alternativa decidimos restar el perfil de cada autor componente a componente y calcular el valor de 
su norma, así de esta forma a mayor norma mayor similitud entre autores.\\
La complementariedad de dos papers se definió al lugar de pertenencia del autor en cuestión.
\section{Papers similares, con un perfil específico}
En esta búsqueda se ingresan porcentajes de tópicos predefinidos y encontrar una solución de 
bundles de manera similar a ~\ref{bus:papSimDisLug} pero siempre teniendo en cuenta que cada 
bundle además de compartir sus similitudes, también deben guardar relación con los tópicos 
buscados.\\
Para esta búsqueda, se especificó un vector con porcentajes de temáticas mencionadas en 
~\ref{int:tblTopicos} y se cálculo su similitud contra todos los demás papers de la misma forma 
que ~\ref{bus:papSimDisLug}.
\section{Intituciones similares, diferentes regiones}
