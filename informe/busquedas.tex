Para la generación de soluciones de bundeles de una determinada búsqueda se debe definir una función de similitud para los items y la complementariedad
de estos. 
\section{Papers similares, diferentes conferencias}\label{bus:papSimDisLug}
El objetivo de esta búsqueda es generar una solución de bundles, en el que cada bundle contenga 
papers de tópicos similares pero que se hayan presentado en distintas conferencias.\\
Para realizar esta búsqueda se debe definir que se entiende por papers de tópicos similares y cual 
es la presentación de la conferencia. Para ello en la base de datos ~\ref{int:baseDeDatos} cuenta 
con la información de la presentación del paper, \texttt{venue} de ahora en adelante, y los 
perfiles de cada paper, \texttt{topicProfile} a partir de ahora.\\
La primer característica se utilizó para la complementariedad de dos papers, por lo que en cada 
bundle no habrá dos papes de un mismo \texttt{venue}. Cada paper se representó
con un vector de dimensión n en el que el valor de cada elemento es el porcentaje del \texttt{topicProfile}.\\ 
La función de similitud se define como la distancia angular entre los vectores.
\section{Autores similares, distintas universidades}
Esta búsqueda consiste en encontrar una solución de bundles en el que cada bundle contiene autores 
similares pero de distinta universidad de afiliación.\\
Para determinar la similitud entre los autores se creó un perfil de cada uno. Para lograrlo lo 
primero que se hizo fue tener en cuenta todos los perfiles de papers en los que participaron cada 
uno de ellos. Al igual que con los papers, el perfil de los autores se representó con un vector. 
Este vector se cálculo sumarizando los vectores de cada uno de los papers en el que participó.\\
Para la función de similitud se utilizaron dos definiciones. La primera, al igual que con los papers,
consiste en la distancia angular. La otra definición es calcular la norma del vector resultante de la resta
vectorial entre los vectores de cada autor.\\
La complementariedad de dos papers se definió al lugar de pertenencia del autor en cuestión.
\section{Papers similares, con un perfil específico}
A la búsqueda de ``papers similares de diferente conferencia'' se le agregó una variante para ponderar
soluciones que contengan \texttt{topicProfile} específicos.
Para ello se agregó un parámetro que con el porcentaje de cada \texttt{topicProfile} para la ponderación.
Encontrar una solución de bundles de manera similar a ~\ref{bus:papSimDisLug} teniendo en cuenta que cada 
bundle además de compartir sus similitudes, también deben guardar relación con los tópicos 
específicos.\\
\section{Intituciones similares, diferentes regiones}
En este caso, la búsqueda es para instituciones similares de diferentes regiones. Al igual que con las 
búsquedas de autores, el perfil de cada institución se determino a partir del \texttt{topicProfile} de los papers
de estas. La complementariedad por la el atributo de la región de la institución. 
Luego el procedimiento para buscar soluciones, fue idéntico al utilizado para el de los autores.