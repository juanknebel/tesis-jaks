%\begin{center}
%\large \bf \runtitulo
%\end{center}
%\vspace{1cm}
\chapter*{\runtitulo}

\noindent Las búsquedas tradicionales ofrecen soluciones que solo tienen en cuenta un solo atributo de los elementos y no la relación que estos tienen con el resto del universo. Las mismas suelen ofrecer una lista ordenada de resultados relacionadas con
el criterio utilizado. Ocasionando, muchas veces, re formular la consulta original para así lograr una solución adecuada al criterio de búsqueda elegido previamente.\\
Los algoritmos de agrupación clásicos o clustering generan conjuntos de soluciones, al igual que las búsquedas tradicionales, en los cuales los elementos dentro de cada conjunto (o cluster) tiene una relación directa con la búsqueda original y además se encuentran relacionados de alguna manera (generalmente una similitud o distancia). Aún así, éstas soluciones no tienen en cuenta ningún atributo que represente la complementaridad que existe entre los elementos, ocasionando que la solución encontrada contenga conjuntos de objetos muy similares sin diversidad. La relación entre los cluster no es sujeto de ningún análisis en estos algoritmos. En orden de lograr una solución con mayor diversidad (evitar cluster similares) y que cumpla las expectativas del usuario, debería existir un análisis entre los clusters.\\
Como respuesta a éste último comportamiento surge Composite Retrieval of Diverse and Complementary Bundles \cite{compositeRetrival}, su objetivo es agrupar elementos en bundles en los cuales los elementos dentro de ellos se encuentran relacionados internamente bajo algún criterio de similitud y a la vez sean complementarios de forma tal que satisfaga las expectativas del usuario y no tenga la necesidad de realizar una nueva intervención logrando así una mejor experiencia de búsqueda.\\
En este trabajo se tomaron las ideas ya desarrolladas previamente en \cite{compositeRetrival}, se propuso un nuevo algoritmo, nuevas mejoras y enfoques en orden de mejorar los resultados que se obtuvieron originalmente y los tiempos de ejecución para instancias más grandes. Las nuevas aplicaciones se aplicaron a la resolución de búsquedas sobre una base de datos de artículos científicos pertenecientes a la Ingeniería de Software \cite{dataDrive} y sobre una instancia conformada por atracciones turísticas de Europa.

\bigskip

\noindent\textbf{Palabras claves:} Composite Retrieval, Similitud, Complementaridad, Búsqueda Tabú, Generación de paquetes, Agrupamiento.