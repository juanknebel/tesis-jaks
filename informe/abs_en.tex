%\begin{center}
%\large \bf \runtitle
%\end{center}
%\vspace{1cm}
\chapter*{\runtitle}

\noindent Las búsquedas tradicionales nos ofrecen una solución que solo tiene en cuenta un atributo de los elementos y no la relación que éstos tienen con el resto del universo. Las mismas nos ofrecen un lista ordenada de resultados que cumplen con el criterio elegido en mayor o menos medida, ocasionando que muchas veces se necesite reformular la consulta original para así lograr la solución que el usuario quiere encontrar.\\
Los algoritmos de agrupación clásicos o clustering generan conjuntos de soluciones, que al igual que las búsquedas tradicionales, los elementos dentro de cada conjunto o cluster cumplen con el criterio elegido y además entre sí comparten alguna propiedad (generalmente una similitud o distancia), pero no se analiza ningún tipo de Complementaridad entre los elementos. La relación entre los cluster no es analizada ocacionando que entre ellos sean muy similares o diferentes dependiendo del universo en el cuál se encuentre trabajando.\\
Es por eso que surge \textbf{Composite Retrieval}, su objetivo es agrupar elementos en cluster bajo un mismo atributo, al mismo tiempo que éstos son complementarios entre sí por algún atributo definido previamente. A diferencia de las anteriores técnicas se permite elegir el grado de interdepencia entre los conjuntos generados permitiendo, según el caso, que los conjuntos que componen la solución sean más o menos parecidos facilitando al usuario la elección de un grupo de elementos que satisfagan su consulta original.

\bigskip

\noindent\textbf{Keywords:} Composite Retrieval, Similarity, Complementary, Tabú Search, Produce Bundles, Clustering.