\documentclass{beamer}
\let\Tiny=\tiny
\usepackage{beamerthemesidebar} % new
\hypersetup{hidelinks}
\usepackage[utf8]{inputenc}
\begin{document}
\title{Algoritmos para recuperación de ``ítems empaquetados''}
\author{Juan Andrés Knebel\\Amit Stein}
\institute[Inst.]{Universidad de Buenos Aires\\Facultad de Ciencias Exactas y Naturales\\Departamento de Computación}
\logo{\includegraphics[width=.2\textwidth,height=.1\textheight]{imagenes/logo.jpg}}

\frame{\titlepage} 

\frame{\frametitle{Contenido}\tableofcontents} 

\section{Comprando discos on-line}
\frame{\frametitle{Titulo} 
Un coleccionista de discos desea incorporar a su catalogo música de \underline{diferentes países}. Lo primordial en su búsqueda es que la elección final de discos pertenezca al mismo período de tiempo, pero si ninguna época en particular. Las consideraciones a tener en cuenta deben ser:
\begin{itemize}
  \item Mismo período.
  \item Diferentes países de origenes.
  \item Correspondan a varios países.
  \item No supere el presupuesto total de $\$ 100$.
\end{itemize}
}

\frame{
lalallala
}

\section{IR vs CR} 
\frame{\frametitle{Titulo}
\begin{itemize}
\item Poner la comparacion entre las busquedas comunes y las nuestras
\end{itemize} 
}

\section{Definici\'on formal} 
\frame{\frametitle{Titulo}
\begin{tabular}{c c c}
Primera formula \\ 
\pause 
Segunda formula \\  
\pause 
Tercera formula \\ 
\end{tabular} 
}

\section{Trabajos anteriores}
\frame{\frametitle{Titulo}
Poner cosas de los trabajos anteriores
}

\section{Nuevas propuestas}
\frame{\frametitle{Titulo}
Poner nuestras nuevas propuestas
}

\section{Experimentaci\'on}
\frame{\frametitle{Titulo}
Poner los resultados
}

\section{Conclusiones}
\frame{\frametitle{Titulo}
Poner las conclusiones
}

\section{Trabajo futuro}
\frame{\frametitle{Titulo}
Poner los trabajos futuros
}

\end{document}
