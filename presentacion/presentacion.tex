%%%%%%%%%%%%%%%%%%%%%%%%%%%%%%%%%%%%%%%%%
% Beamer Presentation
% LaTeX Template
% Version 1.0 (10/11/12)
%
% This template has been downloaded from:
% http://www.LaTeXTemplates.com
%
% License:
% CC BY-NC-SA 3.0 (http://creativecommons.org/licenses/by-nc-sa/3.0/)
%
%%%%%%%%%%%%%%%%%%%%%%%%%%%%%%%%%%%%%%%%%

%----------------------------------------------------------------------------------------
%	PACKAGES AND THEMES
%----------------------------------------------------------------------------------------

\documentclass{beamer}

\usepackage[utf8]{inputenc}

\mode<presentation> {

% The Beamer class comes with a number of default slide themes
% which change the colors and layouts of slides. Below this is a list
% of all the themes, uncomment each in turn to see what they look like.

%\usetheme{default}
%\usetheme{AnnArbor}
%\usetheme{Antibes}
%\usetheme{Bergen}
%\usetheme{Berkeley}
%\usetheme{Berlin}
%\usetheme{Boadilla}
%\usetheme{CambridgeUS}
%\usetheme{Copenhagen}
%\usetheme{Darmstadt}
%\usetheme{Dresden}
%\usetheme{Frankfurt}
%\usetheme{Goettingen}
%\usetheme{Hannover}
%\usetheme{Ilmenau}
%\usetheme{JuanLesPins}
%\usetheme{Luebeck}
\usetheme{Madrid}
%\usetheme{Malmoe}
%\usetheme{Marburg}
%\usetheme{Montpellier}
%\usetheme{PaloAlto}
%\usetheme{Pittsburgh}
%\usetheme{Rochester}
%\usetheme{Singapore}
%\usetheme{Szeged}
%\usetheme{Warsaw}

% As well as themes, the Beamer class has a number of color themes
% for any slide theme. Uncomment each of these in turn to see how it
% changes the colors of your current slide theme.

%\usecolortheme{albatross}
%\usecolortheme{beaver}
%\usecolortheme{beetle}
%\usecolortheme{crane}
%\usecolortheme{dolphin}
%\usecolortheme{dove}
%\usecolortheme{fly}
%\usecolortheme{lily}
%\usecolortheme{orchid}
%\usecolortheme{rose}
%\usecolortheme{seagull}
%\usecolortheme{seahorse}
%\usecolortheme{whale}
%\usecolortheme{wolverine}

%\setbeamertemplate{footline} % To remove the footer line in all slides uncomment this line
%\setbeamertemplate{footline}[page number] % To replace the footer line in all slides with a simple slide count uncomment this line

%\setbeamertemplate{navigation symbols}{} % To remove the navigation symbols from the bottom of all slides uncomment this line
}

\usepackage{graphicx} % Allows including images
\usepackage{booktabs} % Allows the use of \toprule, \midrule and \bottomrule in tables
\let\Tiny=\tiny

%----------------------------------------------------------------------------------------
%	TITLE PAGE
%----------------------------------------------------------------------------------------

\title[Ítems empaquetados]{Algoritmos para recuperación de ``ítems empaquetados''} % The short title appears at the bottom of every slide, the full title is only on the title page

\author{Juan Andrés Knebel \& Amit Stein} % Your name
\institute[UBA] % Your institution as it will appear on the bottom of every slide, may be shorthand to save space
{
Universidad de Buenos Aires\\Facultad de Ciencias Exactas y Naturales\\Departamento de Computación \\ % Your institution for the title page
\begin{center}
\includegraphics[width=2cm,height=2cm,keepaspectratio]{imagenes/logofcen.pdf}                                                                             \end{center}
}
\pgfdeclareimage[height=.1\textheight]{fcen}{imagenes/logofcen.pdf}
\pgfdeclareimage[height=.1\textheight]{dc}{imagenes/logo.jpg}
\logo{\pgfuseimage{dc}}

\date{} % Date, can be changed to a custom date

\begin{document}
\begin{frame}
\titlepage % Print the title page as the first slide
\end{frame}

\begin{frame}
\frametitle{Contenido} % Table of contents slide, comment this block out to remove it
\tableofcontents % Throughout your presentation, if you choose to use \section{} and \subsection{} commands, these will automatically be printed on this slide as an overview of your presentation
\end{frame}

\section{Comprando discos on-line}
\frame{\frametitle{Comprando discos on-line} 
Un coleccionista de discos desea incorporar a su catalogo musical artistas provenientes de \underline{diferentes países}. Lo primordial en su búsqueda es que la elección final de discos pertenezca al mismo período de tiempo, sin importar la época a la que pertenecen. Las consideraciones a tener en cuenta deben ser:
\begin{itemize}
  \item Mismo período.
  \item Diferentes países de origenes.
  \item Correspondan a varios países.
  \item No supere el presupuesto total de $\$ 100$.
\end{itemize}
}

\frame{\frametitle{Comprando discos on-line}
Para que el coleccioniste encuentre su conjunto de discos podría realizar las siguientes consutlas:
\pause

}
%\frame[shrink=20]{\frametitle{Comprando discos on-line} 
%lalallala
%}

\section{IR vs CR} 
\frame{\frametitle{Titulo}
\item Poner la comparacion entre las busquedas comunes y las nuestras
}

\section{Definici\'on formal} 
\frame{\frametitle{Titulo}
\begin{tabular}{c c c}
Primera formula \\ 
\pause 
Segunda formula \\  
\pause 
Tercera formula \\ 
\end{tabular} 
}

\section{Trabajos anteriores}
\frame{\frametitle{Titulo}
Poner cosas de los trabajos anteriores
}

\section{Nuevas propuestas}
\frame{\frametitle{Titulo}
Poner nuestras nuevas propuestas
}

\section{Experimentaci\'on}
\frame{\frametitle{Titulo}
Poner los resultados
}

\section{Conclusiones}
\frame{\frametitle{Titulo}
Poner las conclusiones
}

\section{Trabajo futuro}
\frame{\frametitle{Titulo}
Poner los trabajos futuros
}

\end{document}
